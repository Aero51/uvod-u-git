\chapter*{Instalacija, konfiguracija i prvi projekt}
\addcontentsline{toc}{chapter}{Instalacija, konfiguracija i prvi projekt}

\section*{Instalacija}
\addcontentsline{toc}{section}{Instalacija}

Instalacija gita je relativno jednostavna. Ukoliko ste na na nekom od linuxoidnih operativnih sustava -- sigurno postoji paket za instalaciju. 
Za sve ostale, postoje jednostavne instalacije, a svi linkovi su dostupni na službenim web stranicama\footnote{http://git-scm.com/download}.

Važno je napomenuti da su to samo \emph{osnovni paketi}. 
Oni će biti dovoljni za primjere koji slijede, no za mnoge specifične scenarije postoje dodaci s kojima se git naredbe "obogaćuju" i omogućuju nove stvari.

\section*{Prvi git repozitorij}
\addcontentsline{toc}{section}{Prvi git repozitorij}

Ukoliko ste naviknuti na TFS, subversion ili CVS onda si vjerojatno zamišljate da je za ovaj korak potrebno neko računalo na kojem je instaliran poseban servis ("daemon") i kojemu je potrebno dati do znanja da želite imati novi repozitorij na njemu.
Vjerojatno mislite i to da je sljedeći korak preuzeti taj projekt s tog udaljenog računala/servisa.
Neki sustavi taj korak nazivaju \emph{checkout}, neki \emph{import}, a u gitu je to \emph{clone}, iliti kloniranje projekta.

S gitom je jednostavnije. 
\textbf{Apsolutno svaki direktorij može postati git repozitorij.}
Ne mora \emph{uopće} postojati udaljeni server i neki centralni repozitorij kojeg koriste (i) ostali koji rade na projektu.
Ako vam je to neobično, onda se spremite, jer stvar je još čudnija -- ako već postoji udaljeni repozitorij s kojeg preuzimate izmjene od drugih programera on ne mora biti jedan jedini.
\textbf{Mogu postojati deseci takvih udaljenih repozitorija}, sami ćete odlučiti na koje ćete "slati" svoje izmjene i s kojih preuzimati izmjene.
I vlasnici tih udaljenih repozitorija imaju istu slobodu kao i vi, mogu samo odlučiti čije izmjene će preuzimati kod sebe i kome slati svoje izmjene.

Pomisliti ćete da je rezultat anarhija u kojoj se ne zna tko pije, tko plače, a tko plaća. 
Nije tako.
Stvari, uglavnom, funkcioniraju bez većih problema.

Idemo sad na prvi i najjednostavniji korak -- stvoriti ćemo novi direktorij \\ \verb+moj-prvi-projekt+ i stvoriti novi repozitorij u njemu:

\input{git_output/git_init}

I to je to. 

Ukoliko idete pogledati kakva se to čarolija desila u s tim \verb+git init+, otkriti ćete da je stvoren direktorij \verb+.git+.
U principu, cijela povijest, sve grane, čvorovi i komentari, apsolutno sve vezano uz repozitorij se čuva u tom direktoriju.
Zatreba li nam ikad sigurnosna kopija cijelog repozitorija -- sve što treba napraviti je da sve lokalne promjene spremimo (\emph{commit}amo) u git i spremimo negdje arhivu (\verb+.zip+, \verb+.bz2+, \dots) s tim \verb+.git+ direktorijem.

\section*{Git naredbe}
\addcontentsline{toc}{section}{Git naredbe}

U prethodnom primjeru smo u našem direktoriju inicijalizirali git repozitorij s naredbom \verb+git init+.
Općenito, git naredbe uvijek imaju sljedeći format:

\gitoutput{
\color{blue}{git $<$naredba$>$ $<$opcija1$>$ $<$opcija2$>$ \dots}
}

Izuzetak je pomoćni grafički program s kojim se može pregledavati povijest projekta, a koji dolazi u instalaciji s gitom -- \verb+gitk+.

Za svaku git naredbu možemo dobiti \emph{help} s:

\gitoutputcommand{
git help $<$naredba$>$
}

Na primjer, \verb+git help init+ ili \verb+git help config+.

\section*{Osnovna konfiguracija}
\addcontentsline{toc}{section}{Osnovna konfiguracija}

Ima nekoliko osnovnih stvari koje moramo konfigurirati da bismo nastavili normalan rad. 
Sva git konfiguracija se postavlja pomoću naredbe \verb+git config+. 
Postavke mogu biti \textbf{lokalne} (odnosno vezane uz jedan jedini projekt) ili \textbf{globalne} (vezane uz korisnika na računalu).

Globalne postavke se postavljaju s:

\gitoutputcommand{git config --global $<$naziv$>$ $<$vrijednost$>$}

\dots{}i one se spremaju u datoteku \verb+.gitconfig+ u vašem \emph{home} direktoriju.

Lokalne postavke se spremaju u \verb+.git+ direktorij u direktoriju koji sadrži vaš repozitorij, a tada je format naredbe \verb+git config+:

\gitoutputcommand{git config $<$naziv$>$ $<$vrijednost$>$}

Za normalan rad na nekom projektu, drugi korisnici trebaju znati tko je točno radio koje izmjene na kodu (\emph{commit}ove).
Zato trebamo postaviti ime i email adresu koja će u povijesti projekta biti "zapamćena" uz svaku našu spremljenu izmjenu:

\input{git_output/git_config_1}

Imamo li neki repozitorij koji je vezan za posao, i možda se ne želimo identificirati sa svojom privatnom domenom, tada \emph{u tom direktoriju} trebamo postaviti drukčije postavke:

\input{git_output/git_config_2}

Na taj način će \emph{email} adresa \verb+ana.anic@poslodavac.hr+ figurirati samo u povijesti tog projekta.

Postoje mnoge druge konfiguracijske postavke, no ja vam preporučam da za početak postavite barem dvije \verb+color.ui+ i \verb+merge.tool+.

S \verb+color.ui+ možete postaviti da ispis git naredbi bude obojan:

\input{git_output/git_config_3}

\verb+merge.tool+ određuje koji će se program koristiti u slučaju do \textbf{konflikta} (o tome više kasnije). Ja koristim \verb+gvimdiff+:

\input{git_output/git_config_4}

\section*{.gitignore}
\addcontentsline{toc}{section}{.gitignore}

Prije ili kasnije će se dogoditi situacija da u direktoriju s repozitorijem imamo datoteke koje ne želimo spremati u povijest projekta.
To su, na primjer, konfiguracijske datoteke za različite editore ili klase koje nastaju kompajliranjem (\verb+.class+ za javu, \verb+.pyc+ za python, \verb+.o+ za C, i sl.).
U tom slučaju, trebamo nekako gitu dati do znanja da takve datoteke ne treba nikad snimati.
Otvorite novu datoteku naziva \verb+.gitignore+ u glavnom direktoriju projekta (ne nekom od poddirektorija) i jednostavno unesimo sve ono što ne treba biti dio povijesti projekta.

Ukoliko ne želimo \verb+.class+, \verb+.swo+, \verb+.swp+ datoteke i sve ono što se nalazi u direktoriju \verb+target/+ -- naša \verb+.gitignore+ datoteka će izgledati ovako:

\gitoutputcommand{%
\color{gray}{\# Vim privremene datoteke:}\\
\color{black}{*.swp}\\
\color{black}{*.swo}\\
\color{gray}{\# Java kompajlirane klase:}\\
\color{black}{*.class}\\
\color{gray}{\# Output direktorij s rezultatima kompajliranja i builda:}\\
\color{black}{target/*}%
}

Sve one linije koje započinju sa znakom \verb+#+ su komentari i git će se ponašati kao da ne postoje.

\dots{}i sad smo spremni početi nešto i raditi s našim projektom\dots

%\section*{}
%\addcontentsline{toc}{section}{}

