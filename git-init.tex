\chapter*{Instalacija, konfiguracija i prvi projekt}
\addcontentsline{toc}{chapter}{Instalacija, konfiguracija i prvi projekt}

\begin{itemize}
   \item .git
   \item .gitignore
   \item komande
   \item Username, password
   \item boje
   \item ostale konfiguracije
   \item lokalna i globalna konfiguracija
\end{itemize}

\section*{Instalacija}
\addcontentsline{toc}{section}{Instalacija}

Instalacija gita je relativno jednostavna. Ukoliko ste na na nekom od linuksoidnih operativnih sustava -- sigurno postoji paket za instalaciju gita. 
Za sve ostale, postoje jednostavne instalacije, a svi linkovi su dostupni na službenim web stranicama\footnote{http://git-scm.com/download}.

Važno je napomenuti da su to samo \emph{osnovni paketi}. 
Oni će biti dovoljni za primjere koji slijede. 
No, za mnoge specifične scenarije postoje dodaci s kojima se git naredbe "obogaćuju" i omogućavaju nove stvari.

\section*{Prvi git repozitorij}
\addcontentsline{toc}{section}{Prvi git repozitorij}

Ukoliko ste naviknuti na TFS, subversion ili CVS onda si vjerojatno zamišljate da je za ovaj korak potrebno neko računalo na kojem je instaliran poseban servis ("daemon") i kojemu je potrebno nekako dati do znanja da želite imati novi repozitorij na njemu.
Vjerojatno mislite i to da je sljedeći korak nekako preuzeti taj projekt s tog udaljenog računala/servisa.

S gitom je jednostavnije. 
\emph{Apsolutno svaki direktorij može postati git repozitorij.}
Ne mora \emph{uopće} postojati udaljeni server i neki centralni repozitorij kojeg koriste (i) ostali koji rade na projektu.
Ako Vam je to neobično, stvar je još čudnija -- ako već postoji udaljeni repozitorij s kojeg preuzimate izmjene od drugih programera on ne mora biti jedan jedini.
\emph{Mogu postojati deseci takvih udaljenih repozitorija, sami ćete odlučiti na koje ćete "slati" svoje izmjene i s kojih preuzimati izmjene.}

No, idemo sad na prvi i najjednostavniji korak -- stvoriti ćemo novi direktorij \verb+moj-prvi-projekt+ i stvoriti novi repozitorij u njemu:

\begin{verbatim}
$ mkdir moj-prvi-projekt
$ cd moj-prvi-projekt
$ git init
Initialized empty Git repository in /home/user/moj-prvi-projekt/.git/
$ 
\end{verbatim}

I to je to. 

\section*{Git naredbe}
\addcontentsline{toc}{section}{Git naredbe}

U prethodnom primjeru smo u našem direktoriju inicijalizirali git repozitorij s naredbom \verb+git ini+.
Općenito, git naredbe uvijek imaju sljedeći format:

\begin{verbatim}
git <komanda> <opcija1> <opcija2> ...
\end{verbatim}

Izuzetak je pomoćni grafički program s kojim se može pregledavati povijest projekta, a koji dolazi u instalaciji s gitom -- \verb+gitk+.

\section*{Osnovna konfiguracija}
\addcontentsline{toc}{section}{Osnovna konfiguracija}

Ima nekoliko osnovnih stvari koje \emph{morate} konfigurirati da bi ste nastavili normalan rad

%\section*{}
%\addcontentsline{toc}{section}{}

