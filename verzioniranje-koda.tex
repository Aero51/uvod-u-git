\chapter*{Verzioniranje koda i osnovni pojmovi}
\addcontentsline{toc}{chapter}{Verzioniranje koda i osnovni pojmovi}

\section*{Što je to verzioniranje koda?}
\addcontentsline{toc}{section}{Što je to verzioniranje koda?}

S problemom verzioniranja koda ste se sreli kad ste prvi put napisali program koji rješava neki konkretan problem. 
Bilo da je to neka jednostavna web aplikacija, CMS\footnote{Content Management System}, komandnolinijski pomoćni programčić ili kompleksni ERP\footnote{Enterprise Resource Planning}.

Svaka aplikacija koja ima \textbf{stvarnog} korisnika kojemu rješava neki \textbf{stvarni} problem ima i \textbf{korisničke zahtjeve}.
Taj korisnik možemo biti mi sami, može biti neko hipotetsko tržište (kojemu planiramo prodati riješenje) ili može biti naručitelj.
Korisničke zahtjeve ne možemo nikad točno predvidjeti u trenutku kad krenemo pisati program.
Možemo satima, danima i mjesecima sjediti s budućim korisnicima i planirati što će sve naša aplikacija imati, ali kad korisnik sjedne pred prvu verziju aplikacije, čak i ako je pisana točno prema njegovim specifikacijama, on će naći nešto što ne valja. 
Radi li se o nekoj maloj izmjeni -- možda ćemo ju napraviti na licu mjesta.
No, možda ćemo trebati otići kući, potrošiti nekoliko dana i napraviti \textbf{novu verziju}.

Desiti će se, na primjer, da korisniku damo na testiranje verziju \texttt{1.0}.
On će istestirati i, naravno, naći nekoliko sitnih stvari koje treba ispraviti.
Otići ćemo kući, ispraviti ih, napraviti verziju \texttt{1.1} s kojom će klijent biti zadovoljan.
Nekoliko dana kasnije, s malo više iskustva u radu s aplikacijom, on zaključuje kako sad ima \textbf{bolju} ideju kako je trebalo ispraviti verziju \textbf{1.0}.
Sad, dakle, treba "baciti u smeće" posao koji smo radili za \texttt{1.1}, vratiti se na \texttt{1.0} i od nje napraviti, npr. \texttt{1.1b}.

Grafički bi to izgledalo ovako nekako:

\input{graphs/primjer_s_klijentom}

U trenutku kad je korisnik odlučio da mu trenutna verzija ne odgovara -- trebamo ju "baciti u smeće", vratiti se jedan korak unazad u povijest projekta i započeti novu verziju -- odnosno novu \textbf{granu projekta}.
I nastaviti projekt s tom izmjenom.

I to je samo jedan od mnogih složenih scenarija kakvi se dešavaju s aplikacijama.

\section*{Linearno verzioniranje koda}
\addcontentsline{toc}{section}{Linearno verzioniranje koda}

Linearni pristup verzioniranju koda se najbolje može opisati sljedećom ilustracijom:

\input{graphs/linearni_model}

To je idealna situacija u kojoj točno unaprijed znamo kako aplikacija treba izgledati. 
Započnemo projekt s nekim početnim stanjem \emph a, pa napravimo izmjene \emph b, \emph c, \dots sve dok ne zaključimo da smo spremni izdati prvu verziju za javnost.
I proglasimo to verzijom \texttt{1.0}. 

Postoje mnoge varijacije ovog linearnog modela, jedna česta je:

\input{graphs/linearni_model_2}

Ona je česta u situacijama kad nemate kontrolu nad time koja je točno verzija programa instalirana kod klijenta. 
S web aplikacijama to nije problem, jer vi jednostavno možete aplikaciju prebaciti nas server i odmah svi klijenti koriste novu verziju.
Međutim, ukoliko je vaš program klijentima "spržen" na CD i takav poslan klijentu -- može se desiti da jedan ima instaliranu verziju \texttt{1.0}, a drugi \texttt{2.0}.

I sad, što ako klijent koji je zadovoljan sa starijom verzijom programa otkrije \textbf{bug}?
I zbog nekog razloga ne želi preći na novu verziju?
U tom slučaju, morate imati neki mehanizam kako da se privremeno vratite na staru verziju, ispravite problem, izdate "novu verziju stare verzije". 
Pošaljete je klijentu i nakon toga, vratite se na zadnju verziju i tamo nastavite rad na svoj projektu.

\section*{Grafovi, grananje i spajanje grana}
\addcontentsline{toc}{section}{Grafovi, grananje i spajanje grana}

Prije nego nastavimo s gitom, par riječi o grafovima. 
U ovoj knjižici ćete vidjeti puno grafova kao što su u primjerima s linearnim verzioniranjem koda. 
Zato ćemo se na trenutak zadržati na jednom takvom grafu:

\input{graphs/primjer_s_granama_i_spajanjima}

Svaka točka grafa je stanje projekta. 
Projekt s gornjim grafom je započeo s nekim početnim stanjem \emph a.
Programer je napravio nekoliko izmjena i \textbf{snimio} novo stanje \emph b, zatim \emph c, pa sve do \emph h.

Primijetite da je ovakav graf stanje povijesti projekta, ali iz njega ne možemo zaključiti kojim redom su čvorovi nastajali.
Neke stvari možemo zaključiti -- vidi se, na primjer, da je $d$ nastao nakon $c$, $e$ nakon $d$ ili $z$ nakon $y$.
Ne znamo je li prije nastao $c$ ili $x$.
Ili, čvor \emph 1 je sigurno nastao nakon $g$, no iz grafa se ne vidi je li nastao prije $x$ ili nakon $x$ (vidite li kako?).

Evo, na primjer, jedan scenarij kako je navedeni graf mogao nastati:

\input{graphs/primjer_s_granama_i_spajanjima_1}

Programer je započeo aplikaciju, snimio stanje \emph a, \emph b i \emph c i tada se sjetio da ima neki problem kojeg može riješiti na dva načina, vratio se na \emph b i napravio novu granu. Tamo je napravio izmjene $x$ i $y$:

\input{graphs/primjer_s_granama_i_spajanjima_2}

Zatim se sjetio jedne sitnice koju je mogao napraviti u \emph{originalnoj} verziji; vratio se tamo i dodao \emph d:

\input{graphs/primjer_s_granama_i_spajanjima_3}

Nakon toga se vratio na svoj prvotni eksperiment, i odlučio se da bi bilo dobro tamo imati izmjene koje je napravio u \emph c i \emph d.
Tada je \emph{preuzeo} te izmjene u svoju granu:

\input{graphs/primjer_s_granama_i_spajanjima_4}

Na eksperimentalnoj grani je napravio još jednu izmjenu \emph q.
I tada je odlučio pustiti taj eksperiment malo sa strane i raditi malo na glavnoj grani.
Vratio se na originalnu granu i tamo napredovao s \emph e i \emph f. 

\input{graphs/primjer_s_granama_i_spajanjima_5}

Sjetio se da bi mu sve izmjene iz eksperimentalne grane odgovarale u originalnoj, \emph{preuzeo} ih u početnu granu:

\input{graphs/primjer_s_granama_i_spajanjima_6}

I, zatim je nastavio, stvorio još jednu eksperimentalnu granu(\emph 1, \emph 2, \emph 3, \dots)\dots I tako dalje\dots

\input{graphs/primjer_s_granama_i_spajanjima}

Uočite da izmjena \emph w nije nikad završila u glavnoj grani (na dnu). 

Jedna od velikih prednosti gita je lakoća stvaranja novih grana i preuzimanja izmjena iz jedne u drugu granu. 
Tako je programerima jednostavno u nekom trenutku ovako razmišljati i postupiti: \emph{"Ovaj problem bih mogao riješiti na dva različita načina. Pokušati ću i jedan i drugi, i onda vidjeti koji mi bolje ide."}. Za svaku verziju će napraviti posebnu granu i napredovati prema osjećaju.

Druga velika prednost čestog grananja u programima je kad se dodaje neka nova funkcionalnost koja zahtijeva puno izmjena, a ne želime te izmjene odmah stavljati u glavnu granu programa:

\input{graphs/primjer_s_dugotrajnom_granom}

Trebamo pripaziti da redovito izmjene iz glavne grane programa preuzimamo u sporednu, tako da razlike u kodu ne budu prevelike. 
Te izmjene su na grafu označene sivim strelicama.

Kad novu funkcionalnost završimo, u glavnoj granu treba preuzeti sve izmjene iz sporedbne (crvena strelica).

Na taj način ćemo često imati ne samo dvije grane (glavnu i sporednu) nego nekoliko njih (pa i do nekoliko desetaka). 
Imati ćemo posebne grane za različite nove funkcionalnosti, posebne grane za eksperimente, posebne grane u kojima ćemo isprobavati izmjene koje su napravili drugi programeri, posebne grane za ispravljanje pojedinih \emph{bug}ova, \dots

Osnovna ideja ove knjižice \textbf{nije} da vas uči kako je najbolje organizirati povijest projekta, odnosno kako granati, kad i kako preuzimati izmjene iz pojedinih grana. Osnovna ideja je da vas nauči \textbf{kako} to napraviti s gitom. 

\section*{Mit o timu i sustavima za verzioniranje}
\addcontentsline{toc}{section}{Mit o timu i sustavima za verzioniranje}

Prije nego nastavimo, htio bih ovdje samo srušiti jedan mit. 
Taj mit glasi ovako nekako: "\emph{Sustavi za verzioniranje koda su potrebni kad na nekom projektu radi više ljudi}".

Vjerujte mi, ovo nije istina.

\textbf{Posebno} to nije istina za git i druge distribuirane sustave koji su namijenjeni često grananju.
Ukoliko se natjerate da o projektu razmišljate kao o jednom usmjerenom grafu i tako posebne stvari radite u posebnim granama, to višestruko olakšava samo razmišljanje o projektu.
Ako imate jedan direktorij sa cijelim projektom i u kodu imate paralelno izmjene od tri različite stvari koje radite istovremeno, onda imate problem.
Ostatak ove knjižice bi vas trebao uvjeriti u to\dots

No, nemojte pročitati knjigu i reći "\emph{Nisam baš uvjeren}". 
Probajte git\footnote{Ako vam se git i ne svidi, probajte barem mercurial.} na nekoliko tjedana.

% \section*{}
% \addcontentsline{toc}{section}{}
