\chapter*{Verzioniranje koda i osnovni pojmovi}
\addcontentsline{toc}{chapter}{Verzioniranje koda i osnovni pojmovi}

\begin{itemize}
   \item instalacija
   \item .git
   \item .gitignore
   \item komande
\end{itemize}

\section*{Što je to verzioniranje koda?}
\addcontentsline{toc}{section}{Što je to verzioniranje koda?}

S problemom verzioniranja koda ste se sreli kad ste prvi put napisali program koji riješava neki konkretan problem. 
Bilo da je to neka jednostavna web aplikacija, CMS\footnote{Content Management System}, komandnolinijski pomoćni programčić ili kompleksni ERP\footnote{Enterprise Resource Planning}.

Svaka aplikacija koja ima \textit{stvarnog} korisnika kojemu rješava neki \textit{stvarni} problem ima i \textbf{korisničke zahtjeve}.
Taj korisnik možete biti Vi sami, može biti neko hipotetsko tržište (kojemu planirate prodati riješenje) ili može biti naručioc.
Korisničke zahtjeve ne možete nikad točno procijeniti u trenutku kad krenete pisati program.
Možete satima, danima i mjesecima sjediti s budućim korisnicima i planirati što će sve Vaša aplikacija imati, ali kad korisnik sjedne pred prvu verziju aplikacije, čak i ako je pisana točno prema njegovim specifikacijama, on će naći nešto što ne valja. 
Radi li se o nekoj maloj izmjeni -- možda ćete ju napraviti na licu mjesta. No, možda ćete trebati otići doma, potrošiti nekoliko dana i napraviti \textbf{novu verziju}.

Desiti će se, na primjer, da korisniku date da isproba verziju \ttfamily{1.0}.
On će to isprobati, naći nekoliko sitnih stvari koje treba ispraviti.
Otići ćete kući, ispraviti ih, napraviti verziju \ttfamily{1.1} s kojom će klijent biti zadovoljan.
Nekoliko dana kasnije, s malo više iskustva u radu s aplikacijom, on zaključuje kako sad ima \textit{bolju} ideju kako je trebalo ispraviti verziju \ttfamily{1.0}.
Vi sad, dakle, trebate "baciti u smeće" posao koji ste radili za \ttfamily{1.1}, vratiti se na \ttfamily{1.0} i od nje napraviti, npr. \ttfamily{1.1b}.

Grafički bi to izgledalo ovako nekako:

TODO



\section*{Linearno verzioniranje koda}
\addcontentsline{toc}{section}{Linearno verzioniranje koda}

Linearni pristup verzioniranju koda se najbolje može opisati sljedećom ilustracijom:

% \begin{floatingfigure}[c]{5cm}
\epsfig{figure=linearni-model.eps}
% \end{floatingfigure}

% \begin{floatingfigure}[lrp]{5cm}
% \centering
% \epsfig{figure=linearni-model.eps}
% \caption{fig:default}
% \label{fig:figlabel}end
% \end{floatingfigure}

\section*{Distribuirano verzioniranje koda}
\addcontentsline{toc}{section}{Distribuirano verzioniranje koda}

% \section*{}
% \addcontentsline{toc}{section}{}
