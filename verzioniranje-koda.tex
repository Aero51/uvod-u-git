\chapter*{Verzioniranje koda i osnovni pojmovi}
\addcontentsline{toc}{chapter}{Verzioniranje koda i osnovni pojmovi}

\section*{Što je to verzioniranje koda?}
\addcontentsline{toc}{section}{Što je to verzioniranje koda?}

S problemom verzioniranja koda ste se sreli kad ste ptrvi put napisali program koji riješava neki konkretan problem. 
Bilo da je to neka jednostavna web aplikacija, CMS\footnote{Content Management System}, komandnolinijski pomoćni programčić ili kompleksni ERP\footnote{Enterprise Resource Planning}.

Svaka aplikacija koja ima \textit{stvarnog} korisnika kojemu rješava neki \textit{stvarni} problem ima i \textbf{korisničke zahtjeve}.
Taj korisnik možete biti Vi sami, može biti neko hipotetsko tržište (kojemu planirate prodati riješenje) ili može biti naručioc.
Korisničke zahtjeve ne možete nikad točno predvidjeti u trenutku kad krenete pisati program.
Možete satima, danima i mjesecima sjediti s budućim korisnicima i planirati što će sve Vaša aplikacija imati, ali kad korisnik sjedne pred prvu verziju aplikacije, čak i ako je pisana točno prema njegovim specifikacijama, on će naći nešto što ne valja. 
Radi li se o nekoj maloj izmjeni -- možda ćete ju napraviti na licu mjesta. No, možda ćete trebati otići doma, potrošiti nekoliko dana i napraviti \textbf{novu verziju}.

Desiti će se, na primjer, da korisniku date da isproba verziju \texttt{1.0}.
On će istestirati i, naravno, naći nekoliko sitnih stvari koje treba ispraviti.
Otići ćete kući, ispraviti ih, napraviti verziju \texttt{1.1} s kojom će klijent biti zadovoljan.
Nekoliko dana kasnije, s malo više iskustva u radu s aplikacijom, on zaključuje kako sad ima \textit{bolju} ideju kako je trebalo ispraviti verziju \texttt{1.0}.
Vi sad, dakle, trebate "baciti u smeće" posao koji ste radili za \texttt{1.1}, vratiti se na \texttt{1.0} i od nje napraviti, npr. \texttt{1.1b}.

Grafički bi to izgledalo ovako nekako:

\input{graphs/primjer_s_klijentom}

U trenutku kad je korisnik odlučio da mu trenutna verzija ne odgovara -- trebate ju "baciti u smeće", vratiti se jedan korak unazad u povijest projekta i započeti novu verziju -- odnosno novu |textbf{granu projekta}. I nastaviti projekt s tom izmjenom.

I to je samo jedan od mnogih složenih scenarija kakvi se dešavaju s aplikacijama koje korisnici koriste.

\section*{Linearno verzioniranje koda}
\addcontentsline{toc}{section}{Linearno verzioniranje koda}

Linearni pristup verzioniranju koda se najbolje može opisati sljedećom ilustracijom:

\input{graphs/linearni_model}

To je idealna situacija u kojoj točno unaprijed znate kako aplikacija treba izgledati, počete projekt s nekim početnim stanjem \emph{a}, pa napravite izmjene \emph{b}, \emph{c}, \dots sve dok ne zaključite kako ste spremni izdati prvu verziju za javnost.
Proglasite to verzijom \texttt{1.0}. 

Postoje mnoge varijacije ovog lineranog modela, jedna česta je:

\input{graphs/linearni_model_2}

Ona je česta u situacijama kad nemate kontrolu time koja je točno verzija programa instalirana kod klijenta. 
S web aplikacijama to nije problem, jer Vi jednostavno možete aplikaciju prebaciti nas serveru i odmah svi klijenti koriste novu verziju.
Međutim, ukoliko je Vaš program klijentima "spržen" na CD-u i kao takav poslan klijentu -- može se desiti da jedan ima instaliranu verziju \texttt{1.0}, a drugi \texttt{2.0}.

I sad, što ako klijent koji je zadovoljan sa starijom verzijom programa otkrije \textbf{bug}?
I zbog nekog razloga ne želi preći na novu verziju?
U tom slučaju, morate imati neki mehanizam da se privremeno vratite na staru verziju, ispravite problem, izdate novu verziju stare verzije. 
Pošaljete je klijentu i nakon toga, vratite se na najnoviju verziju i tamo nastavite rad na svoj projektu.

\section*{Grafovi, grananje i spajanje grana}
\addcontentsline{toc}{section}{Grafovi, grananje i spajanje grana}

Prije nego nastavimo s gitom, par riječi o grafovima. 
U ovoj knjižici će vidjeti puno grafova kao što su u primjerima s linearnim verzioniranjem koda. 
Zato ćemo se na trenutak zadržati na jednom takvom grafu:

\input{graphs/primjer_s_granama_i_spajanjima}

Svaka točka grafa je stanje projekta. 
Projekt s gornjim grafom je započeo s nekim početnim stanjem \emph a.
Programer je napravio nekoliko izmjena i \emph{snimio} novo stanje \emph b, zatim \emph c, pa sve do \emph h.
Važno je napomenuti da je ovakav graf stanje povijesti projekta, ali iz njega ne možete zaključiti kojim redosljedom je nastao.

Evo, na primjer, jedan scenarij kako je navedeni graf mogao nastati:

\input{graphs/primjer_s_granama_i_spajanjima_1}

Programer je započeo aplikaciju, snimio stanje \emph a, \emph b i \emph c i tada se sjetio da ima neki problem kojeg može riješiti na dva načina, vratio se na \emph b i napravio novu granu:

\input{graphs/primjer_s_granama_i_spajanjima_2}

Zatim se sjetio jedne sitnice koju je mogao napraviti u \emph{originalnoj} verziji; vratio se tamo i dodao \emph d:

\input{graphs/primjer_s_granama_i_spajanjima_3}

Nakon toga se vratio na svoj prvotni eksperiment, i odlučio se da bi bilo dobro tamo imati izmjene koje je napravio u \emph c i \emph d.
Tada je \emph{preuzeo} te izmjene u svoju granu:

\input{graphs/primjer_s_granama_i_spajanjima_4}

Vratio se na svoju eksperimentalnu granu, napravio još jednu izmjenu \emph q, vratio se na originalnu granu i tamo napredovao s \emph e i \emph f. 

\input{graphs/primjer_s_granama_i_spajanjima_5}

Sjetio se da bi mu sve izmjene iz eksperimentalne grane odgovarale u originalnoj, \emph{preuzeo} ih u početnu granu:

\input{graphs/primjer_s_granama_i_spajanjima_6}

I, zatim je nastavio, stvorio još jednu eksperimentalnu granu(\emph 1, \emph 2, \emph 3, \dots)\dots I tako dalje\dots

\input{graphs/primjer_s_granama_i_spajanjima}

Uočite da izmjena \emph w nije nikad završila u glavnoj grani. 
Jedna od velikih prednosti gita je lakoća stvaranja novih grana i preuzimanja izmjena iz jedne u drugu granu. 
Tako je programerima jednostavno u nekom trenutku odlučiti ovako \emph{"Ovaj problem bih mogao riješiti na dva različita načina. Pokušati ću i jedan i drugi, i onda vidjeti koji mi bolje ide."}. Za svaku verziju će napraviti posebnu granu i napredovati prema osjećaju.

Druga velika prednost čestog grananja u programima je kad se dodaje neka nova funkcionalnost koja će zahtijevati puno izmjena, a ne želime te izmjene odmah stavljati u glavnu granu programa:

\input{graphs/primjer_s_dugotrajnom_granom}

Jedino na što trebate pripaziti da redovito izmjene iz glavne grane programa preuzimate u sporednu, tako da razlike u kodu ne budu prevelike. 
Te izmjene su na grafu označene sivim strelicama.

Kad novu funkcionalnost završite, treba samo u glavnoj granu preuzeti sve izmjene iz sporedbne (crvena strelica).

Na taj način ćete često imati ne samo dvije grane (glavnu i sporednu) nego nekoliko njih (pa i do nekoliko desetaka). 
Imati ćete posebne grane za različite nove funkcionalnosti, posebne grane za eksperimente, posebne grane u kojima ćete isprobavati izmjene koje su napravili drugi programeri, \dots

Osnovna ideja ove knjižice \emph{nije} da vas uči kako je najbolje organizirati povijest projekta, odnosno kako granati, kad i kako preuzimati izmjene iz pojedinih grana. Osnovna ideja je da vas nauči \emph{kako} to napraviti s gitom. Zato sad krećemo s nečim konkretnim.

% \section*{}
% \addcontentsline{toc}{section}{}
