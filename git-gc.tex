\chapter*{"Higijena" repozitorija}
\addcontentsline{toc}{chapter}{"Higijena" repozitorija}

Naš repozitorij je naš životni prostor i s njime živimo dio dana.
Kao što se trudimo držati stan urednim, tako bi trebalo i s našim virtualnim prostorima.
Preko tjedna, kad rano ujutro odlazimo na posao i vraćamo se kasnije popodne, se ponekad desi da nam se u stanu nagomila robe za pranje.
No, nekoliko puta tjedno treba uzeti pola sata vremena i počistiti nered koji je zaostao.

Tako i s repozitorijem.
Ima nekoliko stvari na koje bi trebali pripaziti.

\subsection*{Grane}
\addcontentsline{toc}{subsection}{Grane}

Nemojte si dopustiti da grane u vašem repozitoriju izgledaju ovako "zbrčkano":

\input{graphs/cirkus}

Iako nam git omogućuje da imamo neograničen broj grana, ljudski um nije sposoban vizualizirati si više od 5-10 njih\footnote{Barem moj nije, ako je vaš izuzetak, preskočite sljedećih nekoliko rečenica. Ili jednostavno zamislite da je umjesto "5-10" pisalo "500-1000".}.
Kako stvaramo nove grane -- dešava se da imamo one u kojima smo počeli neki posao i kasnije odlučili da nam ne treba.
Ili smo napravili posao, \emph{merge}ali u \verb+master+, ali nismo obrisali granu.
Nađemo li se s više od 10-15 grana -- \textbf{sigurno} je dio njih tu samo zato što smo ih zaboravili obrisati.

U svakom slučaju, predlažem vam da svakih par dana pogledate malo po lokalnim (a i udaljenim granama) i provjerite one koje više ne koristite.

Ukoliko nismo sigurni je li nam u nekoj grani ostala možda još kakva izmjena koju treba vratiti u \verb+master+, postupak je sljedeći:

\gitoutputcommand{git checkout neka-grana\\git merge master}

\dots{}da bismo preuzeli sve izmjene iz master, tako da stanje bude ažurno.
I sad\dots

\gitoutputcommand{git diff master}

\dots{}da bismo provjerili koje su točno izmjene ostale.
Ako je prazno -- nema razlika i možemo brisati.
Ako nije -- \textbf{ima} izmjena i sad se sami odlučite jesu li te izmjene nešto važno što ćete nastaviti razvijati ili nešto što možemo zanemariti i obrisati tu granu.

\subsection*{Git gc}
\addcontentsline{toc}{subsection}{Git gc}

Druga stvar koju se preporuča ima veze s onim našim \verb+.git/objects+ direktorijem kojeg smo spominjali u "Ispod haube" poglavlju.
Kao što znamo, svaki \emph{commit} ima svoju referencu i svoj objekt (datoteku) u tom direktoriju.
Kad napravimo \verb+git commit --amend+ -- git stvara \textbf{novi} \emph{commit}.
Nije da on samo izmijeni stari\footnote{Ne bi ni mogao izmijeniti stari, jer ima drukčiji sadržaj i SHA1 bi mu se nužno morao promijeniti.}.

Grafički:

\input{graphs/amend}

Dakle, git interno dodaje \textbf{novi} objekt (\emph{f'}) i na njega pomiče referencu \verb+HEAD+ (koja je do tada gledala na \emph f).
On samo "kaže": Od sada na dalje, zadnji \emph{commit} u ovoj grani više nije \emph f, nego \emph{f'}.

No, sad u git repozitoriju imamo i \emph{commit} \emph f, a i \emph{f'}, ali samo jedan od njih se koristi (\emph{f'}).
Commit \emph f je \textbf{i dalje snimljen u} \verb+.git/object+ \textbf{direktoriju}.
I on se više neće koristiti.
Puno tih \verb+git commit --amend+ posljedično ostavlja puno takvog "smeća" u repozitoriju.

To vrijedi i za neke druge operacije kao brisanje grana ili rebase.
Git to čini da bi tekuće operacije bilo što je brže moguće.
Čišćenje takvog smeća (\emph{garbage collection} iliti \emph{gc}) ostavlja za kasnije, a ta radnja nije automatizirana nego se od nas očekuje da ju izvršimo\footnote{Nije automatizirana, ali možemo uvijek sami napraviti neki task koji se izvršava na dnevnoj ili tjednoj bazi, a koji "čisti" sve naše git repozitorije.}.

Naredba je \verb+git gc+:

\input{git_output/git_gc}

\dots{}i nju treba izvršavati s vremena na vrijeme.

Osim \verb+gc+, postoji još nekoliko sličnih naredbi kao \verb+git repack+, \verb+git prune+, no one su manje važne za početnika.
Ako vas zanima -- \verb+git help+ je uvijek na dohvat ruke.
