\chapter*{"Higijena" repozitorija}
\addcontentsline{toc}{chapter}{"Higijena" repozitorija}

Za programere -- repozitorij je životni prostor i s njime se živi dio dana.
Kao što se trudimo držati stan urednim, tako bi trebalo i s našim virtualnim prostorima.
Preko tjedna, kad rano ujutro odlazimo na posao i vraćamo se kasnije popodne, se ponekad desi da nam se u stanu nagomila robe za pranje.
No, nekoliko puta tjedno treba uzeti pola sata vremena i počistiti nered koji je zaostao.

Tako i s repozitorijem -- treba ga redovito održavati i čistiti nered koji ostavljamo za sobom.

\section*{Grane}
\addcontentsline{toc}{section}{Grane}

Nemojte si dopustiti da grane u vašem repozitoriju izgledaju ovako "zbrčkano":

\input{graphs/cirkus}

Iako nam git omogućuje da imamo neograničen broj grana, ljudski um nije sposoban vizualizirati si više od 5-10 njih\footnote{Barem moj nije, ako je vaš izuzetak, preskočite sljedećih nekoliko rečenica. Ili jednostavno zamislite da je umjesto "5-10" pisalo "500-1000".}.
Kako stvaramo nove grane -- dešava se da imamo one u kojima smo počeli neki posao i kasnije odlučili da nam ne treba.
Ili smo napravili posao, \emph{merge}ali u \verb+master+, ali nismo obrisali granu.
Nađemo li se s više od 10-15 grana -- \textbf{sigurno} je dio njih tu samo zato što smo ih zaboravili obrisati.

U svakom slučaju, predlažem vam da svakih par dana pogledate malo po lokalnim (a i udaljenim granama) i provjerite one koje više ne koristite.

Ukoliko nismo sigurni je li nam u nekoj grani ostala možda još kakva izmjena koju treba vratiti u \verb+master+, postupak je sljedeći:

\gitoutputcommand{git checkout neka-grana\\git merge master}

\dots{}da bismo preuzeli sve izmjene iz master, tako da stanje bude ažurno.
I sad\dots

\gitoutputcommand{git diff master}

\dots{}da bismo provjerili koje su točno izmjene ostale.
Ako je prazno -- nema razlika i možemo brisati.
Ako nije -- \textbf{ima} izmjena i sad se sami odlučite jesu li te izmjene nešto važno što ćete nastaviti razvijati ili nešto što možemo zanemariti i obrisati tu granu.

\section*{Git gc}
\addcontentsline{toc}{section}{Git gc}

Druga stvar koju se preporuča ima veze s onim našim \verb+.git/objects+ direktorijem kojeg smo spominjali u "Ispod haube" poglavlju.
Kao što znamo, svaki \emph{commit} ima svoju referencu i svoj objekt (datoteku) u tom direktoriju.
Kad napravimo \verb+git commit --amend+ -- git stvara \textbf{novi} \emph{commit}.
Nije da on samo izmijeni stari\footnote{Ne bi ni mogao izmijeniti stari, jer ima drukčiji sadržaj i SHA1 bi mu se nužno morao promijeniti.}.

Grafički:

\input{graphs/amend}

Dakle, git interno dodaje \textbf{novi} objekt (\emph{f'}) i na njega pomiče referencu \verb+HEAD+ (koja je do tada gledala na \emph f).
On samo "kaže": Od sada na dalje, zadnji \emph{commit} u ovoj grani više nije \emph f, nego \emph{f'}.

No, sad u git repozitoriju imamo i \emph{commit} \emph f, a i \emph{f'}, ali samo jedan od njih se koristi (\emph{f'}).
Commit \emph f je \textbf{i dalje snimljen u} \verb+.git/object+ \textbf{direktoriju}.
I on se više neće koristiti.
Puno tih \verb+git commit --amend+ posljedično ostavlja puno takvog "smeća" u repozitoriju.

To vrijedi i za neke druge operacije kao brisanje grana ili rebase.
Git to čini da bi tekuće operacije bilo što je moguće brže.
Čišćenje takvog "smeća" (\emph{garbage collection} iliti \emph{gc}) ostavlja za kasnije, a ta radnja nije automatizirana nego se od nas očekuje da ju izvršimo\footnote{Nije automatizirana, ali možemo uvijek sami napraviti neki task koji se izvršava na dnevnoj ili tjednoj bazi, a koji "čisti" sve naše git repozitorije.}.

Naredba je \verb+git gc+:

\input{git_output/git_gc}

\dots{}i nju treba izvršavati s vremena na vrijeme.

Osim \verb+gc+, postoji još nekoliko sličnih naredbi kao \verb+git repack+, \verb+git prune+, no one su manje važne za početnika.
Ako vas zanimaju -- \verb+git help+ je uvijek na dohvat ruke.

\section*{Povijest i brisanje grana}
\addcontentsline{toc}{section}{Povijest i brisanje grane}

Spomenuti ćemo još jednu stvar koja bi logički pripadala u poglavlja o granama i povijesti, ali tada nismo imali dovoljno znanja da bi ju shvatili.

Što se dešava s \emph{commit}ovima iz neke grane nakon što ju obrišemo?
Uzmimo tri primjera.
U sva tri imamo dvije grane: \verb+master+ i \verb+grana+.

Prvi primjer:

\input{graphs/git_merge_i_brisanje_grana_1}

\textbf{Svi čvorovi koji su dio povijesti projekta će ostati u repozitoriju i neće biti obrisati s} \verb+git gc+.
Kako znamo koji čvorovi su dio projekta?
Po tome što postoji netko tko ima referencu na njih.
Ta referenca može biti ili grana ili neki drugi čvor ili \emph{tag}.

Podsjetimo se da su strelice redosljed nastajanja, ali reference idu suprotnim smjerom -- sljedbenik ima referencu na prethodnika.
Dakle, $g$ ima referencu na $f$, $f$ na $e$, itd.
Što je sa čvorom $g$? 
Izgleda kao da nitko nema referencu na njega, ali to nije točno -- grana \verb+grana+ je referenca na njega.

Ako obrišemo granu \verb+grana+ -- $g$ više nema nikoga da njega referencira.
Mogli bi ga obrisati, ali onda bi mogli obrisati i $f$, $e$ i $d$ ($b$ ne možemo, jer $c$ ima referencu na njega).

Dakle, kad obrišemo granu koja nije \emph{merge}ana u neku drugu granu, onda se svi njeni čvorovi gube iz povijesti projekta.
I, prije ili kasnije će biti obrisani s \verb+git gc+.

Drugi primjer:

\input{graphs/git_merge_i_brisanje_grana_2}

Ovaj primjer je isti kao i prvi, s jednom razlikom. 
Došlo je do \emph{merge}a.

Znamo da grana nije ništa drugo nego referenca na njen zadnji čvor/\emph{commit}.
Obrišemo li granu \verb+grana+, obrisali smo referencu na njen zadnji čvor $g$.
No, i dalje imamo $q$ koji pokazuje na $g$.
Zbog toga će svi čvorovi grane \verb+grana+ nakon njenog brisanja ostati u repozitoriju.

Treći primjer:

\input{graphs/git_merge_i_brisanje_grana_3}

Ukoliko u ovom promjeru obrišemo \verb+grana+, postoji samo jedan njen čvor koji će biti izgubljen -- $g$.
Bez referencu na granu, niti jedan čvor niti \emph{tag} ne pokazuje na $g$, dakle on prestaje biti dio povijesti našeg projekta.
$z$ ima referencu na $f$, a s $f$ nam garantira da i $e$ i $d$ ostaju dio povijesti projekta.

\section*{\emph{Squash merge} i brisanje grana}
\addcontentsline{toc}{section}{\emph{Squash merge} i brisanje grana}

Uzmimo opet:

\input{graphs/git_merge_i_brisanje_grana_4}

Želimo li u povijesti projekta sačuvati izmjene iz neke grane, ali ne i njenu povijest -- to se može s \verb+git merge --squash+.
Podsjetimo se -- tom operacijom git \textbf{hoće} preuzeti izmjene iz grane, ali \textbf{neće} u čvoru $q$ napraviti referencu na $g$.
Dakle, rezultat je kao da kod klasičnog \emph{merge}a, ali bez reference (u prethodnom grafu crvenom bojom):

\input{graphs/git_merge_i_brisanje_grana_5}

Sad smo preuzeli izmjene iz \verb+grana+ u \verb+master+, ali \verb+git gc+ će prije ili kasnije obrisati $d$, $e$, $f$ i $g$.

S \verb+git merge --squash+ cijelu granu svodimo na jedan \emph{commit} i kasnije gubimo njenu povijest\footnote{\dots{}barem ako ju kasnije ne \emph{merge}amo klasičnim putem.}.
