\chapter*{"Higijena" repozitorija}
\addcontentsline{toc}{chapter}{"Higijena" repozitorija}

Naš repozitorij je naš životni prostor i s njime živimo dio dana.
Kao što se trudimo držati stan urednim, tako bi trebalo i s našim virtualnim prostorima.
Preko tjedna, kad rano ujutro odlazimo na posao i vraćamo se kasnije popodne, se ponekad desi da nam se u stanu nagomila robe za pranje.
No, nekoliko puta tjedno treba uzeti pola sata vremena i počistiti nered koji je zaostao.

Tako i s repozitorijem.
Ima nekoliko stvari na koje bi trebali pripaziti.

\subsection*{Grane}
\addcontentsline{toc}{subsection}{Grane}

Nemojte si dopustiti da grane u vašem repozitoriju izgledaju ovako "zbrčkano":

\input{graphs/cirkus}

Iako nam git omogućuje da imamo neograničen broj grana, ljudski um nije sposoban vizualizirati si više od 5-10\footnote{Barem moj nije, ako je vaš izuzetak, preskočite ovo.}.
Kako budete stvaramo nove grane -- dešava se da imamo one u kojima smo počeli neki posao i kasnije odlučili da nam ne treba.
Ili smo napravili posao, \emph{merge}ali u \verb+master+, ali nismo obrisali granu.
Nađemo li se s više od 10-15 grana -- \textbf{sigurno} je dio njih tu samo zato što smo ih zaboravili obrisati.

U svakom slučaju, predlažem vam da svakih par dana pogledate malo po lokalnim (a i udaljenim granama) i provjerite one koje više ne koristite.

Ukoliko nismo sigurni je li nam u nekoj grani ostala možda još kakva izmjena koju treba vratiti u \verb+master+, postupak je sljedeći:

\gitoutputcommand{git checkout neka-grana\\git merge master}

\dots{}da bismo preuzeli sve izmjene iz master, tako da stanje bude ažurno.
I sad\dots

\gitoutputcommand{git diff master}

\dots{}da bismo provjerili koje su točno izmjene ostale.
Ako je prazno -- nema razlika i možemo brisati.
Ako nije -- \textbf{ima} izmjena i sad se sami odlučite jesu li te izmjene nešto važno što ćete nastaviti razvijati ili nešto što možemo zanemariti i obrisati tu granu.

\subsection*{Git gc}
\addcontentsline{toc}{subsection}{Git gc}

Druga stvar koju se preporuča ima veze s onim našim \verb+.git/objects+ direktorijem kojeg smo spominjali u "Ispod haube" poglavlju.

\begin{itemize}
   \item Čišćenje nekorištenih brancheva
   \item git gc
   \item git repack
   \item git prune
\end{itemize}

%\section*{}
%\addcontentsline{toc}{section}{}

