\chapter*{Grananje}
\addcontentsline{toc}{chapter}{Grananje}

Još ćemo jednom početi s već viđenim grafom:

\input{graphs/primjer_s_imenovanim_granama_i_spajanjima}

Ovaj put s jednom izmjenom, svaka "grana" ima svoj naziv: \verb+novi-feature+, \\ \verb+ispravak-problema-x+ i \verb+master+.
U uvodnom je poglavlju opisan jedan od mogućih scenarija koji je mogao dovesti do ovog grafa.
Ono što je ovdje važno još jednom spomenuti je sljedeće: svaki je čvor grafa stanje projekta u nekom trenutku njegove povijesti. 
Svaka strelica iz jednog u drugi čvor izmjena je koju je programer napravio i snimio u nadi da će dovesti do željenog ponašanja aplikacije.

\section*{Popis grana projekta}
\addcontentsline{toc}{section}{Popis grana projekta}

Jedna od velikih prednosti gita je što omogućuje jednostavan i brz rad s višestrukim granama.
Želimo li vidjeti koje točno grane našeg projekta trenutno postoje naredba je \verb+git branch+.
U većini će slučajeva rezultat te naredbe biti:

\input{git_output/git_branch_s_jednom_granom}

To znači da naš projekt trenutno ima samo jednu granu.
\textbf{Svaki git repozitorij u početku ima jednu jedinu granu i ona se uvijek zove} \verb+master+.

Ako smo naslijedili projekt kojeg je netko prethodno već granao, dobiti ćemo nešto kao:

\input{git_output/git_branch_s_vise_grana}

Ili na primjer ovako:

\input{git_output/git_branch_s_vise_grana_2}

Svaki redak predstavlja jednu granu, a redak koji počinje zvjezdicom (*) \textbf{grana je u kojoj se trenutno nalazimo}.
U toj grani tada možemo raditi sve što i na \verb+master+ -- commitati, gledati njenu povijest, \dots

\section*{Nova grana}
\addcontentsline{toc}{section}{Nova grana}

Ako je trenutni ispis naredbe \verb+git branch+ ovakav:

\input{git_output/git_branch_s_jednom_granom}

\dots{} to znači da je graf našeg projekta ovakav:

\input{graphs/kreiranje_grane_1}

Sad se može desiti da nakon stanja \emph c želimo isprobati dva različita pristupa.
Novu granu možemo stvoriti naredbom \verb+git branch <naziv_grane>+.
Na primjer:

\gitoutputcommand{git branch eksperimentalna-grana}

Sad je novo stanje projekta:

\input{graphs/kreiranje_grane_2}

Imamo granu \verb+eksperimentalna-grana+, ali u njoj nemamo još ni jednog čvora (\emph{commit}a).
U toj grani sad možemo raditi točno onako kako smo se do sada naučili raditi s \verb+master+: mijenjati (dodavati, brisati) datoteke, spremati ih u indeks s \verb+git add+ i \emph{commit}aati s \verb+git commit+.
Sad bismo dobili da naša nova grana ima svoj prvi čvor: 

\input{graphs/kreiranje_grane_2_1}

\section*{Prebacivanje s grane na granu}
\addcontentsline{toc}{section}{Prebacivanje s grane na granu}

Primijetimo da se i dalje "nalazimo" na \verb+master+ grani:

\input{git_output/git_branch_primjer}

Naime, \verb+git branch+ stvorit će nam samo novu granu.
Prebacivanje s jedne grane na drugu granu radi se naredbom \verb+git checkout <naziv_grane>+:

\input{git_output/git_checkout}

Analogno, na glavnu se granu vraćamo s \verb+git checkout master+.

Sada kad smo se prebacili na novu granu, možemo tu uredno \emph{commit}ati svoje izmjene. 
Sve što tu radimo neće biti vidljivo u \verb+master+ grani.

\input{graphs/kreiranje_grane_3}

Kad god želimo, možemo se prebaciti na \verb+master+ i tamo nastaviti rad koji nije nužno vezan uz izmjene u drugoj grani:

\input{graphs/kreiranje_grane_4}

Nakon prebacivanja na \verb+master+, izmjene koje smo napravili u \emph{commit}ovima \emph d, \emph e, \emph f i \emph g neće nam biti vidljive.
Kad se prebacimo na \verb+eksperimentalna-grana+, neće nam biti vidljive izmjene iz \emph x, \emph y i \emph z.

Ako ste ikad radili grane na nekom drugom, klasičnom, sustavu za verzioniranje koda, onda ste vjerojatno naviknuti da to grananje potraje malo duže (od nekoliko sekundi do nekoliko minuta).
Stvar je u tome što u većini ostalih sustava \emph{proces} grananja u stvari podrazumijeva \textbf{kopiranje svih datoteka} na mjesto gdje se čuva nova grana.
To em traje neko vrijeme, em zauzima više memorije na diskovima.

Kod gita je to puno jednostavnije. Nakon što kreiramo novu granu nema nikakvog kopiranja na disku. 
Čuva se samo informacija da smo kreirali novu granu i \textbf{posebne verzije datoteka koje su specifične za tu granu} (o tome više u posebnom poglavlju).
Svaki put kad spremite izmjenu čuva se samo ta izmjena.
Zahvaljujući tome postupak grananja je izuzetno brz i zauzima malo mjesta na disku.

\subsection*{Prebacivanje na granu i tekuće izmjene}
\addcontentsline{toc}{subsection}{Prebacivanje na granu i tekuće izmjene}

Kod prebacivanja s grane na granu s \verb+git checkout+ može nastati manji problem ako imamo ne\emph{commit}anih izmjena.
Postoji nekoliko situacija u kojima nam git neće dopustiti prebacivanje.
Najčešća je kada u dvije grane imamo dvije različite verzije iste datoteke, a tu smo datoteku u tekućoj grani izmijenili, ostavili ne\emph{commit}anu.

Zapamtite, \textbf{najbolje je prebacivati se s grane na granu tek nakon što smo \emph{commit}ali sve izmjene}. Tako će nas u novoj grani dočekati čista situacija, a ne datoteke koje smo izmijenili dok smo radili na prethodnoj grani. 

\section*{Brisanje grane}
\addcontentsline{toc}{section}{Brisanje grane}

Zato što je grananje memorijski i procesorski nezahtjevno i brzo, pripremite se na situacije kad ćete se naći s \textbf{previše} grana.
Možda smo neke grane napravili da bismo isprobali nešto novo, a to se na kraju pokazalo kao loša ideja pa smo granu napustili.
Ili smo je započeli da bismo riješili neki problem, ali taj problem je prije nas riješio netko drugi.

U tom slučaju granu možemo obrisati s \verb+git branch -D <naziv_grane>+. 
Dakle, ako je stanje grana na našem projektu:

\input{git_output/git_branch_primjer}

\dots{}nakon:

\input{git_output/git_branch_D}

\dots{}novo stanje će biti:

\input{git_output/git_branch_s_jednom_granom}

Primijetimo samo da sad ne možemo obrisati \verb+master+:

\input{git_output/git_branch_D_trenutne_grane}

To vrijedi i općenito, ne možemo obrisati granu na kojoj se trenutnačno nalazimo.

Treba znati i to da brisanjem grane ne brišemo njene \emph{commit}ove.
Oni ostaju dio povijesti do daljnjega.
Više detalja o tome koji točno \emph{commit}ovi i u kojim se uvijetima brišu iz povijesti će biti u poglavlju o "higijeni" projekta.

\section*{Preuzimanje datoteke iz druge grane}
\addcontentsline{toc}{section}{Preuzimanje datoteke iz druge grane}

S puno grana, događati će se svakakve situacije.
Relativno česta situacija je da želimo preuzeti samo jednu ili više datoteka iz druge grane, ali ne želimo \textbf{preći} na tu drugu granu.
Znamo da su neke datoteke u drugoj grani izmijenjene i želimo ih preuzeti u trenutnu granu.
To se može ovako:

\gitoutputcommand{git checkout <naziv\_grane> -- <datoteka1> <datoteka2> ...}

Na primjer, ako smo u \verb+master+, a treba nam datoteka \verb+.classpath+ koju smo izmijenili u \verb+eksperiment+, onda ćemo je dobiti s:

\gitoutputcommand{git checkout eksperiment -- .classpath}

