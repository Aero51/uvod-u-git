\chapter*{Grananje}
\addcontentsline{toc}{chapter}{Grananje}

I opet, početi ćemo s ovim, već viđenim, grafom:

\input{graphs/primjer_s_imenovanim_granama_i_spajanjima}

Ovaj put s jednom izmjenom, svaga "grana" ima svoj naziv.
U uvodnom poglavlju je opisan scenarij kakav bi otprilike mogao scenarij kojim je ovaj graf mogao nastati.
Ono što je ovdje važno još jednom spomenuti je sljedeće; svaki čvor grafa je nekakvo stanje projekta. 
Svaka strelica iz jednog u drugi čvor je neka izmjena koju je programer napravio i snimio u nadi da će ona dovesti do željenog ponašanja aplikacije.

Git vam omogućuje da jednostavno i brzo radite višestruke grane projekta.

Želite li vidjeti koje točno grane vašeg projekta postoje u nekom trenutku, probajte s:

%\begin{itemize}
%   \item Kako se grana
%   \item prebacivanje s jedne grane na drugu granu
%   \item preuzimanje samo jednog fajla iz drugog brancha
%\end{itemize}

%\section*{}
%\addcontentsline{toc}{section}{}

