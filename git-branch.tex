\chapter*{Grananje}
\addcontentsline{toc}{chapter}{Grananje}

I opet, početi ćemo s ovim, već viđenim, grafom:

\input{graphs/primjer_s_imenovanim_granama_i_spajanjima}

Ovaj put s jednom izmjenom, svaga "grana" ima svoj naziv.
U uvodnom poglavlju je opisan scenarij koji bi, otprilike, mogao dovesti do ovog grafa.
Ono što je ovdje važno još jednom spomenuti je sljedeće; svaki čvor grafa je nekakvo stanje projekta. 
Svaka strelica iz jednog u drugi čvor je neka izmjena koju je programer napravio i snimio u nadi da će ona dovesti do željenog ponašanja aplikacije.

\section*{Spisak grana projekta}
\addcontentsline{toc}{section}{Spisak grana projekta}

Jedna od velikih prednosti gita je što omogućuje jednostavan i brz rad s višestrukim granama projekta.
Želite li vidjeti koje točno grane vašeg projekta postoje u nekom trenutku, probajte s \verb+git branch+.
U većini slučajeva, rezultat te naredbe će biti:

\input{git_output/git_branch_s_jednom_granom}

To znači da graf vašeg projekta \emph{nema} višestrukih grana. 
Ukoliko ste naslijedili projekt kojeg je netko prethodno već granao, dobiti ćete nešto kao:

\input{git_output/git_branch_s_vise_grana}

Ili, na primjer ovako:

\input{git_output/git_branch_s_vise_grana_2}

Svaki redak predstavlja jednu granu, a redak koji počinje sa zvjezdicom (*) je \emph{grana u kojoj se trenutno nalazite}.

\section*{Nova grana}
\addcontentsline{toc}{section}{Nova grana}

Ukoliko je trenutni ispis komande \verb+git branch+ ovakav:

\input{git_output/git_branch_s_jednom_granom}

\dots{} to znači da je graf vašeg projekta ovakav:

\input{graphs/kreiranje_grane_1}

I sad se može desiti da nakon stanja \emph c želite isprobati dva različita pristupa.
Novu granu možete kreirati naredbom \verb+git branch <naziv_grane>+.
Na primjer:

\gitoutputcommand{git branch eksperimentalna-grana}

Sad je stanje vašeg projekta:

\input{graphs/kreiranje_grane_2}

\section*{Prebacivanje s grane na granu}
\addcontentsline{toc}{section}{Prebacivanje s grane na granu}

Primijetite da se i dalje "nalazite" na master grani:

\input{git_output/git_branch_primjer}

Prebacivanje s jedne grane na drugu granu se radi s komandom \verb+...+:

%\begin{itemize}
%   \item Kako se grana
%   \item prebacivanje s jedne grane na drugu granu
%   \item preuzimanje samo jednog fajla iz drugog brancha
%\end{itemize}

%\section*{}
%\addcontentsline{toc}{section}{}

