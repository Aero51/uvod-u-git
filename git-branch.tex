\chapter*{Grananje}
\addcontentsline{toc}{chapter}{Grananje}

I opet, početi ćemo s ovim, već viđenim, grafom:

\input{graphs/primjer_s_imenovanim_granama_i_spajanjima}

Ovaj put s jednom izmjenom, svaga "grana" ima svoj naziv.
U uvodnom poglavlju je opisan scenarij koji bi, otprilike, mogao dovesti do ovog grafa.
Ono što je ovdje važno još jednom spomenuti je sljedeće; svaki čvor grafa je nekakvo stanje projekta. 
Svaka strelica iz jednog u drugi čvor je izmjena koju je programer napravio i snimio u nadi da će ona dovesti do željenog ponašanja aplikacije.

\section*{Spisak grana projekta}
\addcontentsline{toc}{section}{Spisak grana projekta}

Jedna od velikih prednosti gita je što omogućuje jednostavan i brz rad s višestrukim granama projekta.
Želite li vidjeti koje točno grane vašeg projekta postoje u nekom trenutku, probajte s \verb+git branch+.
U većini slučajeva, rezultat te naredbe će biti:

\input{git_output/git_branch_s_jednom_granom}

To znači da graf vašeg projekta \emph{nema} višestrukih grana. 
Zapamtite i ovo -- svaki git repozitorij na početku ima jednu jedinu granu i ona se uvijek zove \verb+master+.

Ukoliko ste naslijedili projekt kojeg je netko prethodno već granao, dobiti ćete nešto kao:

\input{git_output/git_branch_s_vise_grana}

Ili, na primjer ovako:

\input{git_output/git_branch_s_vise_grana_2}

Svaki redak predstavlja jednu granu, a redak koji počinje sa zvjezdicom (*) je \emph{grana u kojoj se trenutno nalazite}.
U toj grani tada možete raditi sve što i na \verb+master+; commitati, gledati njenu povijest, \dots

\section*{Nova grana}
\addcontentsline{toc}{section}{Nova grana}

Ukoliko je trenutni ispis komande \verb+git branch+ ovakav:

\input{git_output/git_branch_s_jednom_granom}

\dots{} to znači da je graf vašeg projekta ovakav:

\input{graphs/kreiranje_grane_1}

I sad se može desiti da nakon stanja \emph c želite isprobati dva različita pristupa.
Novu granu možete kreirati naredbom \verb+git branch <naziv_grane>+.
Na primjer:

\gitoutputcommand{git branch eksperimentalna-grana}

Sad je stanje vašeg projekta:

\input{graphs/kreiranje_grane_2}

\section*{Prebacivanje s grane na granu}
\addcontentsline{toc}{section}{Prebacivanje s grane na granu}

Primijetite da se i dalje "nalazite" na \verb+master+ grani:

\input{git_output/git_branch_primjer}

Naime, \verb+git branch+ će vam sam kreirati novu granu.
Prebacivanje s jedne grane na drugu granu se radi s komandom \verb+git checkout <naziv_grane>+:

\input{git_output/git_checkout}

Analogno, na glavnu granu se vraćate s \verb+git checkout master+.

Sad, kad ste se prebacili na novu granu, možete tamo uredno \emph{commit}ati svoje izmjene. 
I sve što tu spremite, neće biti vidljivo u \verb+master+ grani.

\input{graphs/kreiranje_grane_3}

I, kad god želite, možete se prebaciti na \verb+master+ i tamo nastaviti rad koji nije nužno vezan uz izmjene u drugoj grani:

\input{graphs/kreiranje_grane_4}

Kad se prebacite na \verb+master+, izmjene koje ste napravili u \emph{commit}ovima \emph d, \emph e, \emph f i \emph g vam neće biti dostupne.
Istko, kad se prebacite na \verb+eksperimentalna-grana+ -- neće vam biti dostupne izmjene iz \emph x, \emph y i \emph z.

Ako ste ikad radili grane na nekom drugom, klasičnom, sustavu za verzioniranje koda, onda ste vjerojatno naviknuti da to grananje potraje malo duže (od nekoliko sekundi do nekoliko minuta).
Stvar je u tome što je, u većini ostalih sustava, \emph{proces} grananja u stvari podradzumijeva \textbf{kopiranje svih datoteka} na mjesto gdje se čuva nova grana.
To, em traje neko vrijeme, em zauzima više memorije na diskovima.

Kod gita to je puno jednostavnije, kad kreirate novi granu, nema nikakvog kopiranja na disku. 
Čuva se samo informacija da ste kreirali novu granu (o tome više u posebnom poglavlju).
Svaki put kad spremite izmjenu, čuva se samo ta izmjena.
Zahvaljujući tome postupak grananja je izuzetno brz i zauzima malo mjesta na disku.

\section*{Brisanje grane}
\addcontentsline{toc}{section}{Brisanje grane}

Zato što je grananje memorijski nezahtjevno i brzo, desiti će vam se da se u jednom trenutku nađete s \emph{previše} grana.
Možda ste neke grane napravili da bi isprobali nešto novo, što se na kraju pokazalo kao loša ideja pa ste ju napustili.
Ili ste neku granu kreirali da bi započeli nešto novo u aplikaciji, ali na kraju je to riješio netko drugi ili se pokazalo da to nije potrebno.

U tom slučaju, granu možete obrisati s \verb+git branch -D <naziv_grane>+. 
Dakle, ako je stanje grana na vašem projektu:

\input{git_output/git_branch_primjer}

\dots{}nakon:

\input{git_output/git_branch_D}

\dots{}novo stanje će biti:

\input{git_output/git_branch_s_jednom_granom}

Primijetite samo da sad ne možete obrisati \verb+master+:

\input{git_output/git_branch_D_trenutne_grane}

I to vrijedi općenito -- ne možete obrisati granu na kojoj se trenutno nalazite.

\section*{Preuzimanje datoteke iz druge grane}
\addcontentsline{toc}{section}{Preuzimanje datoteke iz druge grane}

S puno grana, dešavati će vam se svakakvi scenariji.
Jedan od relativno čestih je situacija kad biste htjeli preuzeti samo jednu ili više datoteka iz druge grane, ali ne želite \emph{preći} na tu drugu granu.
Samo datoteke.
To se može ovako:

\gitoutputcommand{git checkout <naziv\_grane> -- <datoteka1> <datoteka2> ...}

Na primjer, ako ste u \verb+master+, a treba sam datoteka \verb+.classpath+ koju ste izmijenili u \verb+eksperiment+, onda ćete ju dobiti s:

\gitoutputcommand{git checkout eksperiment -- .classpath}
