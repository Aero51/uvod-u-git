\chapter*{Udaljeni repozitoriji}
\addcontentsline{toc}{chapter}{Udaljeni repozitoriji}

Sve ono što smo do sada proučavali smo radili isključivo na lokalnom repozitoriju.
Samo smo spomenuli da je git distribuirani sustav za verzioniranje koda, složiti ćete se da je već krajnje vrijeme da krenemo pomalo obrađivati interakciju s udaljenim repozitorijima.

Postoji puno scenarija kako može funkcionirati ta interakcija.
Koncentrirajmo se sada na sam trenutak kad repozitorij "dođe" ili "nastane" na našem lokalnom računalu.
Moguće je da smo ga stvorili od nule s \verb+git init+, na način kako smo to radili do sada i onda, na neki način, "poslali" na neku udaljenu lokaciju.
Ili smo takav repozitorij ponudili drugim ljudima da ga preuzmu (kloniraju).

No, moguće je i da smo saznali za neki već postojeći repozitorij, negdje na internetu, i sad želimo i mi preuzeti taj kod.
Bilo da je to zato što želimo pomoći u razvoju ili samo proučiti nečiji kod.
Krenimo s prvim tipičnim scenarijem\dots

\section*{Naziv i adresa repozitorija}
\addcontentsline{toc}{section}{Naziv i adresa repozitorija}

Prvu stvar koju ćemo obraditi je kloniranje udaljenog repozitorija.
No, ima prije toga jedna sitnica koju trebamo znati.
Svaki udaljeni repozitorij s kojime će git "komunicirati" mora imati svoju adresu.

Na primjer, ova knjiga "postoji" na \verb+git://github.com/tkrajina/uvod-u-git.git+, i to je jedna njega andresa.
Github\footnote{Web servis koji omogućava da držite svoje git repozitorije} omogućava da se istim repozitoriju pristupi i preko \\ \verb+https://tkrajina@github.com/tkrajina/uvod-u-git.git+.
Osim na Githubu, ona živi i na mom lokalnom računalu i u tom slučaju je njena adresa \verb+/home/puzz/projects/uvod-u-git+ (direktorij u kojemu se nalazi).

Nadalje, svaki udaljeni repozitorij ima i svoje kratko ime.
Nešto kao: \verb+origin+ ili \\ \verb+vanjin-repozitorij+ ili \verb+slobodan+ ili \verb+dalenov-repo+.
Nazivi su vaš slobodan izbor. 
Tako, ako vas četvero radi na istom projektu, njihove udaljene repozitorije možete nazvati \verb+marina+, \verb+ivan+, \verb+karla+.
I sa svakim od njih možete imati nekakvu vrstu interakcije. 
Na neke ćete slati svoje izmjene (ako imate ovlasti), a s nekih ćete preuzimati u svoj repozitorij.

\section*{Kloniranje repozitorija}
\addcontentsline{toc}{section}{Kloniranje repozitorija}

To je postupak kojim kopiramo cijeli repozitorij na nekoj udaljenoj lokaciji na naše lokalno računalo, a da bi onda s njime dalje nastavili raditi.
Postupak je jednostavan, ako znamo adresu (a moramo znati), onda\dots

\input{git_output/git_clone}

\dots{}\textbf{kopira cijeli projekt, zajedno sa cijelom poviješću} na naše računalo.
I to u direktorij \verb+uvod-u-git+.
Sad tu možemo gledati povijest, granati, \emph{commit}ati, \dots Raditi što god nas je volja s tim projektom.

Jasno, ne može bilo tko kasnije svoje izmjene poslati nazad na originalnu lokaciju. 
Za to moramo imati ovlasti, ili moramo vlasnika tog repozitorija pitati je li voljan naše izmjene preuzeti kod sebe.
O tome malo kasnije.

E, i još nešto. Sjećate se kad sam napisao da su nazivi udaljenih repozitorija vaš slobodan izbor.
Nisam baš bio $100\%$ iskren. 
Kloniranje je izuzetak.
Ukoliko kloniramo udaljeni repozitorij, on se za nas zove \verb+origin+.
Ostali repozitoriji koje ćemo dodavati mogu imati nazive kakve god želimo.

\subsection*{Djelomično kloniranje povijesti repozitorija}
\addcontentsline{toc}{subsection}{Djelomično kloniranje povijesti repozitorija}

Našli ste na internetu neki zanimljiv projekt i njegov git repozitorij i htjeli bi ga skinuti i proučiti njegov kod. 
Ništa lakše; \verb+git clone ...+.

E, ali\dots
Tu imamo mali problem.
Git repozitorij sadrži cijelu povijest projekta. 
To znači da sadrži sve \emph{commit}ove koje su radili programeri i koji mogu sezati i preko deset godina unazad.
I zato \verb+git clone+ ponekad može potrajati dosta dugo. 
Posebno ako imate sporu vezu.

No postoji trik.
Želimo li skinuti projekt samo zato da bi pogledali njegov kod, a ne i zato da bi s njime nastavili normalnu interakciju, možemo skinuti nekoliko njegovih zadnjih \emph{commit}ova s:

\gitoutputcommand{git clone --depth 5 --no-hardlinks git://github.com/tkrajina/uvod-u-git.git}

To će biti puno brže, no nećete imati pristup cijeloj povijesti, nego samo zadnjih $5$ \emph{commit}ova.

\section*{Fetch}
\addcontentsline{toc}{section}{Fetch}

\section*{Push}
\addcontentsline{toc}{section}{Push}

\section*{Pull}
\addcontentsline{toc}{section}{Pull}

\section*{Udaljeni repozitoriji}
\addcontentsline{toc}{section}{Udaljeni repozitoriji}

Push, fetch, pull, problemi koji mogu nastajati

tagovi, \dots

%\section*{}
%\addcontentsline{toc}{section}{}

