\chapter*{Udaljeni repozitoriji}
\addcontentsline{toc}{chapter}{Udaljeni repozitoriji}

Sve ono što smo do sada proučavali smo radili isključivo na lokalnom repozitoriju.
Samo smo spomenuli da je git distribuirani sustav za verzioniranje koda, složiti ćete se da je već krajnje vrijeme da krenemo pomalo obrađivati interakciju s udaljenim repozitorijima.

Postoji puno scenarija kako može funkcionirati ta interakcija.
Koncentrirajmo se sada na sam trenutak kad repozitorij "dođe" ili "nastane" na našem lokalnom računalu.
Moguće je da smo ga stvorili od nule s \verb+git init+, na način kako smo to radili do sada i onda, na neki način, "poslali" na neku udaljenu lokaciju.
Ili smo takav repozitorij ponudili drugim ljudima da ga preuzmu (kloniraju).

No, moguće je i da smo saznali za neki već postojeći repozitorij, negdje na internetu, i sad želimo i mi preuzeti taj kod.
Bilo da je to zato što želimo pomoći u razvoju ili samo proučiti nečiji kod.
Krenimo s prvim tipičnim scenarijem\dots

\section*{Naziv i adresa repozitorija}
\addcontentsline{toc}{section}{Naziv i adresa repozitorija}

Prvu stvar koju ćemo obraditi je kloniranje udaljenog repozitorija.
No, ima prije toga jedna sitnica koju trebamo znati.
Svaki udaljeni repozitorij s kojime će git "komunicirati" mora imati svoju adresu.

Na primjer, ova knjiga "postoji" na \verb+git://github.com/tkrajina/uvod-u-git.git+, i to je jedna njega andresa.
Github\footnote{Web servis koji omogućava da držite svoje git repozitorije} omogućava da se istim repozitoriju pristupi i preko \\ \verb+https://tkrajina@github.com/tkrajina/uvod-u-git.git+.
Osim na Githubu, ona živi i na mom lokalnom računalu i u tom slučaju je njena adresa \verb+/home/puzz/projects/uvod-u-git+ (direktorij u kojemu se nalazi).

Nadalje, svaki udaljeni repozitorij ima i svoje kratko ime.
Nešto kao: \verb+origin+ ili \\ \verb+vanjin-repozitorij+ ili \verb+slobodan+ ili \verb+dalenov-repo+.
Nazivi su vaš slobodan izbor. 
Tako, ako vas četvero radi na istom projektu, njihove udaljene repozitorije možete nazvati \verb+marina+, \verb+ivan+, \verb+karla+.
I sa svakim od njih možete imati nekakvu vrstu interakcije. 
Na neke ćete slati svoje izmjene (ako imate ovlasti), a s nekih ćete izmjene preuzimati u svoj repozitorij.

\section*{Kloniranje repozitorija}
\addcontentsline{toc}{section}{Kloniranje repozitorija}

Kloniranje je postupak kojim kopiramo cijeli repozitorij na nekoj udaljenoj lokaciji na naše lokalno računalo, a da bi onda s njime dalje nastavili raditi.
U stvari, kopirati repozitorij je jednostavno, dovoljno je u neki direktorij kopirati \verb+.git+ direktorij drugog repzitorija i onda u njemu napraviti \verb+git checkout HEAD+.

Kloniranje je za nijansu drukčije.
Recimo to ovako, \textbf{kloniranje je kopiranje udaljenog repozitorija, ali tako da novi (lokalni) repozitorij ostaje "svjestan" da je on kopija nekog udaljenog repozitorija}.

Postupak je jednostavan, ako znamo adresu (a moramo znati), onda\dots

\input{git_output/git_clone}

\dots{}\textbf{kopira cijeli projekt, zajedno sa cijelom poviješću} na naše računalo.
I to u direktorij \verb+uvod-u-git+.
Sad tu možemo gledati povijest, granati, \emph{commit}ati, \dots Raditi što god nas je volja s tim projektom.

Jasno, ne može bilo tko kasnije svoje izmjene poslati nazad na originalnu lokaciju. 
Za to moramo imati ovlasti, ili moramo vlasnika tog repozitorija pitati je li voljan naše izmjene preuzeti kod sebe.
O tome malo kasnije.

E, i još nešto. Sjećate se kad sam napisao da su nazivi udaljenih repozitorija vaš slobodan izbor.
Nisam baš bio $100\%$ iskren. 
Kloniranje je izuzetak.
Ukoliko kloniramo udaljeni repozitorij, on se za nas zove \verb+origin+.
Ostali repozitoriji koje ćemo dodavati mogu imati nazive kakve god želimo.

\subsection*{Struktura kloniranog repozitorija}
\addcontentsline{toc}{subsection}{Struktura kloniranog repozitorija}

Od trenutka kad smo klonirali svoj repozitorij pa na dalje -- za nas postoje \textbf{dva repozitorija}.
Možda negdje na svijetu postoji još netko tko je klonirao taj isti repozitorij i na njemu nešto radi (a da mi o tome ništa ne znamo).
No, naš dio svijeta su samo ova dva s kojima direktno imamo posla. 
Jedan je udaljeno kojeg smo klonirali, a drugi je lokalni koji se nalazi pred nama.

Prije negoli počnemo s pričom o tome kako slati ili primati izmjene iz jednog repozitorija u drugi, trebamo nakratko spomenuti kakva je točno struktura lokalnog repozitorija.
Već znamo za naredbu \verb+git branch+, koja nam ispisuje spisak svih grana na našem repozitoriju.
No, sad imamo posla i sa udaljenim repozitorijem -- njega smo klonirali.

S \verb+git branch -a+ ispisujemo \textbf{sve grane koje su nam trenutno dostupne u lokalnom rpeozitoriju}.
Naime, kad smo klonirali repozitorij -- postale su nam dostupne i grane udaljenog repozitorija:

\input{git_output/git_branch_all_1}

Novost ovdje je \verb+remotes/origin/master+.
Prvo, ovo \verb+remotes/+ znači da, iako toj grani imamo pristup na lokalnom repozitoriju, ona je \textbf{samo kopija grane} \verb+master+ \textbf{u repozitoriju} \verb+origin+.
Takve kopije udaljenih repozitorija ćemo uvijek označavati s \\ \verb+<naziv_repozitorija>/<naziv_grane>+.
Konkretno, ovdje je to \verb+origin/master+.

Dakle, grafički bi to mogli prikazati ovako:

\input{graphs/origin_master}

Imamo dva repozitorija, lokalni i udaljeni.
Udaljeni ima samo granu \verb+master+, a lokalni ima dvije kopije te grane. 
U lokalnom \verb+master+ ćemo mi \emph{commit}ati naše izmjene, a u \verb+origin/master+ se nalazi kopija udaljenog \verb+origin/master+ u koju \textbf{nećemo} \emph{commit}ati.
Ovaj \verb+origin/master+ ćemo, s vremenena na vrijeme, osvježavati tako da imamo ažurno stanje udaljenog repozitorija.

Ako vam ovo zvuči zbunjujuće, ništa čudno.
No, sve će sjesti na svoje mjesto kad to počnete koristiti.

\TODO: \verb+checkout origin/master+, log, diff, merge

\subsection*{Djelomično kloniranje povijesti repozitorija}
\addcontentsline{toc}{subsection}{Djelomično kloniranje povijesti repozitorija}

Našli ste na internetu neki zanimljiv projekt i njegov git repozitorij i htjeli bi ga skinuti i proučiti njegov kod. 
Ništa lakše; \verb+git clone ...+.

E, ali\dots
Tu imamo mali problem.
Git repozitorij sadrži cijelu povijest projekta. 
To znači da sadrži sve \emph{commit}ove koje su radili programeri i koji mogu sezati i preko deset godina unazad.
I zato \verb+git clone+ ponekad može potrajati dosta dugo. 
Posebno ako imate sporu vezu.

No, postoji trik.
Želimo li skinuti projekt samo zato da bi pogledali njegov kod a ne zanima nas cijela njegova povijest -- moguće je klonirati samo nekoliko njegovih zadnjih \emph{commit}ova s:

\gitoutputcommand{git clone --depth 5 --no-hardlinks git://github.com/tkrajina/uvod-u-git.git}

To će biti puno brže, no nećete imati pristup cijeloj povijesti, nego samo zadnjih $5$ \emph{commit}ova.

\TODO: Ograničenja takvog kloniranja

\section*{Digresija o grafovima, repozitorijima i granama}
\addcontentsline{toc}{section}{Digresija o grafovima, repozitorijima i granama}

Prije nego što nastavimo, napravio bih kratku digresiju koja je važna za razumijevanje grafova projekata i udaljenih projekata koji će slijediti.

Za razumijevanje nastavka ovog poglavlja važno je shvatiti da je graf\dots

\input{graphs/digresija_o_grafovima_1}

\dots{}ekvivalentan grafu\dots

\input{graphs/digresija_o_grafovima_2}

\dots{}zato što ima zajednički početak povijesti ($a$, $b$, $c$, $d$ i $e$).
Zbog toga, na primjer u sljedećoj situaciji:

\input{graphs/digresija_o_grafovima_3}

Trebamo si zamisliti da je odnos između našeg lokalnog \verb+master+ i udaljenog \verb+master+ kao da imamo jedan graf koji se granao u $e$. Jedino što se svaka grana nalazi u svom repozitoriju, a ne više u istom.
Zanemarimo li na trenutak \verb+origin/master+, odnos između naša dva \verb+master+a je ovakav:

\input{graphs/digresija_o_grafovima_4}

Opisani odnos vrijedi i za \verb+master+ i \verb+origin/master+.

U ilustracijama koje slijede, grafovi su prikazani kao zasebne "crte" samo zato što logički pripadaju nekakvim posebnim entitetima (repozitorijima).
Međutim, oni \textbf{su samo posebne grane istog projekta} iako se nalaze na različitim lokacijama/računalima.

\section*{Fetch}
\addcontentsline{toc}{section}{Fetch}

Što ako je vlasnik udaljenog repozitorija \emph{commit}ao u svoj \verb+master+.
Stanje je sad ovakvo:

\input{graphs/origin_master_2}

Naš, lokalni, \verb+master+ je radna verzija s kojom ćemo mi raditi -- tj. na koju ćemo \emph{commit}ati.
\verb+origin/master+ bi trebao biti lokalna kopija udaljenog \verb+master+, međutim -- ona se ne ažurira automatski.
To što je vlasnik udaljenog repozitorija dodao dva \emph{commit}a ($e$ i $f$) ne znači da će naš repozitorij nekom čarolijom to odmah saznati.

Git je zamišljen kao sustav koji ne zahtijeva stalni pristup internetu.
U većini operacija -- \textbf{od nas se očekuje da iniciramo interakciju s drugim repozitorijima}.
Bez da mi pokrenemo neku radnju, git neće nikad kontaktirati udaljene repozitorije.
Isto tako u drugom smjeru -- ni u jednoj situaciji neki drugi repozitorij ne može inicirati neki proces na našem repozitoriju.
Najviše što što vlasnik udaljenog repozitorija može napraviti je da nas \textbf{zamoli} da mi napravimo neku interakciju s njegovim repozitorijem.

Kao što smo mi inicirali kloniranje, tako i mi moramo inicirati ažuriranje grane \verb+origin/master+.
To se radi s \verb+git fetch+:

\input{git_output/git_fetch}

Nakon toga, stanje naših repozitorija je:

\input{graphs/origin_master_3}

Dakle, \verb+origin/master+ je osvježen tako da mu je stanje isto kao i \verb+master+ udaljenog repozitorija.

S \verb+origin/master+ možemo raditi skoro sve kao i s ostalim lokalnim granama.
Možemo, na primjer, pogledati njegovu povijest s:

\gitoutputcommand{git log origin/master}

Možemo pogledati razlike između njega i naše trenutne grane:

\gitoutputcommand{git diff origin/master}

Možemo se prebaciti da \verb+origin/master+, ali\dots

\input{git_output/git_checkout_origin_master}

Git nam ovdje dopušta prebacivanje na \verb+origin/master+, ali nam jasno daje do znanja da je ta grana ipak po nečemu posebna.
Kao što već znamo, ona nije zamišljena da s njome radimo direktno.
Nju možemo samo osvježavati stanjem iz udaljenog repozitorija.
U \verb+origin/master+ ne smijemo \emph{commit}ati.

Ima, ipak, jedna radnja koju trebamo raditi s \verb+origin/master+, a to je da izmjene iz nje preuzimamo u naš lokalni \verb+master+.
Prebacimo se na njega s \verb+git checkout master+ i\dots{}

\gitoutputcommand{git merge origin/master}

\dots{}i sad je stanje:

\input{graphs/origin_master_4}

I sad smo tek u \verb+master+ dobili stanje udaljenog \verb+master+.
Općenito, to je postupak kojeg ćemo često ponavljati:

\gitoutputcommand{git fetch}

\dots{}da bismo osvježili svoj lokalni \verb+origin/master+.
Sad tu možemo malo proučiti njegovu povijest i izmjene koje uvodi u povijest.
I onda\dots

\gitoutputcommand{git merge origin/master}

\dots{}da bi te izmjene unijeli u naš lokalni repozitorij.

Malo složeniji scenarij je sljedeći -- recimo da smo nakon\dots{}

\input{graphs/origin_master_2}

\dots{}mi \emph{commit}ali u naš lokalni repozitorij svoje izmjene. Recimo da su to $x$ i $y$:

\input{graphs/origin_master_2_2}

Nakon \verb+git fetch+, stanje je:

\input{graphs/origin_master_2_3}

Sad \verb+origin/master+ nakon $e$ ima $f$ i $g$, a \verb+master+ nakon $e$ ima $x$ i $y$.
U biti je to kao da imamom dvije grane koje su nastale nakon $e$. Jedna ima $f$ i $g$, a druga $x$ i $y$.
Ovo je, u biti, najobičniji \emph{merge} (koji će, eventualno, imati i neke konflikte. No, to već znamo riješiti).
Rezultat \emph{merge}a je novi čvor $z$:

\input{graphs/origin_master_2_4}

\section*{Pull}
\addcontentsline{toc}{section}{Pull}

U prethodnom poglavlju smo opisali da je tipični redosljed naredbi koje ćemo izvršiti svaki put kad želimo preuzeti izmjene iz udaljenog repozitorija:

\gitoutputcommand{git fetch\\git merge origin/master}

Obično ćemo nakon \verb+git fetch+ malo pogledati koje izmjene su došle s udaljenog repozitorija, no u konačnici ćemo ove dvije naredbe htjeti skoro uvijek izvršiti.

Zbog toga postoji "kratica" koja je ekvivalentna toj kombinaciji:

\gitoutputcommand{git pull}

\verb+git pull+ je upravo kombinacija \verb+git fetch+ i \verb+git merge origin/master+.

\section*{Push}
\addcontentsline{toc}{section}{Push}

Sve što smo do sada radili s gitom su bile radnje koje smo mi radili na našem lokalnom repozitoriju.
Čak i kloniranje je nešto što mi iniciramo i ničim nismo promijenili udaljeni repozitorij.
Krećemo sad na prvu (i ovdje jedinu) radnju s kojom aktivno mijenjamo neki udaljeni repozitorij.

Uzmimo, kao prvo, najjednostavniji mogući scenarij.
Klonirali smo repozitorij i stanje je, naravno:

\input{graphs/push_1}

Nakon toga smo \emph{commit}ali par izmjena\dots

\input{graphs/push_2}

\dots{}i sad bi htjeli te izmjene "prebaciti" na udaljeni repozitorij.
Prvo i osnovno što nam treba svima biti jasno -- "prebacivanje" naših lokalnih izmjena na udaljeni repozitorij ovisi o tome imate li ovlasti za to ili ne.
Udaljeni repozitorij mora biti tako konfiguriran da biste to mogli činiti.
Ukoliko nemate ovlasti, sve što možete napraviti je zamoliti njegovog vlasnika da pogleda vaše izmjene ($f$ i $g$) i da ih preuzme kod sebe.

Ukoliko, ipak, imate ovlasti, onda je ono što trebate napraviti:

\input{git_output/git_push_origin_master}

Stanje će sad biti:

\input{graphs/push_3}

Ovo je bio jednostavni scenarij u kojem smo u našem \verb+master+ \emph{commit}ali, ali u udaljenom \verb+master+ se nije ništa događalo.
Što da nije tako?
Dakle, dok smo mi radili na $e$ i $f$, vlasnik udaljenog repozitorija je \emph{commit}ao svoje $x$ i $y$:

\input{graphs/push_2_1}

Kad pokušamo \verb+git push origin master+, dogoditi će se ovakvo nešto:

\input{git_output/git_push_origin_master_rejected}

Kod nas lokalno $e$ clijede $f$ i $g$, a na udaljenom repozitoriju $e$ slijede $x$ i $y$. 
I git ovdje ne zna što točno napraviti, osim da mu netko pomogne.
Kao i do sada, pomoć se očekuje od nas, a tu radnju trebamo izvršiti na lokalnom repozitoriju.
Tek onda ćemo moći \emph{push}ati na udaljeni.

Riješenje je standardno:

\gitoutputcommand{git fetch}

\dots{}i stanje je sad:

\input{graphs/push_2_2}

Sad ćemo, naravno:

\gitoutputcommand{git merge origin/master}

Na ovom koraku se može desiti da imate konflikata koje eventualno treba ispraviti s \verb+git mergetool+.
Nakon toga, stanje je:

\input{graphs/push_2_3}

Sad možemo uredno napraviti:

\gitoutputcommand{git push origin master}

\dots{}a stanje naših repozitorija će biti:

\input{graphs/push_2_4}

Iako vam se može učiniti da udaljeni projekt nema izmjene koje smo mi uveli u $f$ i $g$ -- to nije točno.
Naime, $h$ je rezultat preuzimanja izmjena (\emph{merge}) iz \verb+origin/master+ i on u sebi sadrži kombinaciju i $x$ i $y$ i (naših) $f$ i $g$.

\TODO Push tagova

\subsection*{Rebase origin/master}
\addcontentsline{toc}{subsection}{Rebase origin/master}

Želimo li da se naši $f$ i $g$ (iz prethodnog primjera) vide u povijesti udaljenog projekta -- i to se može s:

\gitoutputcommand{git rebase origin/master\\git push origin master}

Ukoliko vam se ovo čini zbunjujuće -- još jednom dobro proučite "Digresiju o grafovima" koja se nalazi par stranica unazad.

\subsection*{Prisilan push}
\addcontentsline{toc}{subsection}{Prisilan push}

Vratimo se na ovu situaciju:

\input{graphs/push_2_2}

Da ponovimo još jednom.
Standardni postupak je \verb+git fetch+ (što je u gornjem primjeru već učinjeno) i \verb+git merge origin/master+.

Međutim, ovdje ima još jedna mogućnost.
Nakon što smo proučili ono što je vlasnik udaljenog repozitorija napravio u \emph{commit}ovima $x$ i $y$, ponekad ćemo zaključiti da to jednostavno ne valja. 
I najradije bi sada jednostavno "pregazili" njegove \emph{commit}ove u udaljenom \verb+master+.

To se može, komanda je:

\gitoutputcommand{git push -f origin master}

\dots{}a rezultat će biti:

\input{graphs/push_2_2_1}

I sad s \verb+git fetch+ možemo još i osvježiti stanje \verb+origin/master+.

Treba, međutim, imati na umu da ovakvo ponašanje nije baš uvijek poželjno. 
Zbog dva razloga:

\begin{itemize}
	\item \textbf{Mi} smo zaključili da \emph{commit}ovi $x$ i $y$ ne valjaju. Možda smo jednostavno pogriješili. Koliko god nam to bilo teško priznati, sasvim moguće je da jednostavno nismo dobro shvatili tuđi kod.
	\item Nitko ne voli da mu se pregaze njegove izmjene kao što smo to mi ovdje napravili vlasniku ovog udaljenog repozitorija. Bilo bi bolje javiti se njemu, objasniti mu što ne valja, predložiti bolje riješenje i dogovoriti da \textbf{on} \emph{revert}a, \emph{reset}ira ili ispravi svoje izmjene.
\end{itemize}

\section*{Rad s granama}
\addcontentsline{toc}{section}{Rad s granama}

Svi primjeri do sada su bili relativno jednostavni utoliko što su oba repozitorija (udaljeni i naš lokalni) imali samo jednu granu -- \verb+master+.
Idemo to sad (još) malo zakomplicirati i pogledati što se dešava ako kloniramo udaljeni repozitorij koji ima više grana:

Nakon \verb+git clone+ rezultat će biti:

\input{graphs/origin_master_s_granama}

Kloniranjem dobijamo samo kopiju lokalnog \verb+master+, dok se sve grane čuvaju pod \verb+origin/+. Dakle imamo \verb+origin/master+ i \verb+origin/test+.
Da je bilo više grana u repozitoriju kojeg kloniramo, imali bi više ovih \verb+origin/+ grana.
To lokalno možemo vidjeti s:

\input{git_output/git_branch_all_2}

Već znamo da se možemo "prebaciti" s \verb+git checkout origin/master+, ali se ne očekuje da tamo stvari i \emph{commit}amo.
Trebamo tu "\verb+remote+" granu granati u naš lokalni repozitorij i tek onda s njom počet raditi.
Dakle, u našem testnom slučaju, napravili bi:

\gitoutputcommand{git checkout origin/test\\
git branch test\\
git checkout test}

Zadnje dvije naredbe smo mogli skratiti u \verb+git checkout -b test+.

Sad je stanje:

\input{graphs/origin_master_s_granama_2}

\dots{}odnosno:

\input{git_output/git_branch_all_3}

Sad u taj lokalni \verb+master+ možemo \emph{commit}ati koliko nas je volja.
Kad ćemo htjeti poslati svoje izmjene na udaljenu granu, postupak je isti kao i do sada, jedino što radim s novom granom umjesto master.
Dakle,

\gitoutputcommand{git fetch}

\dots{}za osvježavanje \textbf{i} \verb+origin/master+ i \verb+origin/test+.
Zatim\dots

\gitoutputcommand{git merge origin/test}

I, na kraju\dots

\gitoutputcommand{git push origin test}

\dots{}da bi svoje izmjene "poslali" u granu \verb+test+ udaljenong repozitorija.

\TODO Zamijeniti naziv grane \verb+test+ u nešto smislenije

\subsection*{Git push i \emph{tracking} grana}
\addcontentsline{toc}{subsection}{Git push i \emph{tracking} grana}

\TODO

\section*{Udaljeni repozitoriji}
\addcontentsline{toc}{section}{Udaljeni repozitoriji}

Kloniranjem na našem lokalnom računalu dobijamo kopiju udaljenog repozitorija s jednim dodatkom -- ta kopija je pupčanom vrpcom vezana za originalni repozitorij.
Dobijamo referencu na \verb+origin+ repozitorij (onaj kojeg smo klonirali).
Dobijamo i dobijamo one \verb+origin/+ brancheve, koji su kopija udaljenih grana i mogućnost da osvježavamo njihovo stanje s \verb+git fetch+.

Imamo i nekoliko ograničenja, a najvažnije je činjenica što možemo imati samo jedan \verb+origin+.
Što ako želimo imati posla s više udaljenih repozitorija.
Dakle, što ako imamo više programera s kojima želimo surađivati od kojih svatko ima \textbf{svoj} repozitorij.

Drugim riječima, sad pomalo ulazimo u onu priču o gitu kao distribuiranom sustavu za verzioniranje.

\subsection*{Dodavanje i brisanje udaljenih repozitorija}
\addcontentsline{toc}{subsection}{Dodavanje i brisanje udaljenih repozitorija}

Svaki udaljeni repozitorij mora biti repozitorij istog projekta. 
Dakle, svi udaljeni repozitoriji s kojima ćemo imati posla su u nekom trenutku svoje povijesti nastali iz jednog jedinog projekta.
Bilo kloniranjem ili je netko kopirao nečiji projekt (ili samo njegov \verb+.git+ direktorij i iz njega \emph{checkout}ao projekt).

Ako dva repozitorija nisu nastali iz istog 

\subsection*{Fetch, merge, pull i push s udaljenim repozitorijima}
\addcontentsline{toc}{subsection}{Fetch, merge, pull i push s udaljenim repozitorijima}

\TODO "Bare" repozitorij

%\section*{}
%\addcontentsline{toc}{section}{}

