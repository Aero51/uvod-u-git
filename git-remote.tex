\chapter*{Udaljeni repozitoriji}
\addcontentsline{toc}{chapter}{Udaljeni repozitoriji}

Sve ono što smo do sada proučavali smo radili isključivo na lokalnom repozitoriju.
Samo smo spomenuli da je git distribuitani sustav za verzioniranje koda, složiti ćete se da je već krajnje vrijeme da krenemo pomalo obrađivati interakciju s udaljenim repozitorijima.

\section*{Kloniranje repozitorija}
\addcontentsline{toc}{section}{Kloniranje repozitorija}

\begin{itemize}
   \item Git repozitorij
   \item Kloniranje repozitorija
   \item Konfiguracija, certifikati
   \item --depth
   \item git, ssh i http(s)
\end{itemize}

\section*{Udaljeni repozitoriji}
\addcontentsline{toc}{section}{Udaljeni repozitoriji}

Push, fetch, pull, problemi koji mogu nastajati

tagovi, \dots

\begin{itemize}
   \item origin kao udaljeni repozitorij
   \item udaljeni repozitoriji, kako ih konfigurirati
   \item pushanje, fetchanje i pullanje s/na udaljenih repozitorija
\end{itemize}

%\section*{}
%\addcontentsline{toc}{section}{}

