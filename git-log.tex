\chapter*{Povijest}
\addcontentsline{toc}{chapter}{Povijest}

Već smo se poznali s naredbom \verb+git log+ s kojom se može vidjeti povijest \emph{commit}ova grane u kojoj se trenutno nalazimo, no ona nije dovoljna za proučavanje povijesti projekta.
Posebno s git projektima, čija povijest zna biti dosta kompleksna (puno grana, \emph{merge}anja, i sl.).

Sigurno će vam se desiti da želite vidjeti koje su se izmjene desile između predzadnjeg i pred-predzanjeg \emph{commit}a ili da vratite neku neku datoteku u stanje kakvo je bilo prije mjesec dana ili da proučite tko je zadnji napravio izmjenu na trinaestoj liniji nekog programa ili tko je prvi uveo funkciju koja se naziva \verb+get_image_x_size+ u projektu\dots 
Čak i ako vam se neki od navedenih scenarija čine malo vjerojatnim, vjerujte mi -- trebati će vam.

U ovom poglavlju ćemo proći neke često korištene naredbe za proučavanje povijesti projekta.

\section*{\emph{Diff}}
\addcontentsline{toc}{section}{\emph{Diff}}

S \verb+git diff+ smo se već sreli. 
Ukoliko ju pozovemo bez ikakvog dodatnog argumenta, ona će nam ispisati razlike između radne verzije repozitorija (tj. stanja projekta kakvo je trenutno na našem računalu) i zadnje verzije u repozitoriju.
Drugi način korištenja naredbe je da provjeravamo razlike između dva \emph{commit}a ili dvije grane (podsjetimo se da su grane u biti samo reference na zadnje \emph{commit}ove).
Na primjer:

\gitoutputcommand{git diff master tesna-grana}

\dots{}će nam ispisati razliku između te dvije grane. 
Treba paziti na redosljed jer je ovdje bitan.
Ukoliko isprobamo s:

\gitoutputcommand{git diff tesna-grana master}

\dots{}dobiti ćemo suprotan ispis. 
Ako smo u \verb+testna-grana+ jedan redak dodali -- u jednom slučaju će \verb+diff+ ispisati da smo ga \textbf{dodali}, a u drugom da smo ga \textbf{oduzeli}.

Želimo li provjeriti koje su izmjene dogodile između predzadnjeg i pred-predzadnjeg \emph{commit}a:

\gitoutputcommand{git diff HEAD\textasciitilde{}2 HEAD\textasciitilde{}1}

\dots{}ili između pred-predzadnjeg i sadašnjeg:

\gitoutputcommand{git diff HEAD\textasciitilde{}2}

\dots{}ili izmjene između \verb+974ef0ad8351ba7b4d402b8ae3942c96d667e199+ i \verb+testna-grana+:

\gitoutputcommand{git diff 974ef testna-grana}

\section*{\emph{Log}}
\addcontentsline{toc}{section}{\emph{Log}}

Standardno s naredbom \verb+git log <naziv_grane>+ dobijemo kratak ispis povijesti te grane.
Sad kad znamo da je grana u biti samo referenca na zadnji \emph{commit}, znamo i da bi bilo preciznije kazati da je ispravna sintaksa \verb+git log <referenca_na_commit>+.
Za git nije previše bitno jeste li mu dali naziv grane ili referencu na \emph{commit} -- naziv grane je ionako samo alias za zadnji \emph{commit} u toj grani.
Ukoliko mu damo neki proizvoljan \emph{commit}, on će jednostavno krenuti "unazad" po grafu i dati vam povijest koju na taj način nađe.

Dakle, ako želimo povijest trenutne grane, ali bez zadnjih pet unosa -- treba nam referenca na peti \emph{commit} unazad:

\gitoutputcommand{git log HEAD\textasciitilde{}5}

Ili, ako želimo povijest grane \verb+testna-grana+ bez zadnjih 10 unosa:

\gitoutputcommand{git log testna-grana\textasciitilde{}10}

Želimo li povijest sa \emph{samo} nekoliko zadnjih unosa, koristimo \\\verb+git log -<broj_ispisa>+ sintaksu. 
Na primjer, ako nam treba samo 10 rezultata iz povijesti:

\gitoutputcommand{git log -10 testna-grana}

\dots{}ili, ako to želimo za trenutnu granu, jednostavno:

\gitoutputcommand{git log -10}

\subsection*{\emph{Whatchanged}}
\addcontentsline{toc}{subsection}{\emph{Whatchanged}}

Naredba \verb+git whatchanged+ je vrlo slična \verb+git log+, jedino što uz svaki \emph{commit} ispisuje i popis svih datoteka koje su se tada promijenile:

\input{git_output/git_whatchanged}

\subsection*{Pretraživanje povijesti}
\addcontentsline{toc}{subsection}{Pretraživanje povijesti}

Vrlo često će vam se dogoditi da tražite neki \emph{commit} iz povijesti.
Ovdje ćemo proći samo dva najčešća slučaja.
Prvi je kad pretražujemo prema tekstu \emph{commit}a, tada se koristi \verb+git log --grep=<regularni_izraz>+.
Na primjer, tražimo li sve \emph{commit}ove koji \textbf{u \emph{commit} komentarima sadrže riječ} \verb+graph+:

\gitoutputcommand{git log --grep=graph}

Drugi česti scenarij je odgovor na pitanje "Kad se \textbf{u kodu} prvi put spomenuo string 'trla baba lan'"? Tada se koristi \verb+git log -S<string>+.
Dakle, ne u komentaru \emph{commit}a nego \textbf{u sadržaju datoteka}.
Recimo da tražimo tko je prvi napisao funkciju \verb+get_image_x_size+:

\gitoutputcommand{git log -Sget\_image\_x\_size}

Treba li pretraživati za string s razmacima:

\gitoutputcommand{git log -S"get image width"}

Zapamtite, ovo će vam samo naći \emph{commit}ove.
Kad ih nađemo, htjeti ćemo vjerojatno pogledati koje su točno bile izmjene.
Ako nam pretraživanje nađe da je \emph{commit} \\ \verb+76cf802d23834bc74473370ca81993c5b07c2e35+, detalji izmjena koje su se njime dogodile su:

\gitoutputcommand{git diff 76cf8 76cf8\textasciitilde{}1}

Podsjetimo se \verb+76cf8+ je kratica za \emph{commit}, a \verb+76cf8~1+ je referenca na njegovog \textbf{prethodnika} (zbog \verb+~1+).

\section*{\emph{Blame}}
\addcontentsline{toc}{section}{\emph{Blame}}

S \verb+git blame <datoteka>+ ćemo dobiti ispis neke datoteke s detaljima o tome \textbf{tko}, \textbf{kad} i u \textbf{kojem \emph{commit}u} je napravio (ili izmijenio) pojedinu liniju u toj datoteci\footnote{Nažalost, linije su preduge da bi se ovdje ispravno vidjelo.}:

\input{git_output/git_blame}

Nađete li liniju koja započinje znakom \verb+^+ -- to znači da je ta linija bila tu i u prvoj verziji datoteke.

U svakom projektu datoteke mijenjaju imena. 
Tako da kod koji je pisan u jednoj datoteci ponekad završi u datoteci nekog trećeg imena.
Ukoliko želimo znati i u kojoj su su datoteci linije naše trenutke datoteke prvi put pojavile, to se može s:

\gitoutputcommand{git blame -C <datoteka>}

\section*{Digresija o micanju datoteka}
\addcontentsline{toc}{section}{Digresija o micanju datoteka}

Vezano uz \verb+git blame -C+, napraviti ćemo jednu malu digresiju.
Pretpostavimo da Ivo i Marijana zajedno pišu pjesmu.
Ivo je napisao prvu verziju i datoteku s pjesmom nazvao \verb+pjesnici.txt+.
Nakon toga je Marijana pjesmu tu datoteku preimenovala u \verb+pjesnici-u-svijetu.txt+.
U trenutku kad je Marijana \emph{commit}ala svoju izmjenu -- sustav za verzioniranje je te izmjene vidio kao da je \verb+pjesnici.txt+ obrisana, a nova pjesma  \verb+pjesnici-u-svijetu+ dodana.

Problem je što je \verb+pjesnici-u-svijetu.txt+ dodana od strane Marijane.
Ispada kao da je ona \textbf{napisala} tu pjesmu, iako je samo preimenovala datoteku.

Zbog ovakve situacije neki drugi sustavi za verzioniranje (kao subversion ili mercurial) zahtijevaju od korisnika da \textbf{preko njihovih komandi} radi preimenovanje datoteke ili njeno micanje u neki drugi repozitorij.
Tako oni znaju da je Marijana \textbf{preimenovala} postojeću datoteku, ali sadržaj je napisao Ivan.
Ukoliko to ne učinite, nego datoteke preimenujete standardnim putem (naredbe \verb+mv+ ili \verb+move+) -- sustav za verzioniranje neće imati informaciju o tome da je preimenovana. 

To je problem, jer kod svakog micanja i preimenovanja datoteke moramo stalno paziti da se to napravi kroz naredbe sustava za verzioniranje.
Ni u kojem slučaju preko standardnih naredbi operativnog sustava jer, ukoliko to učinimo -- gubimo povijest sadržaja datoteke.

Problem je i kod \emph{merge}a.
Ako je u jednoj grani datoteka \textbf{preimenovana} a u drugoj samo \textbf{izmijenjena} -- sustav neće znati da se u stvari radi o istoj datoteci koja je promijenjena, on će to percipirati kao dvije različite datotekke.
Rezultat \emph{merge}a može biti neočekivan.

Spomenuti problem nije problem i u gitu.
Git ima ugrađenu heuristiku koja prati je li datoteka u nekom \emph{commit}u preimenovana.
Ukoliko u novom \emph{commit}u on nađe da je jedna datoteka \textbf{obrisana} -- proučiti će koje datoteke su u istom \emph{commit}u \textbf{nastale}. 
Ako se sadržaj poklapa u dovoljno velikom postotku linija, git će sam zaključiti da se radi o preimenovanju datoteke -- a ne o datoteci koja se prvi put pojavljuje u repozitoriju.
Isto vrijedi ako datoteku nismo preimenovali nego pomaknuli (i, eventualno, preimenovali) na novu lokaciju u repozitorij.

To je princip na osnovu kojeg \verb+git blame -C+ "zna" u kojoj datoteci se neka linija prvi put pojavila.
Zato bez straha možemo koristiti naredbe \verb+mv+, \verb+move+ ili manipulirati ih preko nekog IDE-a.

Nemojte da vas zbuni to što git \textbf{ima} komandu:

\gitoutputcommand{git mv <stara\_datoteka> <nova\_datoteka>}

\dots{}za preimenovanje/micanje datoteka.
Ta naredba ne radi ništa posebno \textbf{u gitu}, ona je samo \emph{alias} za standardnu naredbu operativnog sustava (\verb+mv+ ili \verb+move+).

Zbog toga naredba \verb+git merge+ radi ispravno čak i ako u jednoj grani datoteku preimenujemo i izmijenimo, a u drugoj samo izmijenimo -- git će sam zaključiti da se radi o istoj datoteci različitog imena.

\section*{Preuzimanje datoteke iz povijesti}
\addcontentsline{toc}{section}{Preuzimanje datoteke iz povijesti}

Svima nam se ponekad desilo da smo izmijenili datoteku i kasnije shvatili da ta izmjena ne valja. 
Verzija od prije 5 \emph{commit}ova je bila bolja nego ova trenutno.
Kako da ju vratimo iz povijesti i \emph{commit}amo u novo stanje projekta?

Znamo već da s \verb+git checkout <naziv_grane> -- <datoteka1> <datoteka2>...+ možemo lokalno dobiti stanje datoteke iz te grane.
Odnedavno znamo i da naziv grane nije ništa drugo nego referenca na njen zadnji \emph{commit}.
Ako umjesto grane, tamo stavimo referencu na neki drugi \emph{commit} dobiti ćemo stanje datoteke iz tog trenutka u povijesti.

Dakle, \verb+git checkout+ se, osim za prebacivanje s grane na granu, može koristiti i za pruzimanje neke datoteke iz prošlosti:

\gitoutputcommand{git checkout HEAD\textasciitilde{5} -- pjesma.txt}

\dots{}nam u trenutni direktorij vraća točno stanje datoteke \verb+pjesma.txt+ od prije $5$ \emph{commit}ova.
I sad smo slobodni tu datoteku opet \emph{commit}ati ili ju promijeniti i \emph{commit}ati.

Treba li nam da datoteka kakva je bila u predzadnjem \emph{commit}u grane \verb+test+?

\gitoutputcommand{git checkout test\textasciitilde{}1 -- pjesma.txt}

Isto tako i s bilo kojom drugom referencom na neki \emph{commit} iz povijesti.

\section*{"Teleportiranje" u povijest}
\addcontentsline{toc}{section}{"Teleportiranje" u povijest}

Isto razmatranje kao u prethodnom odjeljku vrijedi i za vraćanje stanja cijelog repozitorija u neki trenutak iz prošlosti.

Na primjer, otkrili smo bug, ne znate gdje točno u kodu, ali znamo da se taj bug nije prije manifestirao. 
Međutim, ne znamo točno \textbf{kada} je bug zepočeo.

Bilo bi zgodno kad bismo cijeli projekt mogli teleportirati na neko stanje kakvo je bilo prije $n$ \emph{commit}ova.
Ako se tamo bug i dalje manifestira -- vratiti ćemo se još malo dalje u povijest.
Ako se tamo manifestirao -- prebaciti ćemo se u malo bližu povijest.
I tako -- mic po mic po povijesti, sve dok ne nađemo trenutak (\emph{commit}) u povijesti repozitorija u kojem je bug stvoren.
Ništa lakše:

\gitoutputcommand{git checkout HEAD\textasciitilde{}10}

\dots{}i za trenutak imamo stanje kakvo je bilo prije $10$ \emph{commit}ova. Sad tu možemo s \verb+git branch+ kreirati novu granu ili vratiti se na najnovije stanje s \verb+git checkout HEAD+.

Ili možemo provjeriti je li bug i dalje tu. Ako jest, idemo probati s\dots

\gitoutputcommand{git checkout HEAD\textasciitilde{}20}

\dots{}ako buga sad nemamo, znamo da se pojavio negdje između \textbf{dvadesetog} i \textbf{desetog} zadnjeg \emph{commit}a.

\section*{\emph{Reset}}
\addcontentsline{toc}{section}{\emph{Reset}}

Uzmimo ovakav scenarij; radimo na nekoj grani.
U jednom trenutku stanje je:

\input{graphs/linearni_model_za_reset}

I sad zaključimo kako tu nešto ne valja -- svi ovi \emph{commit}ovi od $f$ pa na dalje su krenuli krivim smjerom.
Htjeli bi se nekako vratiti na:

\input{graphs/linearni_model_za_reset_2}

\dots{}i od tamo nastaviti cijeli posao, ali nekako drukčije.
E sad, lako je "teleportirati se" s \verb+git checkout ...+, to to ne mijenja stanje grafa -- $f$, $g$, $h$ i $i$ bi i dalje ostali u grafu.
Mi bi ovdje htjeli baš obrisati dio grafa, odnosno zadnjih nekoliko \emph{commit}ova.

Naravno da se i to može, a naredba je \verb+git reset --hard <referenca_na_commit>+.
Na primjer, ako se želimo vratiti na predzadnje stanje i potpuno obrisati zadnji \emph{commit}:

\gitoutputcommand{git reset --hard HEAD\textasciitilde{}1}

Želimo li se vratiti na \verb+974ef0ad8351ba7b4d402b8ae3942c96d667e199+ i maknuti sve \emph{commit}ove koji su se desili nakon njega.
Isto:

\gitoutputcommand{git reset --hard 974ef0a}

Postoji i \verb+git reset --soft <referenca>+. 
S tom varijantom se isto brišu \emph{commit}ovi iz povijesti repozitorija, ali stanje datoteka u našem radnom direktoriju ostaje kakvo jest.

Cijela poanta naredbe \verb+git reset+ je da \textbf{pomiče} \verb+HEAD+.
Kao što znamo, \verb+HEAD+ je referenca na zadnji \emph{commit} u trenutnoj grani.
Za razliku od nje, kad se "vratimo u povijest" s \verb+git checkout HEAD~2+ -- mi \textbf{nismo} dirali \verb+HEAD+.
Git i dalje zna koji mu je \emph{commit} \verb+HEAD+ i, posljedično, i dalje zna kako izgleda cijela grana.

U našem primjeru od početka, mi želimo maknuti \textcolor{red}{crvene} \emph{commit}ove. 
Ako prikažemo graf prema načinu kako git čuva podatke -- svaki \emph{commit} ima referencu na prethodnika, dakle strelice su suprotne od smjera nastajanja:

\input{graphs/linearni_model_za_reset_HEAD_1}

Uopće se ne trebamo truditi \textbf{brisati} čvorove/\emph{commit}ove $f$, $g$, $h$ i $i$.
Dovoljno je reći "Pomakni \verb+HEAD+ tako da pokazuje na $e$.
I to je upravo ono što git čini s \verb+git reset+:

\input{graphs/linearni_model_za_reset_HEAD_2}

Iako su ovi čvorovi ovdje prikazani na grafu, git do njih više ne može doći.
Da bi rekonstruirao svoj graf, on uvijek kreće od \verb+HEAD+, dakle $f$, $g$, $h$ i $i$ su izgubljeni.

\section*{\emph{Revert}}
\addcontentsline{toc}{section}{\emph{Revert}}

Svaki \emph{commit} mijenja povijest projekta, no on to čini tako da \textbf{dodaje na kraju grane}.
Treba imati na umu jednu stvar -- \verb+git reset+ \textbf{mijenja povijest projekta retrogradno}.
Ne dodaje samo čvor na kraj grane, on mijenja postojeću granu tako da obriše nekoliko njegovih zadnjih čvorova.
To može biti problem, a posebno u situacijama kad radite s drugim ljudima, o tome više u sljedećem poglavlju.

To se može riješiti s \verb+git revert+. Uzmimo, na primjer, ovakvu situaciju:

\input{graphs/linearni_model_za_revert_1}

Došli smo do zaključka da je \emph{commit} $f$ neispravan i treba vratiti na stanje kakvo je bilo u $e$.
Znamo već da bi \verb+git reset+ jednostavno maknuo $f$ s grafa, ali recimo da to ne želimo.
Tada se može napraviti sljedeće \verb+git revert <commit>+. 
Na primjer, ako želimo \emph{revert}ati samo zadnji \emph{commit}:

\gitoutputcommand{git revert HEAD}

Rezultat će biti da je dosadašnji graf ostao potpuno isti, no \textbf{dodan je novi commit} koji miče sve izmjene koje su uvedene u $f$:

\input{graphs/linearni_model_za_revert_1_revertano}

Dakle, stanje u $g$ će biti isto kao i u $e$.

To se može i sa \emph{commit}om koji nije zadnji u grani:

\input{graphs/linearni_model_za_revert_2}

Ako je SHA1 referenca \emph{commit}a $f$ \verb+402b8ae3942c96d667e199974ef0ad8351ba7b4d+, onda ćemo s:

\gitoutputcommand{git revert 402b8ae39}

\dots{}dobiti:

\input{graphs/linearni_model_za_revert_2_revertano}

\dots{}gdje \emph{commit} $j$ ima stanje kao u $i$, ali bez izmjena koje su uvedene u $f$.

Naravno, ako malo o tome razmislite, složiti ćete se da sve to radi idealno samo ako $g$, $h$ i $i$ \textbf{nisu dirali isti kod kao i $f$}.
\verb+git revert+ uredno radi ako revertamo \emph{commit}ove koji su bliži kraju grane.
Teško će, ako ikako, \emph{revert}ati stvari koje smo mijenjali prije dvije godine u kodu koji je, nakon toga, puno puta bio mijenjan.
Ukoliko to i pokušamo, git će javiti grešku i tražiti nas da mu sami damo do znanja kako bi \emph{revert} trebao izgledati.

\section*{Gitk}
\addcontentsline{toc}{section}{Gitk}

U standardnoj instalaciji gita dolazi i pomoćni programčić koji nam grafički prikazuje povijest trenutne grane.
Pokreće se s:

\gitoutputcommand{gitk}

\dots{}a izgleda ovako:

\gitgraphics{images/gitk.png}

Sučelje ima pet osnovnih dijelova.
Ovdje opisani redom s lijeva na desno od gore prema dolje:

\begin{itemize}
	\item grafički prikaz povijesti grane i \emph{commit}ova iz drugih grana koji su imali veze s tom granom (bilo zbog grananja, \emph{merge}anja ili \emph{cherry-pick} \emph{merge}),
	\item popis ljudi koji su \emph{commit}ali,
	\item datum i vrijeme, 
	\item pregled svih izmjena,
	\item pregled datoteka koje su izmijenjene.
\end{itemize}

Kad odaberete \emph{commit} dobiti ćete sve izmjene i datoteke koje su sudjelovale u izmjenama.
Kad odaberete datoteku, \verb+gitk+ će u dijelu sa izmjenama "skočiti" na izmjene koje su se dogodile na toj datoteci.

Gitk vam može i pregledati povijest samo do nekog \emph{commit}a u povijesti:

\gitoutputcommand{gitk HEAD\textasciitilde{}10}

\dots{}ili:

\gitoutputcommand{gitk 974ef0ad8351ba7b4d402b8ae3942c96d667e199}

\dots{}ili određene grane:

\gitoutputcommand{gitk testna-grana}

Gitk prikazuje povijest samo trenutne grane, odnosno samo onih \emph{commit}ova koji su na neki način sudjelovali u njoj.
Želimo li povijest svih grana -- treba ga pokrenuti su:

\gitoutputcommand{gitk --all}

Kao i \verb+git gui+, tako i \verb+gitk+ na prvi pogled možda baš i nije jako intuitivan.
No, brz je i praktičan za rad kad se naučite na njegovo sučelje.

%\section*{}
%\addcontentsline{toc}{section}{}
