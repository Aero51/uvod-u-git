\chapter*{Povijest}
\addcontentsline{toc}{chapter}{Povijest}

Već smo se poznali s naredbom \verb+git log+ s kojom se može vidjeti povijest \emph{commit}ova grane u kojoj se trenutno nalazimo, no ona zasigurno nije dovoljna za proučavanje povijesti projekta.
Posebno s git projektima, čija povijest zna biti dosta kompleksna (puno grana, \emph{merge}anja, isl.).

Sigurno će vam se ponekad desiti da želite vidjeti koje se se izmjene desile između predzadnjeg i pred-predzanjeg \emph{commit}a ili da vratite neku neku datoteku u stanje kakvo je bilo prije mjesec dana ili da proučite tko je zadnji napravio izmjenu na trinaestoj liniji nekog programa ili tko je prvi uveo funkciju koja se naziva \verb+get_image_x_xizse+ u projektu\dots 
Čak i ako vam se neki od navedenih scenarija, vjerujte mi -- trebati će vam.

U ovom poglavlju ćemo proći samo neke od često korištenih varijanti naredbi za proučavanje povijesti projekta.

\section*{Diff}
\addcontentsline{toc}{section}{Diff}

Važna naredba je i \verb+git diff+. 
S njome provjeravate razlike između dva \emph{commit}a.
Na primjer:

\gitoutputcommand{git diff master tesna-grana}

\dots{}će nam ispisati razliku između dvije grane. Pripazite, jer redosljed je ovdje bitan.
Ukoliko isprobate s:

\gitoutputcommand{git diff tesna-grana master}

\dots{}dobiti ćete suprotan ispis. 
Ako ste u \verb+testna-grana+ jedan redak dodali -- u jednom slučaju će \verb+diff+ ispisati da ste ga dodali, a u drugom oduzeli.

Želite li provjeriti koje su izmjene dogodile između predzadnjeg i pred-predzadnjeg commita:

\gitoutputcommand{git diff HEAD\textasciitilde{}2 HEAD\textasciitilde{}1}

\dots{}ili između pred-predzadnjeg i sadašnjeg:

\gitoutputcommand{git diff HEAD\textasciitilde{}2}

\dots{}ili izmjene između \verb+974ef0ad8351ba7b4d402b8ae3942c96d667e199+ i \verb+testna-grana+:

\gitoutputcommand{git diff 974ef testna-grana}

\section*{Log}
\addcontentsline{toc}{section}{Log}

Standardno s \verb+git log <naziv_grane>+ će vam ispisati povijest te grane.
Sad kad znamo da je grana u biti samo referenca na zadnji \emph{commit}, znamo i da bi bilo preciznije kazati da je ispravna sintaksa \verb+git log <referenca_na_commit>+.
Za git nije previše bitno jeste li mu dali naziv grane ili referencu na \emph{commit}, on će jednostavno krenuti "unazad" po grafu i dati vam povijest koju na taj način nađe.
Pa tako, ako želimo povijest trenutne grane, ali bez zadnjih pet unosa, pitati ćemo jednostavno:

\gitoutputcommand{git log HEAD\textasciitilde{}5}

Ili, ako želimo povijest grane \verb+testna-grana+ bez zadnjih 10 unosa:

\gitoutputcommand{git log testna-grana\textasciitilde{}10}

Želimo li povijest sa \emph{samo} nekoliko zadjnih unosa, koristimo \verb+git log -n+ sintaksu:

\gitoutputcommand{git log -10 testna-grana}

\dots{}ili, ako to želimo za trenutnu granu, jednostavno:

\gitoutputcommand{git log -10}

\subsection*{Pretraživanje povijesti}
\addcontentsline{toc}{subsection}{Pretraživanje povijesti}

Vrlo često će vam se dogoditi da tražite neki \emph{commit} iz povijesti.
Ovdje ćemo proći samo dva najčešća slučaja.
Prvi je kad pretražujete prema tekstu \emph{commit}a, tada se koristi \verb+git log --grep=<regularni_izraz>+.
Na primjer, tražim li sve \emph{commit}ove koji u sebi sadrže riječ \verb+graph+:

\gitoutputcommand{git log --grep=graph}

Drugi česti scenarij je odgovor na pitanje "Kad se u kodu prvi put spomenuo string 'x'"? Tada se koristi \verb+git log -S<string>+.
Recimo da tražite tko je prvi napisao funkciju \verb+get_image_width+:

\gitoutputcommand{git log -Sget\_image\_width}

Treba li pretraživati za string s razmacima:

\gitoutputcommand{git log -S"get image width"}

Zapamtite, ovo će vam samo naći \emph{commit}ove.
Kad ih nađete, sigurno ćete htjeti pogledati koje su točno bile izmjene.
Ako vam pretraživanje nađe da je \emph{commit} \verb+76cf802d23834bc74473370ca81993c5b07c2e35+, detalji izmjena koje su se njime dogodile su:

\gitoutputcommand{git diff 76cf8 76cf8\textasciitilde{}1}

\section*{Blame}
\addcontentsline{toc}{section}{Blame}

S \verb+git blame <datoteka>+ ćete dobiti ispis datoteke s detaljima o tome \textbf{tko}, \textbf{kad} i u \textbf{kojeg \emph{commit}u} je napravio svaku liniju u toj datoteci i \textbf{iz koje datoteke} je ta izmjena došla ovdje:

\input{git_output/git_blame}

\section*{Whatchanged}
\addcontentsline{toc}{section}{Whatchanged}

Naredba \verb+git whatchanged+ je vrlo slična \verb+git log+, jedino što uz svaki \emph{commit} ispisuje i spisak svih datoteka koje su se tada promijenile:

\input{git_output/git_whatchanged}

\section*{Preuzimanje datoteke iz povijesti}
\addcontentsline{toc}{section}{Preuzimanje datoteke iz povijesti}

Znamo već da s \verb+git checkout <naziv_grane> -- <datoteka1> <datoteka2>...+ možemo lokalno dobiti stanje datoteke iz te grane.
Odnedavno znamo i da naziv grane nije ništa drugo nego referenca na njen zadnji \emph{commit}.
Analogno, ako umjesto grane, tamo stavimo referencu na neki drugi \emph{commit} dobiti ćemo stanje datoteke iz tog trenutka u povijesti.

Dakle, s:

\gitoutputcommand{git checkout HEAD\textasciitilde{}10 -- pjesma.txt}

\dots{}dobijamo točno stanje datoteke \verb+pjesma.txt+ od prije $10$ \emph{commit}ova.

Isto tako i s bilo kojom drugom referencom na neki \emph{commit} iz povijesti.

\section*{Reset}
\addcontentsline{toc}{section}{Reset}

\section*{Revert}
\addcontentsline{toc}{section}{Revert}

\section*{Gitk}
\addcontentsline{toc}{section}{Gitk}

%\section*{}
%\addcontentsline{toc}{section}{}
