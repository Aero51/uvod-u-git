\chapter*{Ispod haube}
\addcontentsline{toc}{chapter}{Ispod haube}

\section*{Kako biste vi\dots}
\addcontentsline{toc}{section}{Kako biste vi\dots}

Da kojim slučajem danas morate dizajnirati i implementirati sustav za verzioniranje koda, kako biste to napravili?
Kako biste čuvali povijest svake datoteke?

Prije nego što ste počeli koristiti takve sustave, vjerojatno ste radili sljedeće: kad bi zaključili da ste došli do nekog važnog stanja u projektu, kopirali bi cijeli projekt u direktorij naziva \verb+projekt_backup+ ili \verb+projekt_2012_04_05+ ili neko drugo slično ime.
Rezultat je da ste imali gomilu sličnih "backup" projekata.
Svaki direktorij predstavlja neko stanje projekta (dakle to je \emph{commit}.
I to je nekakvo verzioniranje koda, no puno toga nedostaje.

Na primjer, nemate komentare uz \emph{commit}ove, ali to bi se moglo srediti tako da u svaki direktorij spremite datoteku naziva \verb+komentar.txt+.
Nemate niti nekakav graf, odnosno redosljed kako su direktorij nastajali jedan iz drugog.
I to bi se moglo riješit tako da u svakom direktoriju u nekoj posebnoj datoteci, npr. \verb+parents+ nabrojite nazive direktorija koji su "roditelji" trenutnom direktoriju.

No, sve je to prilično neefikasno što se tiče diskovnog prostora. 
Imate li u repozitoriju jednu datoteku veličine 100 kilobajta koju \emph{nikad} ne mijenjate, njena kopija će opet zauzimati 100 kilobajta u svakoj kopiji projekta.
To je gnjavaža.

Zato bi možda bilo bolje da umjesto \textbf{kopije direktorija} za \emph{commit} napravimo novi u kojeg ćemo staviti \textbf{samo one datoteke koje su izmijenjene}.
Zahtijevalo bi malo više posla jer morate točno znati koje su datoteke izmijenjene, ali i to se može riješiti.
Mogli bi napraviti neku jednostavno \emph{shell} skriptu koja će to napraviti a nas. 

S time bi se problem diskovnog prostora drastično smanjio. 
Rezultat bi mogli još malo poboljšati tako da datoteke kompresiramo.

Još jedna varijanta bi bila da ne čuvate izmijenjene datoteke, nego samo izmijenjene linije koda.
Tako bi, vaše datoteke, umjesto cijelog sadržaja imale nešto tipa "Peta linija izmijenjena iz 'def suma\_brojeva()' u 'def zbroj\_brojeva()'".
To su takozvane "delte".
Još jedna varijanta bi bila da ne radite kopije direktorija, nego sve snimate u jednom tako da za svaku datoteku čuvate originalnu verziju i nakon toga (u istoj datoteci) dodajete delte.
Onda će vam trebati nekakav pomoćni programčić kako iz povijesti izvući zadnju verziju bilo koje datoteke, jer on mora izračunati sve delte od početne verzije.

Sve te varijante imaju jedan suptilni, ali neugodan, problem.
Problem konzistencije.

Vratimo se na trenutak na ideju s kopiranjem direktorija sa izmijenjenim datotekama.
Ako svaki direktorij sadrži samo izmijenjene datoteke, onda prvi direktorij mora sadržavati \emph{sve} datoteke.
Pretpostavite da imate jednu datoteku koja nije nikad izmijenjena od prve do zadnje verzije. 
Ona će se nalaziti samo u prvom (originalnom) direktoriju.

Što ako neki haker upadne u vaš sustav i izmijeni takvu datoteku?
Razmislimo malo, on je upravo izmijenio ne samo početno stanje takve datoteke nego je \textbf{promijenio cijelu njenu povijest}!
Kad bi vi svoj sustav pitali "daj mi zadnju verziju projekta", on bi protrčao kroz sva stanja projekta i dao bi vam hakerovu varijantu.
Jeste li sigurni da bi primijetili podvaljenu datoteku?

Riješenje je da je vaš sustav nekako dizajniran tako da sam prepoznaje takve izmijenjene datoteke. 
Odnosno, da je tako dizajniran da, ako bilo tko promijeni nešto u povijesti -- sam sustav prepozna da nešto s njime ne valja.
To se može na sljedeći način: neka jedinstveni identifikator svakog \emph{commit}a bude neki podatak koji je \textbf{izračunat} iz sadržaja i koji jedinstveno određuje sadržaj.
Takav jedinstveni identifikator će se nalaziti u grafu projekta, i sljedeći \emph{commit}ovi će znati da im je on prethodnik.

Ukoliko bilo tko promijeni sadržaj nekog \emph{commit}a, onda on više neće odgovarati tom identifikatoru.
Promijeni li i identifikator, onda graf više neće biti konzistentan -- sljedeći \emph{commit} će sadržavati identifikator koji više ne postoji.
Dakle, haker bi trebao promijeniti sve \emph{commit}ove do zadnjeg. 
U biti, trebao bi promijeniti previše stvari da bi mu cijeli poduhvat mogao proći nezapaženo.

Još ako je naš sustav distribuiran (dakle i drugi korisnici imaju povijest cijelog projekta) onda mu je još teže -- jer tu radnju mora ponoviti na računalima svih ljudi koji imaju kopije.
S distribuiranim sustavima, nitko nikad niti ne zna tko sve ima kopije.

Znate li malo matematike čuli ste za jednosmerne funkcije.
Ako i niste, nije bitno. 
To su funkcije koje je lako izračunati, ali je izuzetno teško iz rezultat zaključiti kakav je mogao biti rezultat.
Takve su, na primjer, \emph{hash} funkcije, a jedna od njih je SHA1.

\section*{SHA1}
\addcontentsline{toc}{section}{SHA1}

Sa SHA1 možete uzeti bilo koji string i on će vam iz njega dati novi string duljine 40 znakova.
Primjer takvog stringa je \verb+974ef0ad8351ba7b4d402b8ae3942c96d667e199+.
Izgleda poznato?

SHA1 ima sljedeća svojstva:

\begin{itemize}
	\item \emph{Nije} jedinstvena. Dakle, sigurno postoje različiti ulazni stringovi koji daju isti rezultat, no \textbf{praktički ih je nemoguće naći}. 
	\item Kad dobijete rezultat funkcije (npr. \verb+974ef0ad8351ba7b4d402b8ae3942c96d667e199+) is njega je \textbf{praktički nemoguće izračunati string iz kojeg je nastala}.
\end{itemize}

Takvih $40-$znamenkastih stringova ćete vidjeti cijelu gomilu u \verb+.git+ direktoriju.

Git nije ništa drugo nego graf SHA1 stringova, od kojih svaki jedinstveno identificira neko stanje projekta \textbf{i izračunati su iz tog stanja}.
Osim SHA1 identifikatora git uz svaki \emph{commit} čuva i neke metapodatke kao, na primjer:

\begin{itemize}
	\item Datum i vrijeme kad je nastao.
	\item Komentar
	\item SHA1 \emph{commit}a koji mu je prethodio
	\item SHA1 \emph{commit}a iz kojeg su preuzete izmjene za \emph{merge} (\textbf{ako} je taj \emph{commit} rezultat \emph{merge}a).
	\item \dots
\end{itemize}

\section*{Grane}
\addcontentsline{toc}{section}{Grane}

Razmislimo o još jednom detalju, uz poznati graf:

\input{graphs/primjer_s_imenovanim_granama_i_spajanjima}

Graf kao ovaj gore matematičari zovu usmjereni graf jer su veze između čvorova usmjerene: $\vec{ab}$, $\vec{bc}$, itd.
Znamo već da svaki čvor tov grafa predstavlja stanje nekog projekta, a svaka strelica neku izmjenu u novo stanje.

Sad kad znamo ovo malo pozadine oko toga kako git interno pamti podatke, mogli jednako tako ovaj gornji graf prikazati ovako:

\input{graphs/primjer_s_imenovanim_granama_i_spajanjima_suprotne_strelice}

Sve strelice su ovdje usmjerene suprogno negoli u grafovima kakve smo do sada imali.
Naime, čvor \emph g ima referencu na na \emph g ($\vec{hg}$), \emph g ima reference na \emph f ($\vec{gh}$) i na \emph q ($\vec{gq}$), itd.
Uočite da nam uopće nije potrebno znati da se grana \verb+novi-feature+ sastoji od \emph x, \emph y, \emph z, \emph q i \emph w.
Dovoljan nam je $w$.
Iz njega možemo, prateći reference "unazad" (suprotno od kronološkog reda nastajanja) doći sve do mjesta gdje je grana nastala.

Zato gitu interno grane i nisu ništa drugo neki reference na njihove zadnje \emph{commit}ove.

\section*{.git direktorij}
\addcontentsline{toc}{section}{.git direktorij}

\input{git_output/dot_git_folder}

\input{git_output/find_dot_git_refs}

.git/refs

HEAD

.git/objects

.git/hooks

%\section*{}
%\addcontentsline{toc}{section}{}

