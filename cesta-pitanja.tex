\tocChapter{Česta pitanja}

Jedno je razumijeti naredbe i terminologiju gita, a potpuno drugo je imati iskustvo u radu s gitom.
Da bi nekako došli do iskustva, trebamo imati osjećaj o tome koji su problemi koji se pojavljuju u radu i trebamo automatizirati postupak njihovog riješavanja.
U ovom poglavlju ćemo proći nekoliko takvih "situacija".

\tocSection{Jesmo li $push$ali svoje izmjene na udaljeni repozitorij?}

S klasičnim sustavima za verzioniranje, kod kojeg smo izmijenili može biti ili lokalno ne$commit$an ili $commit$an na centralnom repozitoriju.

Kao što sad već znamo, s gitom je stvar za nijansu složenija.
Naš kod može biti lokalno ne$commit$an, može biti $commit$an na našem lokalnom repozitoriju, a može biti i $push$an na udaljeni repozitorij.
Više puta mi se desilo da se kolega (koji tek uči git) pita "Kako to da moje izmjene nisu završile na produkciji\footnote{\dots{}ili produkcijskom $build$u.}, \textbf{iako sam ih $commit$ao}?".
Odgovor je jednostavan -- $commit$ao ih je lokalno, ali nije $push$ao na naš glavni repozitorij.

Problem kojeg on ima je u tome što nije nigdje jasno vidljivo jesu li izmjene iz njegove \verb+master+ grane $push$ane na udaljeni repozitorij.

Jedan jednostavan način da to provjerite je da provjerite odnos između \verb+master+ i \verb+origin/master+.
Za svaki slučaj, prvo ćemo osvježiti stanje udaljenog repozitorija s:

\gitoutput{git fetch}

\dots{}i sad idemo \textbf{vizualno} proučiti odnos između naše dvije grane:

\gitoutput{gitk master origin/master}

Sad pogledajte na grafu je li:
\begin{itemize}
    \item \verb+master+ \textbf{ispred} \verb+origin/master+, u tom slučaju vi imate više $commit$ova od udaljenog repozitorija i možete ih $push$ati,
    \item \verb+master+ \textbf{iza} \verb+origin/master+, u tom slučaju vi imate manje $commit$ova od udaljenog repozitorija i trebate ih pokupiti s udaljenog repozitorija ($pull$ ili $rebase$),
    \item \verb+master+ i \verb+origin/master+ se nalaze na dvije grane koje su međusobno divergirale (u tom slučaju vi \textbf{imate} ne$commit$ane izmjene, ali trebate prvo napraviti $pull$ i onda ih $push$ati).
\end{itemize}

Primjer situacije gdje nemamo ništa za $push$ati na na udaljeni repozitorij (nego čak imamo nešto za \textbf{preuzeti} iz njega):

\gitgraphics{images/origin_master_ispred_master.png}{5cm}

Primjer gdje imamo dva $commit$a koje nismo $push$ali, a mogli bismo:

\gitgraphics{images/master_ispred_origin_master.png}{5cm}

Primjer gdje imamo tri $commit$a za $push$anje, ali trebamo prije toga preuzeti četiri $commit$a it \verb+origin/master+ i $merge$ati ih u našu granu:

\gitgraphics{images/master_i_origin_master_divergirani.png}{5cm}

I, situacija u kojoj je lokalni \verb+master+ potpuno isti kao udaljeni \verb+origin/master+:

\gitgraphics{images/master_i_origin_master_isti.png}{6.5cm}

\tocSection{$Commit$ali smo u krivu granu}

\TODO

\tocSection{$Commit$ali smo u granu X, ali te commitove želimo prebaciti u novu granu}

\TODO

\tocSection{Imamo ne$commitane$ izmjene i git nam ne da prebacivanje na drugu granu}

\TODO

\tocSection{Zadnjih $n$ $commit$ova treba "stisnuti" u jedan $commit$}

\TODO

\tocSection{$Push$ali smo u remote repozitorij izmjenu koju nismo htjeli}

\TODO

