\chapter*{Spremanje izmjena}
\addcontentsline{toc}{chapter}{Spremanje izmjena}

Vratimo se na trenutak na naša dva primjera, linearni model verzioniranja koda:

\input{graphs/linearni_model}

\dots{}i primjer s granama:

\input{graphs/primjer_s_granama_i_spajanjima}

U oba slučaja, čvorovi grafova su stanja projekta u nekom trenutku.
Na primjer, kad smo prvi put inicirali projekt s \verb+git init+, dodali smo nekoliko datoteka i \textbf{spremili ih}. 
U tom trenutku je nastao čvor \emph a.
Nakon toga smo možda izmijenili neke od tih datoteka, možda neke obrisali, neke nove dodali i opet -- spremili novo stanje i dobili stanje \emph b.

To što smo radili između svaka dva stanja (tj. čvora) je naša stvar i ne tiče se gita\footnote{Neki sustavi za verzioniranje, kao na primjer TFS, zahtijevaju stalnu vezu na internet i serveru dojavljuju svaki put kad krenete editirati neku datoteku. Git nije takav.}.
Trenutak kad se odlučimo spremiti novo stanje projekta u naš repozitorij -- to je gitu jedino važno i to se zove \emph{commit}.

Važno je ovdje napomenuti da u gitu, za razliku od subversiona, CVS-a ili TFS-a \textbf{nikad ne commitamo u udaljeni repozitorij}. 
Svoje lokalne promjene \emph{commit}amo, odnosno spremamo u \textbf{lokalni} repozitorij na našem računalu.
Interakcija s udaljenim repozitorijem će biti tema poglavlja o udaljenim repozitorijima.

\section*{Status}
\addcontentsline{toc}{section}{Status}

Da bismo provjerili imamo li uopće nešto za spremiti, koristi se naredba \verb+git status+.
Na primjer, kad na projektu koji nema lokalnih izmjena za spremanje utipkamo \verb+git status+, dobiti ćemo:

\input{git_output/git_status_0}

Recimo da smo napravili tri izmjene na projektu:
Izmijenili smo datoteke \verb+README.md+ i \verb+setup.py+ i obrisali \verb+TODO.txt+:
Sad će rezultat od \verb+git status+ izgledati ovako:

\input{git_output/git_status_1}

Najbitniji podatak su linije u kojima piše \verb+modified:+ i \verb+deleted:+, jer to su datoteke koju smo \emph{mijenjali}, ali ne još \emph{commit}ali.

Želimo li pogledati \textbf{koje su točne razlike} u tim datotekama u odnosu na stanje kakvo je snimljeno u repozitoriju, odnosno u \textbf{zadnjoj verziji} repozitorija, to možemo dobiti s \verb+git diff+. 
Primjer ispisa te naredbe je:

\input{git_output/git_diff_1}

Najvažniji dijelovi su linije koje počinju si \verb+diff+, jer one govore o kojim datotekama se radi.
Nakon njih slijedi nekoliko linija s općenitim podacima i zatim kod \textbf{oko} dijela datoteke koji je izmijenjen i onda ono najvažnije -- linije obojane u crveno i one obojane u plavo.

Linije koje započinju s "-" (crvene) su linije koje su obrisane, a one koje počinju s "+" (u plavom) su one koje su dodane. 
Primijetite da git ne zna da smo neku liniju izmijenili. 
Ukoliko jesmo -- on se ponaša kao da smo staru obrisali, a novu dodali.

Rezultat \verb+diff+ naredbe su samo linije koda koje smo izmijenili i nekoliko linija \textbf{oko njih}.
Ukoliko želimo malo veću tu "okolinu" oko naših izmjena, možemo ju izmijeniti s opcijom \verb+-U<broj_linija>+.
Na primjer, ukoliko želimo $10$ linija oko izmjenjenih dijelova koda, to ćemo dobiti sa:

\gitoutputcommand{git diff -U10}

\section*{Indeks}
\addcontentsline{toc}{section}{Indeks}

Iako često govorimo o tome kako ćemo "\emph{commit}ati datoteku" ili "staviti datoteku u indeks" ili\dots -- treba imati na umu da git ne čuva "datoteke" (kao nekakav apstraktni pojam) nego stanja, odnosno \textbf{verzije datoteka}.
Dakle, za jednu te istu datoteku -- git čuva njena različita stanja, kako se mijenjala kroz povijest.
Mi datoteke u našem projektu mijenjamo, a sami odlučujemo u kojem trenutku su one takve da bismo ih snimili.

U gitu postoji poseban "međuprostor" u koji se "stavljaju" datoteke koje ćemo spremiti (\emph{commit}ati).
Dakle, sad imamo tri različita "mjesta" u kojima se čuvaju datoteke -- odnosno konkretna stanja pojedinih datoteka:

\begin{description}
	\item[Git repozitorij] čuva različita stanja iste datoteke (\textbf{povijest} datoteke).
	\item[Radna verzija repozitorija] je stanje datoteka u našem direktoriju. Ono može biti isto ili različito u odnosu na stanje datoteka u repozitoriju.
	\item[Poseban "međuprostor" \emph{commit}] gdje privremeno spremamo trenutno stanje datoteka prije nego što ih \emph{commit}amo.
\end{description}

Ovo zadnje stanje, odnosno, taj "međuprostor za \emph{commit}" se zove \emph{index} iliti indeks.
U literaturi ćete često naći i naziv \emph{staging area} ili \emph{cache}\footnote{Nažalost, git ovdje nije konzistentan pa i u svojoj dokumentaciji ponekad koristi \emph{stage}, a ponekad \emph{cache}.}.
Naredba \verb+git status+ je upravo namijenjena pregledavanju statusa indeksa i radne verzije projekta.
Na primjer, u trenutku pisanja ovog poglavlja, \verb+git status+ je:

\input{git_output/git_status_prije_indeksa}

Ovaj ispis govori kako je jedna datoteka izmijenjena, ali nije još \emph{commit}ana niti stavljena u indeks.

Ukoliko je stanje na radnoj verziji našeg projekta potpuno isto kao i u zadnjoj verziji git repozitorija, onda će nas \verb+git status+ obavijestiti da nemamo ništa za \emph{commit}ati.
U suprotnom, reći će koje datoteke su izmijenjene, a na nama je da sad u indeks stavimo (samo) one datoteke koje ćemo u sljedećem koraku \emph{commit}ati.

Trenutno stanje direktorija s projektom ćemo u nastavku referencirati kao \textbf{radna verzija projekta}.
Radna verzija projekta može, ali i ne mora, biti jednaka stanju projekta u repozitoriju ili indeksu.

\subsection*{Spremanje u indeks}
\addcontentsline{toc}{subsection}{Spremanje u indeks}

Recimo da smo promijenili datoteku \verb+uvod.tex+\footnote{To je upravo datoteka u kojem se nalazi poglavlje koje trenutno čitate.}.
Nju možemo staviti u indeks s:

\gitoutputcommand{git add uvod.tex}

\dots{}i sad je status:

\input{git_output/git_status_nakon_indeksa}

Primijetite dio u kojem piše: \verb+Changes to be committed+ -- e to je popis datoteka koje smo stavili u indeks.

Nakon što smo datoteku spremili u indeks -- spremni smo za \emph{commit} ili možemo nastaviti dodavati druge datoteke s \verb+git add+ sve dok se ne odlučimo za snimanje.

Čak i ako smo datoteku \emph{obrisali} -- moramo ju dodati u indeks naredbom \verb+git add+.
Ako vas to zbunjuje -- podsjetimo se da \textbf{u indeks ne stavljamo u stvari datoteku nego neko njeno (izmijenjeno) stanje}.
Kad smo datoteku obrisali, u indeks treba spremiti novo stanje te datoteke -- "izbrisano stanje".

\verb+git add+ ne moramo nužno koristiti s jednom datotekom.
Ukoliko spremamo cijeli direktorij datoteka, možemo ga dodati s:

\gitoutputcommand{git add naziv\_direktorija/*}

Ili, ako želimo dodati apsolutno sve što se nalazi u našoj radnoj verziji:

\gitoutputcommand{git add .}

\subsection*{Micanje iz indeksa}
\addcontentsline{toc}{subsection}{Micanje iz indeksa}

Recimo da smo datoteku stavili u indeks i kasnije se predomislili -- lako ju iz indeksa maknemo naredbom:

\gitoutputcommand{git reset HEAD -- <datoteka1> <datoteka2> \dots}

Događati će nam se situacija da smo promijenili neku datoteku, no kasnije zaključimo da ta izmjena nije bila potrebna. 
I sad ju ne želimo spremiti nego vratiti u prethodno stanje -- odnosno točno onakvo stanje kakvo je u zadnjoj verziji repozitorija.
To se može ovako:

\gitoutputcommand{git checkout HEAD -- <datoteka1> <datoteka2> \dots}

Više detalja o \verb+git checkout+ i zašto ta gornja naredba radi to što radi će biti kasnije.

\subsection*{O indeksu i stanju datoteka}
\addcontentsline{toc}{subsection}{O indeksu i stanju datoteka}

Ima još jedan detalj koji bi vas mogao zbuniti. 
Uzmimo situaciju da smo samo jednu datoteku izmijenili i spremili u indeks:

\input{git_output/git_status_nakon_indeksa}

Izmijenimo li tu datoteku direktno u projektu -- novo stanje će biti ovakvo:

\input{git_output/git_status_nakon_indeksa_2}

Dotična datoteka je sad u radnoj verziji označena kao izmijenjena, ali ne još stavljena u indeks.
Istovremeno, ona je i u indeksu!
Iz stvarnog svijeta smo naviknuti da jedna stvar ne može biti na dva mjesta, međutim iz dosadašnjeg razmatranja znamo da gitu nije toliko bitna datoteka (kao apstraktan pojam) nego \textbf{konkretne verzije (ili stanja) datoteke}.
I, u tom smislu, nije ništa neobično da imamo jedno stanje datoteke u indeksu, a drugo stanje datoteke u radnoj verziji našeg projekta.

Ukoliko sad želimo osvježiti indeks sa zadnjom verzijom datoteke (onu koja je, \emph{de facto} spremljena u direktoriju), onda ćemo jednostavno:

\gitoutputcommand{git add $<$datoteka$>$}

Ukratko, indeks je prostor u kojeg spremamo grupu datoteka (\textbf{stanja} datoteka!).
Takav skup datoteka treba predstavljati neku logičku cjelinu koju ćemo spremiti u repozitorij.
To spremanje je jedan \emph{commit}, a tim postupkom smo grafu našeg repozitorija dodali još jedan čvor. 

Prije \emph{commit}a datoteke možemo stavljati u indeks ili izbacivati iz indeksa.
To činimo sve dok nismo sigurni da indeks predstavlja točno one datoteke koje želimo u našoj sljedećoj izmjeni (\emph{commit}u).

Razlika između tog novog čvora i njegovog prethodnika su upravo datoteke koje smo imali u indeksu u trenutku kad smo \emph{commit} izvršili.

\section*{Prvi commit}
\addcontentsline{toc}{section}{Prvi commit}

Izmjene možemo spremiti s:

\gitoutputcommand{git commit -m "Nova verzija"}

U stringu nakon \verb+-m+ \textbf{moramo} unijeti komentar uz svaku promjenu koju spremamo u repozitorij.
Git ne dopušta spremanje izmjena bez komentara\footnote{U stvari, dopušta ako dodate \texttt{--allow-empty-message}, ali bolje da to ne radite.}.

Sad je status projekta opet:

\input{git_output/git_status_0}

\subsection*{Indeks i \emph{commit} grafički}
\addcontentsline{toc}{subsection}{Indeks i \emph{commit} grafički}

Cijela ova priča s indeksom i \emph{commit}anjem bi se grafički mogla prikazati ovako:
U nekom trenutku je stanje projekta ovakvo:

\input{graphs/linearni_bez_zadnjeg_cvora_0}

To znači da je stanje projekta u direktoriju potpuno isti kao i stanje projekta u zadnjem čvoru našeg git grafa.
Nakon toga smo u direktoriju izmijenili nekoliko datoteka:

\input{graphs/linearni_bez_zadnjeg_cvora_1}

To znači, napravili smo izmjene, no one još nisu dio repozitorija.
Zapamtite, samo čvorovi u grafu su ono što git čuva u repozitoriju.
Strelice su samo "postupak" ili "proces" koji je doveo od jednog čvora/\emph{commit}a do drugog.
Zato u prethodnom grafu imamo strelicu koja \textbf{ne vodi do čvora}.

Nakon toga odlučujemo koje ćemo datoteke spremiti u indeks s \verb+git add+.
Kad smo to učinili, s \verb+git commit+ \emph{commit}amo ih u repozitorij, i tek sad je stanje projekta:

\input{graphs/linearni_sa_zadnjim_cvorom}

Dakle, nakon \emph{commit}a smo dobili novi čvor $d$.

\subsection*{Datoteke koje ne želimo u repozitoriju}
\addcontentsline{toc}{subsection}{Datoteke koje ne želimo u repozitoriju}

Situacija koji se često događa je sljedeća:
Greškom smo u repozitorij spremili datoteku koja tamo ne treba biti. 
Međutim, tu datoteku ne želimo obrisati s našeg diska, nego samo ne želimo njenu povijest imati u repozitoriju.

To se dešava, na primjer, kad nam editor ili IDE spremi konfiguracijske datoteke koje su njemu važne, ali nisu bitne za projekt.
Eclipse tako zna snimiti \verb+.project+, a Vim sprema radne datoteke s ekstenzijama \verb+.swp+ ili \verb+.swo+.
Ako smo takvu datoteku jednom dodali u repozitorij, a naknadno smo zaključili da ju više ne želimo, onda ju prvo trebamo dodati u \verb+.gitignore+.
Nakon toga -- git zna da \textbf{ubuduće} neće biti potrebno snimati izmjene na njoj.

No, ona je i dalje u repozitoriju!
Ne želimo ju obrisati s diska, ali ne želimo ju ni u povijesti projekta (od sad pa na dalje).
Neka je to, na primjer, \verb+test.pyc+.
Postupak je:

\gitoutputcommand{git rm --cached test.pyc}

To će nam u indeks dodati stanje kao da je datoteka obrisana, iako ju ostavlja netaknutu na disku.
Drugim riječima \verb+git rm --cached+ sprema "obrisano stanje" datoteke u indeks.
Sad tu izmjenu treba \emph{commit}ati da bi git znao da od ovog trenutka na dalje datoteku može obrisati iz svoje povijesti.

Budući da smo datoteku prethodno dodali u \verb+.gitignore+, git nam ubuduće neće nuditi da ju \emph{commit}amo.
Odnosno, što god radili s tom datotekom, \verb+git status+ će se ponašati kao da ne postoji.

\section*{Povijest projekta}
\addcontentsline{toc}{section}{Povijest projekta}

Sve prethodne \emph{commit}ove možemo pogledati s \verb+git log+:

\input{git_output/git_log_0}

Više riječi o povijesti repozitorija će biti u posebnom poglavlju. 
Za sada je važno znati da u gitu svaki \emph{commit} ima jedinstveni string koji ga identificira.
Taj string ima 40 znamenaka i primjere istih možemo vidjeti u rezultatu od \verb+git log+.
Na primjer, \\\verb+bf4fc495fc926050fb10260a6a9ae66c96aaf908+ je jedan takav.

\section*{Ispravljanje zadnjeg \emph{commit}a}
\addcontentsline{toc}{section}{Ispravljanje zadnjeg \emph{commit}a}

Meni se, prije gita, događalo da \emph{commit}am neku izmjenu u repozitorij, a nakon toga shvatim da sam mogao još jednu sitnicu ispraviti.
I logično bi bilo da je ta "sitnica" dio prethodnog \emph{commit}a, ali \emph{commit} sam već napravio.
Nisam ga mogao naknadno promijeniti.

S gitom se to može.

Prvo učinimo tu izmjenu u radnoj verziji projekta. 
Recimo da je to bilo na datoteci \verb+README.md+.
Dodamo tu datoteku u indeks s \verb+git add README.md+ kao da se spremamo napraviti još jedan \emph{commit}.
Umjesto \verb+git commit+, sad je naredba:

\gitoutputcommand{git commit --amend -m "Nova verzija, promijenjen README.md"}

Ovaj \verb+--amend+ gitu naređuje da izmijeni zadnji \emph{commit} u povijesti tako da sadrži \textbf{i izmjene koje je već imao i izmjene koje smo upravo dodali}.
Možemo provjeriti s \verb+git log+ šte se desilo i vidjeti ćemo da zadnji \emph{commit} sad ima novi komentar.

\verb+git commit --amend+ nam omogućava da u zadnji \emph{commit} dodamo neku datoteku ili čak i \emph{maknemo} datoteku koju smo prethodno \emph{commit}ali. 
Treba samo pripaziti da se taj commit nalazi samo na našem lokalnom repozitoriju, a ne i na nekom od udaljenih. 
Više o tome malo kasnije.

\section*{Git gui}
\addcontentsline{toc}{section}{Git gui}

Kad spremamo \emph{commit} s puno datoteka, onda može postati naporno non-stop tipkati \verb+git add+.
Zbog toga postoji poseban grafički program kojemu je glavna namjena upravo to.
U komandnoj liniji:

\gitoutputcommand{git gui}

Otvoriti će se sljedeće:

\gitgraphics{images/git-gui.png}

Program se sastoji od četiri polja. 

\begin{itemize}
	\item Polje za datoteke koje su izmijenjene, ali nisu još u indeksu (gore lijevo).
	\item Polje za prikaz izmjena u pojedinim datotekama (gore desno). 
	\item Polje za datoteke koje su izmijenjene i stavljene su u indeks (dolje lijevo).
	\item Polje za \emph{commit} (dolje lijevo).
\end{itemize}

Klik na neku datoteku će prikazati sve izmjene koja ta datoteka sadrži u odnosu na zadnje snimljeno stanje u repozitoriju.
Format je isti kao i kod \verb+git diff+.
Klik na ikonu uz naziv datoteke će istu prebaciti iz polja izmijenjenih datotka u polje s indeksom i suprotno.
Nakon što odaberemo datoteke za koje želimo da budu dio našeg \emph{commit}a, trebamo unijeti komentar i kliknuti na "Commit" za snimanje izmjene.

Ovdje, kao i u radu s komandnom linijom ne moramo sve izmijenjene datoteke snimiti u jednom \emph{commit}u. 
Možemo dodati nekoliko datoteka, upisati komentar, snimiti i nakon toga dodati sljedećih nekoliko datoteka, opisati novi komentar i snimiti sljedeću izmjenu.
Drugim riječima, izmjene možemo snimiti u nekoliko posebnih \emph{commit}ova, tako da svaki od njih čini zasebnu logičku cjelinu.

S \verb+git gui+ imamo još jednu korisnu opciju -- možemo u indeks dodati \textbf{ne cijelu datoteku, nego samo nekoliko izmijenjenih linija} datoteke.
Za tu datoteku, u polju s izmijenjenim linijama odaberimo samo linije koje želimo spremite, desni klik i:

\gitgraphics{images/git-gui-stage-lines-to-commit.png}

Ako smo na nekoj datoteci napravili izmjenu koju \emph{ne} želimo snimiti -- takvu datoteku možemo resetirati, odnosno vratiti u početno stanje. 
Jednostavno odaberemo ju u meniju \verb+Commit+ $\rightarrow$ \verb+Revert changes+.

Osim ovoga, \verb+git gui+ ima puno korisnih mogućnosti koje nisu predmet ovog poglavlja.
Preporučam vam da nađete vremena i proučite sve menije i kratice s tastaturom u radu, jer to će vam značajno ubrzati daljnji rad.

%\begin{itemize}
%   \item Tipični scenarij
%   \item Stash?
%   \item Pointeri na commit (hash, HEAD, HEAD~1, HEAD~2, ... master~1, master~2, master~3 )
%   \item Brisanje fajla iz repozitrija (ali ne i lokalnog filesystema)
%   \item Logoff objasniti
%\end{itemize}

%\section*{}
%\addcontentsline{toc}{section}{}

