\chapter*{Spremanje izmjena}
\addcontentsline{toc}{chapter}{Spremanje izmjena}

Vratimo se na trenutak na naša dva primjera, linerani model verzioniranja koda:

\input{graphs/linearni_model}

\dots{}i primjer s granama:

\input{graphs/primjer_s_granama_i_spajanjima}

U oba slučaja, čvorovi tih grafova su stanje projekta u nekom trenutku.
Na primjer, kad ste prvi put inicirali projekt s \emph+git init+, dodali ste nekoliko datoteka i \emph{spremili ih}. 
U tom trenutku je nastao čvor \emph a.
Nakon toga ste možda izmijenili neke od tih datoteka, možda neke obrisali, neke nove dodali i opet -- spremili novo stanje i dobili stanje \emph b.

To što ste radili između svaka dva stanja je vaša stvar i ne tiče se gita.
No, trenutak kad se odlučite spremiti novo stanje projekta u vaš repozitorij -- to je gitu jedino važno i to se zove \emph{commit}.

Važno je ovdje napomenuti da u gitu, za razliku od subversiona, CVS-a ili TFS-a \emph{nikad ne commitate u udaljeni repozitorij}. 
Svoje lokalne promjene \emph{commit}ate, odnosno \emph{spremate} u \emph{lokalni} repozitorij.
Interakcija s udaljenim repozitorijem će biti tema poglavlja o udaljenim repozitorijima\ref{udaljeni_repozitoriji}.

\section*{Status}
\addcontentsline{toc}{section}{Status}

Da biste provjerili imate li uopće nešto za spremati, koristi se naredba \verb+git status+.
Na primjer, kad na projektu na kojem trenutno radim (a to je upravo \emph{ova knjiga}), dobijem ovakav ispis:

\input{git_output/git_status_1}

Najbitniji podatak je linija u kojoj piše \verb+modified: git-commit.tex+, jer to je datoteka koju sam \emph{mijenjao}, ali ne još snimio.

Promijenim li još jednu datoteku, tada će rezultat biti:

\input{git_output/git_status_2}

Dakle, sad su izmijenjene datoteke \verb+git-commit.tex+ i \verb+git.tex+.

Želite li pogledati \emph{koje su točne razlike} u tim datotekama u odnosu na stanje kakvo je snimljeno u repozitoriju, odnosno u \emph{zadnjoj verziji} repozitorija, to ćete dobiti s \verb+git diff+. 
Opet, u slučaju moje knjige, taj ispis izgleda ovako nekako:

\input{git_output/git_diff_1}

Ako vam ovo izgleda zbunjujuće -- postoji i način kako da se to ljepše prikaže, no, čisto za vježbu, nije loše pokušati interpretirati što ovo znači.
Najvažniji dijelovi su linije oni koji počinju si \verb+diff+, jer one govore o kojim datotekama se radi.
Nakon njih slijedi nekoliko linija s općenitim podacima i zatim kod \emph{oko} dijela koji je izmijenjen i onda ono najvažnije:

\begin{verbatim}
-\section*{TODO}
-\addcontentsline{toc}{section}{TODO}
+\section*{Status}
+\addcontentsline{toc}{section}{Status}
+
+Da biste provjerili 
\end{verbatim}

Linije koje započinju s "-" su linije koje su \emph{obrisane}, a one koje počinju s "+" su one koje su \emph{dodane}. 
Primijetite da git ne zna da ste neku liniju izmijenili. 
Ukoliko jeste -- on se ponaša kao da ste staru obrisalu, a novu dodali.

U drugoj datoteci \verb+git.tex+, samo je jedna linije \emph{dodana}, i to prazan redak.

\begin{itemize}
   \item add, remote
   \item git gui
   \item Ammend
   \item Stash?
   \item Pointeri na commit (hash, HEAD, HEAD~1, HEAD~2, ... master~1, master~2, master~3 )
   \item Brisanje fajla iz repozitrija (ali ne i lokalnog filesystema)
\end{itemize}

%\section*{}
%\addcontentsline{toc}{section}{}

