\chapter*{Spremanje izmjena}
\addcontentsline{toc}{chapter}{Spremanje izmjena}

Vratimo se na trenutak na naša dva primjera, linerani model verzioniranja koda:

\input{graphs/linearni_model}

\dots{}i primjer s granama:

\input{graphs/primjer_s_granama_i_spajanjima}

U oba slučaja, čvorovi tih grafova su stanje projekta u nekom trenutku.
Na primjer, kad ste prvi put inicirali projekt s \emph+git init+, dodali ste nekoliko datoteka i \emph{spremili ih}. 
U tom trenutku je nastao čvor \emph a.
Nakon toga ste možda izmijenili neke od tih datoteka, možda neke obrisali, neke nove dodali i opet -- spremili novo stanje i dobili stanje \emph b.

To što ste radili između svaka dva stanja je vaša stvar i ne tiče se gita.
No, trenutak kad se odlučite spremiti novo stanje projekta u vaš repozitorij -- to je gitu jedino važno i to se zove \emph{commit}.

Važno je ovdje napomenuti da u gitu, za razliku od subversiona, CVS-a ili TFS-a \emph{nikad ne commitate u udaljeni repozitorij}. 
Svoje lokalne promjene \emph{commit}ate, odnosno \emph{spremate} u \emph{lokalni} repozitorij.
Interakcija s udaljenim repozitorijem će biti tema poglavlja o udaljenim repozitorijima\ref{udaljeni_repozitoriji}.

\section*{TODO}
\addcontentsline{toc}{section}{TODO}

\begin{itemize}
   \item add, remote
   \item git gui
   \item Ammend
   \item Stash?
   \item Pointeri na commit (hash, HEAD, HEAD~1, HEAD~2, ... master~1, master~2, master~3 )
   \item Brisanje fajla iz repozitrija (ali ne i lokalnog filesystema)
\end{itemize}

%\section*{}
%\addcontentsline{toc}{section}{}

