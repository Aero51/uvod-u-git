\chapter*{Terminologija}
\addcontentsline{toc}{chapter}{Terminologija}

Mi (informatičari, programeri, IT stručnjaci i "stručnjaci", \dots) se redovito služimo stranim pojmovima i nisu nam neobične posuđenice kao \emph{mrđanje}, \emph{brenčanje}, \emph{ekspajranje}, \emph{eksekjutanje}.
Naravno, poželjno bi bilo koristiti alternative koje su više u duhu jezika, ali da ne petjeramo s raznim \emph{vrtoletima}\footnote{Helikopter}, \emph{čegrtasim velepamtilima}\footnote{\emph{Hard disk}}, \emph{nadstolnim klizalima}\footnote{Miš}, \emph{razbubnicima}\footnote{\emph{Debugger}}, \emph{uključnicima}\footnote{\emph{Plugin}} i sl.%
\footnote{Pretpostavljam da će ovo čitati o govornici drugih varijanti ovih naših južnoslavenskih jezika. Pa čisto da znate, kod nas je devedesetih godina vladala opsesija nad time da bi svim stranim stručnim riječima trebali naći prijevode "u duhu jezika". Pa su tako nastale neke od navedenih riječi. Neke od njih su zaista pokušali progurati kao "službene", a druge su samo sprdnja javnosti nad cijelim tim "projektom".}
No, izuzmemo li ove besmislice -- izrazi u duhu jezika za neke termine i ne postoje.
Mogao sam ih ja izmisliti, ali\dots

Meni osobno je besmisleno izmišljati nove riječi za potrebe priručnika koji bi bio uvod u git.
Koristiti termine koje bih sam izmislio i paralelno učiti git bi, za potencijalnog čitatelja, predstavljao dvostruki problem -- em bi morao učiti nešto novo, em bi morao učiti \textbf{moju} terminologiju drukčiju od one kojom se služe stručnjaci.
A stručnjaci su odlučili -- oni govore \emph{fetch}anje (iliti \emph{fečanje}) i \emph{commit}anje (iliti \emph{komitanje}).

Dodatni problem je i to što prijevod često \textbf{nije} ono što se na prvi pogled čini ispravno.
OK, \emph{branch}anje bi bilo "grananje", no \emph{merge}anje \textbf{nije} "spajanje grana". 
Spajanjem grana bi rezultat bio jedna jedina grana, ali \emph{merge}anjem -- obje grane nastavljaju svoj život. 
I kasnije se mogu opet \emph{merge}ati, ali ne bi se mogle još jednom "spajati".
Jedino što se izmjene iz jedne preuzimaju i u drugu. 
Ispravno bi bilo "preuzimanje izmjena iz jedne grane u drugu", ali to zvuči nespretno da bi se koristilo u svakodnevnom govoru.

Činjenica je da većina pojmova jednostavno nemaju ustaljen hrvatski prijevod\footnote{Jedan od glavnih krivaca za to su predavači na fakultetima koji \textbf{ne} misle da je verzioniranje koda tema za fakultetske kolegije.}. 
I zato sam ih koristio na točno onakav način kako se one upotrebljavaju u (domaćem) programerskom svijetu.

Ne znam za vas, ali ja ne vjerujem da će naši jezici nestati zbog stranih izraza\footnote{Ako se ne slažete sa mnom -- slobodni ste napisati svoju knjigu sa \emph{razbubnicima} i \emph{čegrtastim velepamtilima}.}.

Nakon malo eksperimentiranja, rezultat je da sam sve pojmove koristio u izvornom obliku, ali \emph{ukošenim} fontom. Na primjer \emph{fast-forward merge}, \emph{merge}anje, \emph{merge}ati, \emph{fetch}ati, \emph{fetch}anje ili \emph{commit}anje.

Pa, (ne)kome krivo, a (ne)kome pravo\dots

\section*{Popis korištenih termina}
\addcontentsline{toc}{section}{Popis korištenih termina}

Svi termini su objašnjeni u knjizi, ali ako se izgubite u šumi \emph{push}eva, \emph{merge}va i \emph{squash}eva -- evo kratak pregled:

\begin{description}
    \item[\emph{Bare} repozitorij] je repozitorij koji nije predviđen da ima radnu verziju projekta. Njegov smisao je da bude na nekom serveru i da se na njega može \emph{push}ati i s njega \emph{pull}ati i \emph{fetch}ati.
    \item[\emph{Branch}] je grana.
    \item[\emph{Cherry-pick}] je preuzimanje izmjena iz samo jednog \emph{commit}a druge grane.
    \item[\emph{Commit}] je spremanje izmjena na projektu u sustav za verzioniranje.
    \item[Čvor] je \emph{commit}, ali koristi se kad se povijest projekta prikazuje grafom.
    \item[\emph{Diff}] je pregled izmjena između dva \emph{commit}a (ili dvije grane ili dva stanja iste grane).
    \item[\emph{Fast-forward}] je proces koji se događa kad vršimo \emph{merge} dva grafa, pri čemu je zadnji čvor ciljne grana ujedno i točka grananja dva grafa.
    \item[\emph{Fetch}] je preuzimanje izmjena (\emph{commit}ova) s udaljenog repozitorija na lokalni.
    \item[\emph{Log}] je pregled izmjena koje su se desile između \emph{commit}ova u nekoj grani. Ili pregled izmjena između radne verzije i stanja u repozitoriju.
    \item[Indeks] je "međuprostor" u kojeg spremamo izmjene prije nego što ih \emph{commit}amo.
    \item[\emph{Pull}] je kombinacija \emph{fetch}a i \emph{merge}a. S njime se izmjene s udaljenog repozitorija preuzimaju u lokalnu granu.
    \item[\emph{Pull request}] je zahtjev vlasniku udaljenog repozitorija (na kojeg nemamo ovlasti \emph{push}ati) da preuzme izmjene koje smo mi napravili.
    \item[\emph{Push}] je "slanje" lokalnih \emph{commit}ova na udaljeni repozitorij.
    \item[\emph{Radna verzija repozitorija}] je stanje direktorija našeg projekta. Ono može i ne mora biti jednako zadnjem snimljenom stanju u grani repozitorija u kojoj se trenutno nalazimo.
    \item[\emph{Rebase}] je proces kojim točku grananja jednog grafa pomičemo na kraj drugog grafa.
    \item[\emph{Referenca}] je informacija na osnovu koje možemo jedinstveno odrediti neki \emph{commit} ili granu ili \emph{tag}.
    \item[\emph{Reset}] je vraćanje stanja repozitorija na neko stanje. I to \textbf{ne} privremeno vraćanje nego baš izmjenu povijesti repozitorija pri čemu se briše zadnjih nekoliko \emph{commit}ova iz povijesti.
    \item[\emph{Revert}] je spremanje izmjene koja poništava izmjene snimljene u nekom prethodnom \emph{commit}u.
    \item[Repozitorij] je projekt koji je snimljen u nekom sustavu za verzioniranje koda. Repozitorij sadržava cijelu povijest projekta.
    \item[\emph{Staging area}] je sinonim za \textbf{indeks}.
    \item[\emph{Squash merge}] je \emph{merge}, ali na način da novostvoreni čvor nema referencu na granu iz koje su izmjene preuzete.
    \item[\emph{Tag}] je oznaka iliti imenovana referenca na neki \emph{commit}.
%    \item[\emph{}] 
\end{description}

%\section*{}
%\addcontentsline{toc}{section}{}

