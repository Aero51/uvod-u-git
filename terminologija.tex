\chapter*{Terminologija}
\addcontentsline{toc}{chapter}{Terminologija}

Mi (informatičari, programeri, IT stručnjaci i "stručnjaci", \dots) se redovito služimo stranim pojmovima i nisu nam posuđenice kao \emph{mrđanje}, \emph{brenčanje}, \emph{ekspajranje}, \emph{eksekjutanje}.
Bilo bi poželjno koristiti alternative koje su više u duhu jezika\footnote{Naravno, ne treba ni pretjerati s raznim \emph{vrtoletima}, \emph{čegrtasim velepamtilima}, \emph{nadstolnim klizalima}, i sl.}.
Međutim, besmisleno mi je bilo izmišljati nove riječi za potrebe priručnika koji bi bio uvod u git.
Koristiti termine koje sam izmislio u letu i paralelno učiti git bi, za potencijalnog čitatelja, predstavljao dvostruki problem -- em bi morao učiti nešto novo, em bi morao učiti \textbf{moju} terminologiju drukčiju od one kojom se služe stručnjaci.

Činjenica je da većina pojmova jednostavno nemaju ustaljen hrvatski prijevod\footnote{Jedan od glavnih krivaca za to su predavači na fakultetima koji ne misle da je verzioniranje koda tema za fakultetske kolegije. Oni su ti, a ne ljudi "na terenu" kao ja, koji su najviše zaduženi za to da budući stručnjaci koriste izraze u duhu jezika.}. 
I zato sam ih koristio na točno onakav način kako se one upotrebljavaju u (domaćem) programerskom svijetu.

Dodatni problem je i to što prijevod često \textbf{nije} ono što se na prvi pogled čini ispravno.
OK, \emph{branch}anje bi bilo "grananje", no \emph{merge}anje \textbf{nije} "spajanje grana". 
Spajanjem grana bi rezultat bio jedna jedina grana, ali \emph{merge}anjem -- obje grane nastavljaju svoj život. 
I kasnije se mogu opet \emph{merge}ati, ali ne bi se mogle još jednom "spajati".
Jedino što se izmjene iz jedne preuzimaju i u drugu. 
Ispravno bi bilo "preuzimanje izmjena iz jedne grane u drugu", ali to zvuči nespretno da bi se koristilo u standardnom govoru.

Dakle, svi pojmovi sam koristio u izvornom obliku, ali \emph{ukošenim} fontom. Na primjer \emph{fast-forward merge}, \emph{merge}anje ili \emph{commit}anje.

%\section*{}
%\addcontentsline{toc}{section}{}

