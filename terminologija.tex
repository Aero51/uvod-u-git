\chapter*{Terminologija}
\addcontentsline{toc}{chapter}{Terminologija}

Mi (informatičari, programeri, IT stručnjaci, \dots) se redovito služimo stranim pojmovima i nisu nam strane riječi tipa \emph{mrđanje}, \emph{brenčanje}\footnote{Jedan kolega s posla, koji često \emph{merge}-a ima nadimak "mrđer"}, \emph{ekspajranje}, \emph{eksekjutanje}.
No, poželjno bi bilo koristiti alternative koje su više u duhu jezika\footnote{Naravno, ne treba ni pretjerati s raznim \emph{vrtoletima}, \emph{čegrtasim velepamtilima}, \emph{nadstolnim klizalima}, isl.}.

Ovo što slijedi je moj pokušaj prevođenja termina koji se koriste u radu s gitom i sustavima za verzioniranje koda općenito: 

\begin{description}
\item[Version control:] je \emph{verzioniranje koda}.
\item[Version control system:] je \emph{sustav za verzioniranje koda} ili, kraće, verzioniranje koda.
\item[Revision control:] je isto što i \emph{version control}.
\item[Repository:] je \emph{repozitorij}.
\item[Branch:] \emph{grana} sustava za verzioniranje koda.
\item[Merge:] bi nekima prirodno bilo prevesti kao \emph{spajanje grana}, no to nije baš točno. \emph{Merge}anjem dvije grane se one ne spajaju, nego se samo izmjene iz jedne \emph{preuzimaju} u drugu. No svaka grana može nastaviti svoj "život". Zato sam se ja odlučio to prevesti kao \emph{preuzimanje izmjena iz jedne grane u drugu}.
\item[Commit:] je spremanje izmjena u sustav za verzioniranje koda. Kraće \emph{spremanje izmjena}.
\item[Remote repository:] je \emph{udaljeni repozitorij}.
\item[Pull:] je \emph{povlačenje i preuzimanje izmjena} s udaljenog repozitorija.
\item[Pull request:] je \emph{zahtjev za povlačenje i preuzimanje izmjena} s udaljenog repozitorija.
\item[Push:] je \emph{spremanje izmjena na udaljeni repozitorij}.
\item[Fetch:] je \emph{preuzimanje izmjena} s udaljenog repozitorija.
%\item[:] 
\end{description}

Sve ostale pojmove koji nisu spomenuti ovdje sam koristio u izvornom obliku, ali \emph{ukošenim} fontom. Na primjer \emph{fast-forward merge}.

%\section*{}
%\addcontentsline{toc}{section}{}

