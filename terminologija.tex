\chapter*{Terminologija}
\addcontentsline{toc}{chapter}{Terminologija}

Mi (informatičari, programeri, IT stručnjaci i "stručnjaci", \dots) se redovito služimo stranim pojmovima i nisu nam posuđenice kao \emph{mrđanje}, \emph{brenčanje}, \emph{ekspajranje}, \emph{eksekjutanje}.
Naravno, poželjno bi bilo koristiti alternative koje su više u duhu jezika, ali da ne petjeramo s raznim \emph{vrtoletima}\footnote{Helikopter}, \emph{čegrtasim velepamtilima}\footnote{\emph{Hard disk}}, \emph{nadstolnim klizalima}\footnote{Miš}, \emph{razbubnicima}\footnote{\emph{Debugger}}, \emph{uključnicima}\footnote{\emph{Plugin}} i sl.%
\footnote{Pretpostavljam da će ovo čitati o govornici drugih varijanti ovih naših južnoslavenskih jezika. Pa čisto da znate, kod nas je devedesetih godina vladala opsesija nad time da bi svim stranim stručnim riječima trebali naći prijevode "u duhu jezika". Pa su tako nastale neke od navedenih riječi. Neke od njih su zaista pokušali progurati kao "službene", a druge su samo sprdnja javnosti nad cijelim tim "projektom".}
No, izuzmemo li ove besmislice -- izrazi u duhu jezika za neke termine i ne postoje.
Mogao sam ih ja izmisliti, ali\dots

Meni osobno je besmisleno izmišljati nove riječi za potrebe priručnika koji bi bio uvod u git.
Koristiti termine koje bih sam izmislio i paralelno učiti git bi, za potencijalnog čitatelja, predstavljao dvostruki problem -- em bi morao učiti nešto novo, em bi morao učiti \textbf{moju} terminologiju drukčiju od one kojom se služe stručnjaci.
A stručnjaci su odlučili -- oni govore \emph{fetch}anje (iliti \emph{fečanje}) i \emph{commit}anje (iliti \emph{komitanje}).
Ja ne vjerujem da će naši jezici nestati zbog stranih izraza\footnote{Ako se ne slažete sa mnom -- slobodni ste napisati svoju knjigu sa \emph{razbubnicima} i \emph{čegrtastim velepamtilima}.}.

Činjenica je da većina pojmova jednostavno nemaju ustaljen hrvatski prijevod\footnote{Jedan od glavnih krivaca za to su predavači na fakultetima koji \textbf{ne} misle da je verzioniranje koda tema za fakultetske kolegije.}. 
I zato sam ih koristio na točno onakav način kako se one upotrebljavaju u (domaćem) programerskom svijetu.

Dodatni problem je i to što prijevod često \textbf{nije} ono što se na prvi pogled čini ispravno.
OK, \emph{branch}anje bi bilo "grananje", no \emph{merge}anje \textbf{nije} "spajanje grana". 
Spajanjem grana bi rezultat bio jedna jedina grana, ali \emph{merge}anjem -- obje grane nastavljaju svoj život. 
I kasnije se mogu opet \emph{merge}ati, ali ne bi se mogle još jednom "spajati".
Jedino što se izmjene iz jedne preuzimaju i u drugu. 
Ispravno bi bilo "preuzimanje izmjena iz jedne grane u drugu", ali to zvuči nespretno da bi se koristilo u standardnom govoru.

Nakon malo eksperimentiranja, rezultat je da sam sve pojmove koristio u izvornom obliku, ali \emph{ukošenim} fontom. Na primjer \emph{fast-forward merge}, \emph{merge}anje, \emph{merge}ati, \emph{fetch}ati, \emph{fetch}anje ili \emph{commit}anje.

Pa, (ne)kome krivo, a (ne)kome pravo\dots

%\section*{}
%\addcontentsline{toc}{section}{}

