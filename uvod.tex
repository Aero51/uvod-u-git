\chapter*{Uvod}
\addcontentsline{toc}{chapter}{Uvod}

Git nije jednostavan.

\section*{Format ove knjige}
\addcontentsline{toc}{section}{Format ove knjige}

Ova knjiga nije zamišljena kao detaljan uvod u to kako git funkcionira i kao općeniti priručnik u kojem ćete tražiti riješenje svaki put kad negdje zapnete.
Osnovna ideja mi je bila da za svaku "radnju" s gitom opišem problem, ilustriram ga grafikonom, malo razradim teorijom, potkrijepim primjerima i onda opišem 5-10 git naredbi koje se najčešće koriste u kombinaciji. 
Uspijem li u tom naumu -- nakon što pročitate knjigu, trebali biste biti sposobni git koristiti u svakodnevnom radu. 

\section*{Pretpostavke}
\addcontentsline{toc}{section}{Pretpostavke}

Da biste uredno "probavili" ovaj knjižuljak, pretpostavljam da

\begin{itemize}
	\item znate programirati s nekim programskim jezikom,
	\item poznajete princip funkcioniranja klasičnih sustava za verzioniranje koda (CVS, SVN, ...),
	\item ne bojite se komandne linije,
	\item poznajete osnove rada s unix naredbama.
\end{itemize}

Par riječi o zadnje dvije stavke.
Iako git nije nužno ograničen na unix/linux operativne sustave, njegovo komandnolinijsko sučelje je tamo nastalo i drži se tih osnovnih principa.
Git nije nužno komandnolinijski alat, međutim mnoge iole složenije stvari je teško uopće implementirari u nekom grafičkom sučelju. 
Moj prijedlog je da git naučite koristiti u komandnoj liniji, a tek onda krenete s nekim grafičkim alatom. 
Tek tako ćete ga zaista savladati.

% \section*{}
% \addcontentsline{toc}{section}{}
