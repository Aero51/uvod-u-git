\chapter*{Uvod}
\addcontentsline{toc}{chapter}{Uvod}

Git je alat kojeg je razvio Linus Torvalds da bi mu olakšao vođenje jednog velikog i kompleksnog projekta -- linux kernela.
U početku to \emph{nije} bio program kojemu je namjena bila današnja; Linus je zamislio da git bude osnova \emph{drugim sustavima za razvijanje koda}.
Dakle, drugi alati bi trebali razvijati svoje sučelje na osnovu gita.
Međutim, kao s mnogim drugim projektima otvorenog koda, ljudi su ga počeli koristiti takvog kakav jest, on je organski rastao onako kako su ga ljudi koristili.

Rezultat je program koji ima drukčiju terminologiju i, nerijetko, malo neintuitivnu sintaksu. Usprkos tome, milijuni programera diljem svijeta su prihvatili git. 
Nastali su brojne platforme za \emph{hosting} projekata, kao što je github\footnote{http://github.com}, a drugi su morali dodati git jednostavno zato što su to njihovi korisnici tražili (Google Code\footnote{http://code.google.com}, Bitbucket\footnote{http://bitbucket.com}, Sourceforge\footnote{http://sourceforge.net}.

Nekoliko je razloga zašto je to tako:

\begin{itemize}
	\item distribuirani sustavi za razvoj koda omogućuju da na istom projektu radi više ljudi. Postojeći sustavi su obično zahtijevali da na istom kodu radi točno određeni broj programera, a novi bi eventualno mogli predlagati poboljšanja. S distribuiranim sustavima, bilo tko je mogao "forkati" repozitorij, isprogramirati izmjenu i predložiti vlasniku originalnog repozitorija da preuzme svoje izmjene. (TODO: ovo malo razraditi i napisati kao posebno poglavlje, preseliti ovo u drugo poglavlje),
	\item git je izuzetno brz,
	\item vrlo je lako granati, isprobavati izmjene koje su radili drugi ljudi i preuzeti ih u svoj kod,
\end{itemize}

\section*{Format ove knjige}
\addcontentsline{toc}{section}{Format ove knjige}

Ova knjiga nije zamišljena kao detaljan uvod u to kako git funkcionira i kao općeniti priručnik u kojem ćete tražiti riješenje svaki put kad negdje zapnete.
Osnovna ideja mi je bila da se za svaku "radnju" s gitom opiše problem, ilustriram ga grafikonom, malo razradi teorija, potkrijepi primjerima i onda opiše nekoliko osnovnih git naredbi koje se najčešće koriste.
Uspijem li u tom naumu -- nakon što pročitate knjigu, trebali biste biti sposobni git koristiti u svakodnevnom radu. 

\section*{Pretpostavke}
\addcontentsline{toc}{section}{Pretpostavke}

Da biste uredno "probavili" ovaj knjižuljak, pretpostavljam da:

\begin{itemize}
	\item znate programirati s nekim programskim jezikom,
	\item poznajete princip funkcioniranja klasičnih sustava za verzioniranje koda (CVS, SVN, ...),
	\item ne bojite se komandne linije,
	\item poznajete osnove rada s unix naredbama.
\end{itemize}

Nekoliko riječi o zadnje dvije stavke.
Iako git nije nužno ograničen na unix/linux operativne sustave, njegovo komandnolinijsko sučelje je tamo nastalo i drži se istih principa.
Git nije nužno komandnolinijski alat, međutim mnoge iole složenije stvari je teško uopće implementirari u nekom grafičkom sučelju. 
Moj prijedlog je da git naučite koristiti u komandnoj liniji, a tek onda krenete s nekim grafičkim alatom -- tek tako ćete ga zaista savladati.

% \section*{}
% \addcontentsline{toc}{section}{}
