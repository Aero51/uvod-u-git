\chapter*{Uvod}
\addcontentsline{toc}{chapter}{Uvod}

Git je alat kojeg je razvio Linus Torvalds da bi mu olakšao vođenje jednog velikog i kompleksnog projekta -- linux kernela.
U početku to \emph{nije} bio program s današnjom namjenom; Linus je zamislio da git bude osnova \emph{drugim sustavima za razvijanje koda}.
Dakle, drugi alati bi trebali razvijati svoje sučelje na osnovu gita.
Međutim, kao s mnogim drugim projektima otvorenog koda, ljudi su ga počeli koristiti takvog kakav jest, on je organski rastao onako kako su ga ljudi koristili.

Rezultat je program koji ima drukčiju terminologiju i, nerijetko, malo neintuitivnu sintaksu. Usprkos tome, milijuni programera diljem svijeta su prihvatili git. 
Nastali su brojne platforme za \emph{hosting} projekata, kao što je Github\footnote{http://github.com}, a drugi su morali dodati git jednostavno zato što su to njihovi korisnici tražili (Google Code\footnote{http://code.google.com}, Bitbucket\footnote{http://bitbucket.com}, Sourceforge\footnote{http://sourceforge.net}).

Nekoliko je razloga zašto je to tako:

\begin{itemize}
	\item Postojeći sustavi za verzioniranje su obično zahtijevali da se točno zna tko sudjeluje u projektu (tj. tko je \emph{comitter}). To je automatski demoraliziralo puno ljudi koji bi možda pokušali pomoći projektima kad mi imali priliku. S distribuiranim sustavima, bilo tko je mogao "forkati" repozitorij, isprogramirati izmjenu i predložiti vlasniku originalnog repozitorija da preuzme svoje izmjene. Rezultat je da je broj ljudi koji su potencijalni mogli pomoći projektu puno veći. A vlasnik projekta je ionako imao kontrolu nad time čije će izmjene prihvatiti, a čije neće.
	\item git je \textbf{brz},
	\item vrlo je lako i brzo granati, isprobavati izmjene koje su radili drugi ljudi i preuzeti ih u svoj kod,
	\item Linux kernel se razvijao na gitu, tako da je u svijetu otvorenog koga (\emph{open source}) git stekao nekakvu auru važnosti.
\end{itemize}

\section*{Format ove knjige}
\addcontentsline{toc}{section}{Format ove knjige}

Ova knjiga nije zamišljena kao detaljan uvod u to kako git funkcionira i kao općeniti priručnik u kojem ćete tražiti riješenje svaki put kad negdje zapnete.
Osnovna ideja mi je bila da se za svaku "radnju" s gitom opišem problem, ilustriram ga grafikonom, malo razradim teoriju, potkrijepim primjerima i onda opišem nekoliko osnovnih git naredbi koje se najčešće koriste.
Uspijem li u tom naumu -- nakon što pročitate knjigu, trebali biste biti sposobni git koristiti u svakodnevnom radu. 

Zapnete li, a odgovora ne nađete ovdje, pravac Stackoverflow\footnote{http://stackoverflow.com}, Google, 
forumi, blogovi i, naravno, \verb+git help+.

\section*{Pretpostavke}
\addcontentsline{toc}{section}{Pretpostavke}

Da biste uredno "probavili" ovaj knjižuljak, pretpostavljam da:

\begin{itemize}
	\item znate programirati u nekim programskom jeziku,
	\item poznajete princip funkcioniranja klasičnih sustava za verzioniranje koda (CVS, SVN, ...),
	\item ne bojite se komandne linije,
	\item poznajete osnove rada s unixoidnim operativnim sustavima.
\end{itemize}

Nekoliko riječi o zadnje dvije stavke.
Iako git nije nužno ograničen na unix/linux operativne sustave, njegovo komandnolinijsko sučelje je tamo nastalo i drži se istih principa.
Git nije nužno komandnolinijski alat, međutim mnoge iole složenije stvari je teško uopće implementirati u nekom grafičkom sučelju. 
Moj prijedlog je da git naučite koristiti u komandnoj liniji, a tek onda krenete s nekim grafičkim alatom -- tek tako ćete ga zaista savladati.

\section*{Našla/našao sam grešku}
\addcontentsline{toc}{section}{Našla/našao sam grešku}

Svjestan sam toga da ova knjižica vrvi greškama.
Ja nisam profesionalan pisac, a ova knjiga nije prošla kroz ruke profesionalnog lektora.

Grešaka ima i pomalo ih ispravljam.
No, ako želite pomoći -- zahvalan sam vam.
Evo nekoliko načina kako to možete učniti:

\begin{itemize}
    \item Pošaljite email na \verb+tkrajina@gmail.com+,
    \item \emph{Twitt}nite na \verb+@puzz+,
    \item \emph{Fork}ajte i pošaljite \emph{pull request} s ispravkom. Repozitorij s tekstom je na: \\ \verb+https://github.com/tkrajina/uvod-u-git+
\end{itemize}

Ukoliko odaberete bilo koju varijantu osim zadnje (\emph{fork}) -- dovoljan je kratak opis s greškom (stranica, rečenica, redak) i šifra koja se nalazi na dnu naslovnice\footnote{Na primjer ono što piše \emph{Commit} b5af8ec79a7384a5a291d15d050fc932eb474e79. E, ovaj nerazumljivi dugi string meni olakšava traženje verzije za koju prijavljujete grešku.}.

Ukoliko i uočite grešku, možda je ona već ispravljena u zadnjoj verziji. 
Na \\ \verb+https://github.com/tkrajina/uvod-u-git+ uvijek možete naći zadnju verziju ove knjižice.

\section*{Naredbe i operativni sustavi}
\addcontentsline{toc}{section}{Naredbe i operativni sustavi}

Sve naredbe koje nisu specifične za git, kao na primjer "stvaranje novog direktorija", "ispis datoteka u direktoriju", i sl. će biti prema POSIX standardu\footnote{http://en.wikipedia.org/wiki/POSIX}.
Dakle, u primjerima ćemo koristiti naredbe koje se koriste na UNIX, OS X i linux operativnim sustavima. 
No, i za korisnike Microsoft Windowsa to ne bi trebao biti problem jer se radi o relativno malo broju naredbi kao što su \verb+mkdir+ umjesto \verb+md+, \verb+ls+ umjesto \verb+dir+, i slično.
