\chapter*{Manje korištene naredbe}
\addcontentsline{toc}{chapter}{Manje korištene naredbe}

U ovom poglavlju ćemo proći neke rijeđe korištene naredbe gita.
Neke od njih ćete koristiti jako rijetko, a neke vjerojatno nikad.
Zato nije ni potrebno da ih detaljno razumijete, važno je samo da znate da one postoje. 
Ovdje ćemo ih samo nabrojati i generalno opisati čemu služe, a ako nekad zatrebaju -- lako ćete saznati kako se koriste s \verb+git help+.

\section*{Clean}
\addcontentsline{toc}{section}{Clean}

S naredbom\do\TODO

\gitoutputcommand{git clean}

\dots{}git iz radnog direktorija briše sve one datoteke koje nisu u povijesti repozitorija.

To je korisno kad želite obrisati privremene datoteke koje su rezultat kompajliranja ili privremen direktorije.
Ukoliko dodate \verb+-x+ -- naredba će obrisati i sve datoteke koje su popisane u \verb+.gitignore+.

\section*{Bisect}
\addcontentsline{toc}{section}{Bisect}

\emph{Bisect} je pomoćna git naredba koja se koristi kad imamo bug u programu, a ne znamo točno trenutak u povijesti repozitorija kad je nastao.
Imamo li jednostavan način kako reproducirati bug -- možemo se s \verb+git checkout+ igrati sve dok ne "pogodimo" trenutak u povijesti kad je bug nastao. 
Kad ga nađemo -- treba samo proučiti kod koji je uveden ili izmijenjen s tim \emph{commit}om.

Od nas se očekuje samo da imamo referencu na \emph{commit} u kojemu je bug prisutan (\verb+bad+) i na \emph{commit} u kojemu bug \textbf{nije} prisutan (\verb+good+).
\emph{Bisect} nam tada u nekoliko koraka pomaže naći trenutak kada je bug nastao\footnote{Metodom binarnog pretraživanja.}.

\section*{Filter-branch}
\addcontentsline{toc}{section}{Filter-branch}

Rijetko korištena naredba s kojom možemo promijeniti cijelu povijest projekta.
Na primjer, \emph{commit}ali smo u projekt s našom privatnom email adresom, i sad bismo htjeli promijeniti sve naše \emph{commit}ove tako da sadrže službeni email.
Slično, možemo mijenjati datume \emph{commit}ova, dodati datoteke ili obrisati datoteke iz \emph{commit}ova, isl.

Trebamo imati na umu da tako promijenjeni repozitorij ima različite SHA1 stringove \emph{commit}ova.
To znači da, ako naredbu primijenimo na jednom repozitoriju, drugi distribuirani repozitorij istog projekta \textbf{više neće imati zajedničku povijest s našim}.

Najbolje je to učiniti na našem privatnom repozitoriju kad smo sigurni da nitko drugi nema klon repozitorija ili ako na projektu radimo s točno određenim krugom ljudi.
U ovom drugom slučaju -- dogovorimo se s njima da svi iz\emph{commit}aju svoje grane u naš repozitorij, izvršiti ćemo \verb+git filter-branch+ i nakon toga zamolimo sve ostale da sad iznova kloniraju repozitorij.

\section*{Shortlog}
\addcontentsline{toc}{section}{Shortlog}

\verb+git shortlog+ ispisuje rekapitulaciju \emph{commit}ova prema autoru.

\section*{Format-patch}
\addcontentsline{toc}{section}{Format-patch}

Koristi se kad šaljemo \emph{patch} emailom\footnote{Čini mi se da je to danas jako rijedak običaj.}.
Na primjer, napravili smo lokalno nekoliko \emph{commit}ova i sad ih želimo poslati emailom vlasniku udaljenog projekta.
S \verb+git format-patch+ ćemo pripremiti emailove sa svim potrebnim detaljima o \emph{commit}ovima (odnosno naše \emph{patch}eve).

\section*{Am}
\addcontentsline{toc}{section}{Am}

Radnja suprotna onome što radimo s \verb+git format-patch+.
U ovom slučaju smo \textbf{mi} oni koji smo primili \emph{patch}eve emailom, i sad ih treba "pretvoriti" u \emph{commit}ove.
To se radi s \verb+git am+.

\section*{Fsck}
\addcontentsline{toc}{section}{Fsck}

\TODO

\section*{Instaweb}
\addcontentsline{toc}{section}{Instaweb}

\verb+git instaweb+ omogućuje jednostavno web sučelje za pregled povijesti repozitorija.

\section*{Name-rev}
\addcontentsline{toc}{section}{Name-rev}

Ja znam da je \verb+e0d22c0608ca0867b501f4890b4155486e8896b8+ \emph{commit} u mom repozitoriju.
Gitu je to dovoljno, ali svima nama bi puno više značilo da nam netko kaže "peti \emph{commit} prije verzije \verb+1.0+" ili "drugi \emph{commit} nakon što smo \emph{branch}ali granu \verb+test+".

Na primjer, meni \verb+git name-rev e0d22+ ispisuje \verb+manje-koristene-naredbe~6+, što znači da je to šesti \emph{commit} prije kraja grane \verb+manje-koristene-naredbe+.

\section*{Stash}
\addcontentsline{toc}{section}{Stash}

\TODO

\section*{Submodule}
\addcontentsline{toc}{section}{Submodule}

\TODO

\section*{Rev-list}
\addcontentsline{toc}{section}{Rev-list}

Rev list za zadane \emph{commit} objekte daje spisak svih \emph{commit}ova koji su dostupni.
Ovu naredbu ćete vjerojatno koristiti tek ako radite neki skriptu (ili git \emph{plugin}), a vrlo rijetko direktno u komandnoj liniji.

%\section*{}
%\addcontentsline{toc}{section}{}
