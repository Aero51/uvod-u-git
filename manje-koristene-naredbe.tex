\chapter*{Manje korištene naredbe}
\addcontentsline{toc}{chapter}{Manje korištene naredbe}

U ovom poglavlju ćemo proći neke rijeđe korištene naredbe gita.
Neke od njih ćete koristiti jako rijetko, a neke vjerojatno nikad.
Zato nije ni potrebno da ih detaljno razumijete, važno je samo da znate da one postoje. 
Ovdje ćemo ih samo nabrojati i generalno opisati čemu služe, a ako nekad zatrebaju -- lako ćete saznati kako se koriste s \verb+git help+.

\section*{Clean}
\addcontentsline{toc}{section}{Clean}

S naredbom\do\TODO

\gitoutputcommand{git clean}

\dots{}git iz radnog direktorija briše sve one datoteke koje nisu u povijesti repozitorija.

To je korisno kad želite obrisati privremene datoteke koje su rezultat kompajliranja ili privremen direktorije.
Ukoliko dodate \verb+-x+ -- naredba će obrisati i sve datoteke koje su popisane u \verb+.gitignore+.

\section*{Bisect}
\addcontentsline{toc}{section}{Bisect}

\emph{Bisect} je pomoćna git naredba koja se koristi kad imamo bug u programu, a ne znamo točno trenutak u povijesti repozitorija kad je nastao.
Imamo li jednostavan način kako reproducirati bug -- možemo se s \verb+git checkout+ igrati sve dok ne "pogodimo" trenutak u povijesti kad je bug nastao. 
Kad ga nađemo -- treba samo proučiti kod koji je uveden ili izmijenjen s tim \emph{commit}om.

Od nas se očekuje samo da imamo referencu na \emph{commit} u kojemu je bug prisutan (\verb+bad+) i na \emph{commit} u kojemu bug \textbf{nije} prisutan (\verb+good+).
\emph{Bisect} nam tada u nekoliko koraka pomaže naći trenutak kada je bug nastao\footnote{Metodom binarnog pretraživanja.}.

\section*{Rev-list}
\addcontentsline{toc}{section}{Rev-list}

\TODO

\section*{Filter-branch}
\addcontentsline{toc}{section}{Filter-branch}

Rijetko korištena naredba s kojom možete promijeniti cijelu povijest projekta.
Na primjer, \emph{commit}ali ste u projekt s vašom privatnom email adresom, i sad biste htjeli promijeniti sve vaše \emph{commit}ove tako da sadrže službeni email.
Slično, možete mijenjati datume \emph{commit}ova, dodati datoteke ili obrisati datoteke iz \emph{commit}ova, isl.

Trebate imati na umu da tako promijenjeni repozitorij ima različite SHA1 stringove \emph{commit}ova.
To znači da, ako naredbu primijenite na jednom repozitoriju, drugi repozitorij više neće imati zajedničku povijest s vašim.

Najbolje je to učiniti na vašem privatnom repozitoriju (kad ste sigurni da nitko drugi nema klon repozitorija) ili ako na projektu radite s točno određenim krugom ljudi.
U ovom drugom slučaju -- dogovorite se s njima da svi \emph{commit}aju svoje grane u vaš repozitorij, izvršite \verb+git filter-branch+ i nakon toga zamolite sve ostale da sad iznova kloniraju repozitorij.

\section*{Shortlog}
\addcontentsline{toc}{section}{Shortlog}

\TODO

\section*{Format-patch}
\addcontentsline{toc}{section}{Format-patch}

\TODO

\section*{Am}
\addcontentsline{toc}{section}{Am}

\TODO

\section*{Fsck}
\addcontentsline{toc}{section}{Fsck}

\TODO

\section*{Instaweb}
\addcontentsline{toc}{section}{Instaweb}

\TODO

\section*{Name-rev}
\addcontentsline{toc}{section}{Name-rev}

\TODO

\section*{Stash}
\addcontentsline{toc}{section}{Stash}

\TODO

\section*{Submodule}
\addcontentsline{toc}{section}{Submodule}

\TODO

%\section*{}
%\addcontentsline{toc}{section}{}
