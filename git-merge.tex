\chapter*{Preuzimanje izmjena iz jedne grane u drugu}
\addcontentsline{toc}{chapter}{Preuzimanje izmjena iz jedne grane u drugu}

Vratimo se opet na jednu viđenu ilustraciju:

\input{graphs/git_merge_1}

Da ponovoimo, prebacivanjem na granu \verb+master+, izmjene koje ste napravili u \emph{commit}ovima \emph d, \emph e, \emph f i \emph g vam neće biti dostupne.
Istko, kad se prebacite na \verb+eksperimentalna-grana+ -- neće vam biti dostupne izmjene iz \emph x, \emph y i \emph z.

To je sve divno i krasno dok svoje izmjene želite raditi u relativnoj izolaciji od ostatka izmjena u kodu. 
No, što ako ste u \verb+master+ riješili neki gadan bug i htjeli biste sad tu ispravku preuzeti u \verb+eksperimentalna-grana+.

Ili, što ako zaključite kako je rezultat eksperimenta kojeg ste isprobali u \verb+eksperimentalna-grana+ uspješan i želite to sad imati u \verb+master+?
Ono što vam sad treba je da nekako \emph{izmjene iz jedne grane preuzmete u drugu granu}.
U gitu, to se naziva \textbf{merge}.
Iako bi merge mogli doslovno prevesti kao "spajanje", to nije ispravna riječ. 
Rezultat spajanja bi bila samo jedna grana. 
No, \emph{merge}anjem dvije grane -- one nastavljaju svoj život. 
Jedino što se sve izmjene koje su do tog trenutka rađene u jednoj granu preuzimaju u drugoj grani.

\section*{Git merge}
\addcontentsline{toc}{section}{Git merge}

Pretpostavimo, na primjer, da sve izmjene iz \verb+eksperimentalna-grana+ želite u \verb+master+. 
To se radi s naredbom \verb+git merge+.
Zapamtite, da bi to napravili kako treba, trebate se nalaziti u onoj grani u koju želite preuzeti izmjene (u našem slučaju \verb+master+), i onda:

\input{git_output/git_merge_1}

Ovaj ispis je ukratko opisan rezultat procesa preuzimanja izmjena; koliko je linija dodano, koliko obrisano, koliko je datoteka dodano, koliko oduzeto, i tako dalje\dots

Grafički se \verb+git merge+ može prikazati ovako:

\input{graphs/git_merge_2}

Kad ste preuzeli izmjene iz jednog grafa u drugi, nitko vas ne prisiljava da onaj prvi obrišete. 
Grana može uredno nastaviti svoj život; dalje možete uredno u nju \emph{commit}ati, preuzimati izmjene iz jedne grane u drugu.
I, eventualno, jednog dana kad odlučite da vam više ne treba, možete ju obrisati.

\input{graphs/git_merge_3}

\input{graphs/primjer_s_imenovanim_granama_i_spajanjima}

\section*{Što merge radi kad\dots}
\addcontentsline{toc}{section}{Što merge radi kad\dots}

Možda vas zanima što merge radi u raznim varijantama.
Recimo, u eksperimentalnoj grani ste dodali novu datoteku, a u glavnoj ste na toj datoteci napravili neku izmjenu.

U principu, vjerujte mi, git napravi točno ono što treba. 
Kad sam ga ja počeo koristiti imao sam neku malu tremu i nesigurnosti s tim \verb+git merge+.
Svaki put sam išao pogledati je li napravio ono što treba i provjeravati da mi nije možda pregazio neku datoteku ili važan komad koda.
I nije, uvijek radi točno onako kako bi intuicija govorila da \emph{merge} treba izgledati.

Umjesto beskonačnih primjera što radi, najbolje je da to jednostavno isprobate, a ja ću ovdje samo ukratko popisati što radi i zadržati se na jednoj varijanti. Dakle, 
Uzmimo poznati slučaj:

\input{graphs/git_merge_2}

Dakle, što će biti rezultat \emph{merge}anja, ako ste\dots

\begin{itemize}
	\item \dots{}u eksperimentalnoj grani izmijenili datoteku, a u \verb+master+ niste -- izmjene iz eksperimentalne će se dodati u \verb+master+.
	\item \dots{}u eksperimentalnoj grani dodali datoteku -- ta datoteka će biti dodana i u \verb+master+.
	\item \dots{}u eksperimentalnoj grani izbrisali datoteku -- datoteka će biti obrisana u glavnoj.
	\item \dots{}u eksperimentalnoj grani \textbf{izmijenili i preimenovali} datoteku, a u \verb+master+ ste samo izmijenili datoteku -- ako izmjene na kodu nisu bile \textbf{konfliktne}, onda će se u \verb+master+ datoteka preimenovati i sadržavati će izmjene iz obje grane.
	\item \dots{}u eksperimentaloj grani obrisali datoteku, a u glavnoj ju izmijenili -- \textbf{konflikt}.
	\item itd\dots
\end{itemize}

Vjerojatno slutite što znači ova riječ koja je ispisana masnim slovima: \textbf{konflikt}.
Postoje slučajevi u kojima git ne zna što napraviti. 
I tada se očekuje od korisnika da on riješi problem. 
Pokušati ću to ilustrirati tako da opišem jedan od gornjih slučajeva; što se desi kad\dots

\section*{Što se desi kad\dots}
\addcontentsline{toc}{section}{Što se desi kad\dots}

Vjerojatno to i sami slutite -- stvar nije uvijek tako jednostavna.
Dogoditi će se da u jednoj grani napravite izmjenu u jednoj datoteci, a u drugoj grani napravite izmjenu na \emph{istoj} datoteci.
I što onda?

Pokušat ću to ilustrirati na jednom jednostavnom primjeru\dots
Uzmimo jedan malo vjerojatan scenarij; neka je Antun Branko Šimić još živ i piše pjesme, naravno.
Napiše pjesmu, pa s njome nije baš zadovoljan, pa malo križa po papiru, pa izmijeni prvi stih, pa izmijeni zadnji stih.
Ponekad mu se rezultat sviđa, ponekad ne.
Ponekad krene iznova.
Ponekad ima ideju, napiše nešto nabrzinu, i onda kasnije napravi dvije verzije iste pjesme.
Ponekad\dots

Scenarij kao stvoren za git, nije li?

Recimo da je autor krenuo sa sljedećom verzijom pjesme:

	\gitoutput{%
	PJESNICI U SVIJETU\\%
	\\%
	Pjesnici su čuđenje u svijetu\\%
	\\%
	Oni idu zemljom i njihove oči\\%
	velike i nijeme rastu pored stvari\\%
	\\%
	Naslonivši uho\\%
	na tišinu sto ih okružuje i muči\\%
	oni su vječno treptanje u svijetu}

I sad ovdje nije baš bio zadovoljan sa cjelinom i htio je isprobati dvije varijante.
Budući da ga je netko naučio git, iz početnog stanja (\emph a) napravio je dvije verzije.

\input{graphs/ab_simic_1}

U prvoj varijanti (\emph b), izmijenio je naslov, tako da je sad pjesma glasila:

	\gitoutput{%
	\textcolor{blue}{PJESNICI}\\%
	\\%
	Pjesnici su čuđenje u svijetu\\%
	\\%
	Oni idu zemljom i njihove oči\\%
	velike i nijeme rastu pored stvari\\%
	\\%
	Naslonivši uho\\%
	na tišinu sto ih okružuje i muči\\%
	oni su vječno treptanje u svijetu}

\dots{}dok je u drugoj varijanti (\emph c) izmijenio zadnji stih:

	\gitoutput{%
	PJESNICI U SVIJETU\\%
	\\%
	Pjesnici su čuđenje u svijetu\\%
	\\%
	Oni idu zemljom i njihove oči\\%
	velike i nijeme rastu pored stvari\\%
	\\%
	Naslonivši uho\\%
	\textcolor{blue}{na ćutanje sto ih okružuje i muči\\%
	pjesnici su vječno treptanje u svijetu}}

\dots{}i bio je zadovoljan i s jednim i s drugim riješenjem.
Odlučio je izmjene iz varijante \verb+varijanta+ preuzeti u \verb+master+.
Nakon \verb+git checkout master+ i \verb+git merge varijanta+, rezultat je bio:

\input{graphs/ab_simic_2}

\dots{}odnosno, pjesnikovim riječima:

	\gitoutput{%
	\textcolor{blue}{PJESNICI}\\%
	\\%
	Pjesnici su čuđenje u svijetu\\%
	\\%
	Oni idu zemljom i njihove oči\\%
	velike i nijeme rastu pored stvari\\%
	\\%
	Naslonivši uho\\%
	\textcolor{blue}{na ćutanje sto ih okružuje i muči\\%
	pjesnici su vječno treptanje u svijetu}}

I to je jednostavno.
U obje grane je mijenjao istu datoteku, ali u jednoj je dirao početak, a u drugoj kraj.
I rezultat \emph{merge}anja je bio očekivan -- datoteka u kojoj je izmijenjen i početak i kraj.

\section*{Konflikti}
\addcontentsline{toc}{section}{Konflikti}

No, što da je u obje grane dirao isti dio pjesme?
Što da je stanje nakon:

\input{graphs/ab_simic_1}

\dots{} stanje bilo ovakvo:
U verziji \emph a je pjesma sad glasila:

	\gitoutput{%
	PJESNICI\\%
	\\%
	Pjesnici su čuđenje u svijetu\\%
	\\%
	\textcolor{blue}{Pjesnici idu zemljom i njihove oči\\%
	velike i nijeme rastu pored ljudi}\\%
	\\%
	Naslonivši uho\\%
	na ćutanje sto ih okružuje i muči\\%
	pjesnici su vječno treptanje u svijetu}

\dots{}a u verziji \emph b ovako:

	\gitoutput{%
	PJESNICI\\%
	\\%
	Pjesnici su čuđenje u svijetu\\%
	\\%
	\textcolor{blue}{Oni idu zemljom i njihova srca\\%
	velika i nijema rastu pored stvari}\\%
	\\%
	Naslonivši uho\\%
	na ćutanje sto ih okružuje i muči\\%
	pjesnici su vječno treptanje u svijetu}

Sad je rezultat naredbe \verb+git merge eksperimentalna-grana+ ovakav:

\input{git_output/git_merge_konflikt}

To znači da git nije znao kako da \emph{automatski} preuzme izmjene iz \emph b u \emph a.
I sad se od autora očekuje da ispravi gitovu nedoumicu.
Idete li sad editirati datoteku s pjesmom naći ćete ovakvo nešto:

	\gitoutput{%
	PJESNICI U SVIJETU\\%
	\\%
	Pjesnici su čudenje u svijetu\\%
	\\%
	\textcolor{red}{<<<<<<< HEAD\\%
	Oni idu zemljom i njihova srca\\%
	velika i nijema rastu pored stvari\\%
	=======\\%
	Pjesnici idu zemljom i njihove oči\\%
	velike i nijeme rastu pored ljudi\\%
	>>>>>>> eksperimentalna-grana}\\%
	\\%
	Naslonivši uho na tišinu sto ih okružuje i muči\\%
	oni su vječno treptanje u svijetu}

\section*{Merge, branch i povijest projekta}
\addcontentsline{toc}{section}{Merge, branch i povijest projekta}

\section*{Fast forward}
\addcontentsline{toc}{section}{Fast forward}

\section*{Rebase}
\addcontentsline{toc}{section}{Rebase}

% \begin{itemize}
%    \item mergeanje znači mergeanje svih izmjena u povijesti, sve od mjesta gdje su se dvije grane bifucirale
%    \item izuzetak je git cherry-pick (grafovi s obojanim čvorovima)
%    \item --no-ff --no-commit
%    \item rebase
%    \item cherry-pick
%    \item preuzimanje samo jednog fajla iz drugog brancha
%    \item kreiranje privremenog brancha za eksperimentalni merge
% \end{itemize}

%\section*{}
%\addcontentsline{toc}{section}{}

