\chapter*{Preuzimanje izmjena iz jedne grane u drugu}
\addcontentsline{toc}{chapter}{Preuzimanje izmjena iz jedne grane u drugu}

Vratimo se opet na jednu viđenu ilustraciju:

\input{graphs/git_merge_1}

Da ponovoimo, prebacivanjem na granu \verb+master+, izmjene koje ste napravili u \emph{commit}ovima \emph d, \emph e, \emph f i \emph g vam neće biti dostupne.
Istko, kad se prebacite na \verb+eksperimentalna-grana+ -- neće vam biti dostupne izmjene iz \emph x, \emph y i \emph z.

To je sve divno i krasno dok svoje izmjene želite raditi u relativnoj izolaciji od ostatka izmjena u kodu. 
No, što ako ste u \verb+master+ riješili neki gadan bug i htjeli biste sad tu ispravku preuzeti u \verb+eksperimentalna-grana+.

Ili, što ako zaključite kako je rezultat eksperimenta kojeg ste isprobali u \verb+eksperimentalna-grana+ uspješan i želite to sad imati u \verb+master+?
Ono što vam sad treba je da nekako \emph{izmjene iz jedne grane preuzmete u drugu granu}.
U gitu, to se naziva \textbf{merge}.
Iako bi merge mogli doslovno prevesti kao "spajanje", to nije ispravna riječ. 
Rezultat spajanja bi bila samo jedna grana. 
No, \emph{merge}anjem dvije grane -- one nastavljaju svoj život. 
Jedino što se sve izmjene koje su do tog trenutka rađene u jednoj granu preuzimaju u drugoj grani.

\section*{Git merge}
\addcontentsline{toc}{section}{Git merge}

Pretpostavimo, na primjer, da sve izmjene iz \verb+eksperimentalna-grana+ želite u \verb+master+. 
To se radi s naredbom \verb+git merge+.
Zapamtite, da bi to napravili kako treba, trebate se nalaziti u onoj grani u koju želite preuzeti izmjene (u našem slučaju \verb+master+), i onda:

\input{git_output/git_merge_1}

Ovaj ispis je ukratko opisan rezultat procesa preuzimanja izmjena; koliko je linija dodano, koliko obrisano, koliko je datoteka dodano, koliko oduzeto, i tako dalje\dots

Grafički se \verb+git merge+ može prikazati ovako:

\input{graphs/git_merge_2}

Kad ste preuzeli izmjene iz jednog grafa u drugi, nitko vas ne prisiljava da onaj prvi obrišete. 
Grana može uredno nastaviti svoj život; dalje možete uredno u nju \emph{commit}ati, preuzimati izmjene iz jedne grane u drugu.
I, eventualno, jednog dana kad odlučite da vam više ne treba, možete ju obrisati.

\input{graphs/git_merge_3}

\input{graphs/primjer_s_imenovanim_granama_i_spajanjima}

\section*{Merge primjer}
\addcontentsline{toc}{section}{Merge primjer}

Vjerojatno to i sami slutite -- stvar nije uvijek tako jednostavna.
Dogoditi će se da u jednoj grani napravite izmjenu u jednoj datoteci, a u drugoj grani napravite izmjenu na \emph{istoj} datoteci.
I što onda?

Pokušat ću to ilustrirati na jednom jednostavnom primjeru\dots
Uzmimo jedan malo vjerojatan scenarij; neka je Antun Branko Šimić još živ i piše pjesme, naravno.
Napiše pjesmu, pa s njome nije baš zadovoljan, pa malo križa po papiru, pa izmijeni prvi stih, pa izmijeni zadnji stih.
Ponekad mu se rezultat sviđa, ponekad ne.
Ponekad krene iznova.
Ponekad ima ideju, napiše nešto nabrzinu, i onda kasnije napravi dvije verzije iste pjesme.
Ponekad\dots

Scenarij kao stvoren za git, nije li?

Recimo da je autor krenuo sa sljedećom verzijom pjesme:

\vspace{5mm}
\noindent\texttt{%
PJESNICI U SVIJETU\\%
\\%
Pjesnici su čuđenje u svijetu\\%
\\%
Oni idu zemljom i njihove oči\\%
velike i nijeme rastu pored stvari\\%
\\%
Naslonivši uho\\%
na tišinu sto ih okružuje i muči\\%
oni su vječno treptanje u svijetu}
\vspace{5mm}

I sad ovdje nije baš bio zadovoljan sa cjelinom i htio je isprobati dvije varijante.
Budući da ga je netko naučio git, iz početnog stanja (\emph a) napravio je dvije verzije.

\input{graphs/ab_simic_1}

U prvoj varijanti (\emph b), izmijenio je naslov, tako da je sad pjesma glasila:

\vspace{5mm}
\noindent\texttt{%
\textcolor{blue}{PJESNICI}\\%
\\%
Pjesnici su čuđenje u svijetu\\%
\\%
Oni idu zemljom i njihove oči\\%
velike i nijeme rastu pored stvari\\%
\\%
Naslonivši uho\\%
na tišinu sto ih okružuje i muči\\%
oni su vječno treptanje u svijetu}
\vspace{5mm}

\dots{}dok je u drugoj varijanti (\emph c) izmijenio zadnji stih:

\vspace{5mm}
\noindent\texttt{%
PJESNICI U SVIJETU\\%
\\%
Pjesnici su čuđenje u svijetu\\%
\\%
Oni idu zemljom i njihove oči\\%
velike i nijeme rastu pored stvari\\%
\\%
Naslonivši uho\\%
\color{blue}{na ćutanje sto ih okružuje i muči\\%
pjesnici su vječno treptanje u svijetu}}
\vspace{5mm}

\dots{}i bio je zadovoljan i s jednim i s drugim riješenjem.
Odlučio je izmjene iz varijante \verb+varijanta+ preuzeti u \verb+master+.
Nakon \verb+get checkout master+ i \verb+git merge varijanta+, rezultat je bio:

\input{graphs/ab_simic_2}

\dots{}odnosno, pjesnikovim riječima:

\vspace{5mm}
\noindent\texttt{%
\textcolor{blue}{PJESNICI}\\%
\\%
Pjesnici su čuđenje u svijetu\\%
\\%
Oni idu zemljom i njihove oči\\%
velike i nijeme rastu pored stvari\\%
\\%
Naslonivši uho\\%
\color{blue}{na ćutanje sto ih okružuje i muči\\%
pjesnici su vječno treptanje u svijetu}}
\vspace{5mm}

I to je jednostavno.
U obje grane je mijenjao istu datoteku, ali u jednoj je dirao početak, a u drugoj kraj.
I rezultat \emph{merge}anja je bio očekivan -- datoteka u kojoj je izmijenjen i početak i kraj.

\section*{Konflikti}
\addcontentsline{toc}{section}{Konflikti}

Kao što slutite, stvar nije uvijek tako jednostavna.

\section*{Merge, branch i povijest projekta}
\addcontentsline{toc}{section}{Merge, branch i povijest projekta}

\section*{Fast forward}
\addcontentsline{toc}{section}{Fast forward}

\section*{Rebase}
\addcontentsline{toc}{section}{Rebase}

% \begin{itemize}
%    \item mergeanje znači mergeanje svih izmjena u povijesti, sve od mjesta gdje su se dvije grane bifucirale
%    \item izuzetak je git cherry-pick (grafovi s obojanim čvorovima)
%    \item --no-ff --no-commit
%    \item rebase
%    \item cherry-pick
%    \item preuzimanje samo jednog fajla iz drugog brancha
%    \item kreiranje privremenog brancha za eksperimentalni merge
% \end{itemize}

%\section*{}
%\addcontentsline{toc}{section}{}

