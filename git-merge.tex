\chapter*{Preuzimanje izmjena iz jedne grane u drugu}
\addcontentsline{toc}{chapter}{Preuzimanje izmjena iz jedne grane u drugu}

Vratimo se opet na jednu viđenu ilustraciju:

\input{graphs/git_merge_1}

Da ponovoimo, prebacivanjem na granu \verb+master+, izmjene koje ste napravili u \emph{commit}ovima \emph d, \emph e, \emph f i \emph g vam neće biti dostupne.
Istko, kad se prebacite na \verb+eksperimentalna-grana+ -- neće vam biti dostupne izmjene iz \emph x, \emph y i \emph z.

To je sve divno i krasno dok svoje izmjene želite raditi u relativnoj izolaciji od ostatka izmjena u kodu. 
No, što ako ste u \verb+master+ riješili neki gadan bug i htjeli biste sad tu ispravku preuzeti u \verb+eksperimentalna-grana+.

Ili, što ako zaključite kako je rezultat eksperimenta kojeg ste isprobali u \verb+eksperimentalna-grana+ uspješan i želite to sad imati u \verb+master+?
Ono što vam sad treba je da nekako \emph{izmjene iz jedne grane preuzmete u drugu granu}.
U gitu, to se naziva \textbf{merge}.
Iako bi merge mogli doslovno prevesti kao "spajanje", to nije ispravna riječ. 
Rezultat spajanja bi bila samo jedna grana. 
No, \emph{merge}anjem dvije grane -- one nastavljaju svoj život. 
Jedino što se sve izmjene koje su do tog trenutka rađene u jednoj granu preuzimaju u drugoj grani.

\section*{Git merge}
\addcontentsline{toc}{section}{Git merge}

Pretpostavimo, na primjer, da sve izmjene iz \verb+eksperimentalna-grana+ želite u \verb+master+. 
To se radi s naredbom \verb+git merge+.
Zapamtite, da bi to napravili kako treba, trebate u onoj grani u koju želite preuzeti izmjene (u našem slučaju \verb+master+), i onda:

\input{git_output/git_merge_1}

Ovaj ispis je ukratko opisan rezultat procesa preuzimanja izmjena; koliko je linija dodano, koliko obrisano, koliko je datoteka dodano, koliko oduzeto, i tako dalje\dots

Grafički se \verb+git merge+ može prikazati ovako:

\input{graphs/git_merge_2}

Neka bude jasno -- kad ste preuzeli izmjene iz jednog grafa u drugi, nitko vas ne prisiljava da onaj prvi obrišete. 
I dalje možete uredno u njega \emph{commit}ati.

TODO

% \begin{itemize}
%    \item mergeanje znači mergeanje svih izmjena u povijesti, sve od mjesta gdje su se dvije grane bifucirale
%    \item izuzetak je git cherry-pick (grafovi s obojanim čvorovima)
%    \item --no-ff --no-commit
%    \item rebase
%    \item cherry-pick
%    \item preuzimanje samo jednog fajla iz drugog brancha
%    \item kreiranje privremenog brancha za eksperimentalni merge
% \end{itemize}

%\section*{}
%\addcontentsline{toc}{section}{}

