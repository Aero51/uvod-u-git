\chapter*{Dodaci}
\addcontentsline{toc}{chapter}{Dodaci}

\section*{Git hosting}
\addcontentsline{toc}{section}{Git hosting}

Projekt na kojem radi samo jedna osoba je jednostavno organizirati. 
Ne trebaju udaljeni repozitoriji.
Dovoljno je jedno računalo i neki mehanizam snimanja sigurnosnih kopija \verb+.git+ direktorija.
Radimo li s drugim programerima ili možda imamo ambiciju kod našeg projekta pokazati svijetu -- tada nam treba repozitorij na nekom vidljivijem mjestu.

Prvo što se moramo odlučiti je -- hoće li taj repozitorij biti na našem serveru ili ćemo ga \emph{host}ati na nekom od postojećih javnih servisa.
Ukoliko je u pitanju ovo drugo, to je jednostavno.
Većina ozbiljnih servisa za verzioniranje koda podržava git.
Ovdje ću samo nabrojati neke od najpopularnijih:

\begin{itemize}
    \item GitHub (http://github.com) -- besplatan za projekte otvorenog koda, košta za privatne projekte (cijena ovisna o broju repozitorija i programera). Najpopularniji, brz i pregledan.
    \item BitBucket (http://bitbucket.org/) -- besplatan čak i za privatne repozitorije, malo manje popularan. U početku je bio zamišljen samo za projekte na mercurialu, ali sad nudi mercurial i git.
    \item Google Code (http://code.google.com) -- također ima mogućnost hostanja na gitu. Samo za projekte otvorenog koda.
    \item Sourceforge (http://sourceforge.net) -- jedan od najstarijih takvih servisa. Isključivo za projekte otvorenog koda.
    \item codeplex (http://www.codeplex.com) -- Microsoftova platforma za projekte otvorenog koda. Iako oni "guraju" TFS -- vjerojatno im je postalo očito da je git danas \emph{de facto} standard za tvoreni kod.
\end{itemize}

Za privatne repozitorije s više članova, moja preporuka je da platite tih par dolara Githubu ili BitBucketu. 
Osim što dobijete vrhunsku uslugu -- tim novcem implicitno subvencionirate hosting svim ostalim projektima otvorenog koda koji su hostani tamo.

\section*{Vlastiti server}
\addcontentsline{toc}{section}{Vlastiti server}

Druga varijanta, umjesto hostinga je koristiti vlastiti server.
Najjednostavniji scenarij je da jednostavno korisimo ssh protokol. 
Git uredno tako funkcionira.
Dakle, sve što trebamo učiniti je inicirati neki udaljeni git repozitorij i koristiti ga kao remote.

Ako je naziv servera \verb+server.com+, korisničko ime s kojim se prijavljujemo \verb+git+, a direktorij s repozitorijem \verb+projekt/+, onda ga možemo početi koristiti s:

\gitoutputcommand{git remote add moj-repozitorij git@server.com:projekt}

To će vjerojatno biti dovoljno za jednog korisnika, no ima nekih nedostataka.

Ukoliko želimo još nekome dati mogućnost da \emph{push}a ili \emph{fetch}a na/s našeg repozitorija. 
Moramo mu dati i sve potrebne podatke da bi ostvario ssh konekciju na naš server.
Ukoliko to učinimo, on se može povezati \emph{ssh}om i listati datoteke na serveru, gledati koji servisi su startani, i sl.
To ponekad ne želimo.

Drugi problem je što ne možemo jednostavno nekome dati mogućnost da \emph{fetch}a i \emph{push}, a nekome drugome da samo \emph{fetch}a.
Ako smo dali ssh pristup -- onda je on punopravan korisnik na tom serveru i ima iste ovlasti kao i bilo tko drugi tko se može prijaviti kao taj korisnik.

\subsection*{Git shell}
\addcontentsline{toc}{subsection}{Git shell}

\emph{Git shell} riješava prvi od dva prethodno spomenuta problema. 
Kao što (pretpostavljam) znamo, na svakom UNIXoidnom operativnom sustavu korisnici imaju definiran \emph{shell}, odnosno nekakav program u kojem on može izvršavati naredbe.

\emph{Git shell} je posebna vrsta takvog \emph{shell}a koja korisniku \textbf{omogućuje ssh pristup, ali i korištenje samo određenom broja naredbi}.
Postupak je jednostavan, treba kreirati novog korisnika (u primjeru koji slijedi, to je korisnik \verb+git+).
Naredbom:

\gitoutputcommand{chsh -s /usr/bin/git-shell git}

\dots{}mu se početni \emph{shell} mijenja u \verb+git-shell+.
I sad u njegovom \emph{home} direktoriju treba kreirati direktorij \verb+git-shell-commands+ koji sadrži samo one komande koje će se ssh-om moći izvršavati.
Neke distribucije linuxa će već imati predložak takvog direktorija kojeg treba samo kopirati i dati prava za izvršavanje datotekama.
Na primjer:

\gitoutputcommand{cp -R /usr/share/doc/git/contrib/git-shell-commands /home/git/\\
chmod +x /home/git/git-shell-commands/help\\
chmod +x /home/git/git-shell-commands/list}

Sad, ako se netko (tko ima ovlasti) pokuša spojiti s \emph{ssh}om, moći će izvršavati samo \verb+help+ i \verb+list+ naredbe.

Ovakav pristup ne riješava problem ovlasti čitanja/pisanja nad repozitorijima, on vam samo omogućuje da ne dajete prava klasičnog korisnika na sustavu.

\subsection*{Certifikati}
\addcontentsline{toc}{subsection}{Certifikati}

S obzirom da je najjednostavniji način da se git koristi preko ssh, praktično je podesiti certifikate na lokalnom/udaljenom računalu tako da ne moramo svaki put tipkati lozinku.
To se može tako da naš javni ssh certifikat kopiramo na udaljeno računalo.

U svojem \emph{home} direktoriju bi trebali imati \verb+.ssh+ direktorij.
Ukoliko nije tamo, naredba:

\gitoutputcommand{ssh-keygen -t dsa}

\dots{}će ga kreirati zajedno s javnim certifikatom \verb+id_rsa.pub+.
Kopirajte sadržaj te datoteke u \verb+~/.ssh/authorized_keys+ na udaljenom računalu.

Ako je sve prošlo bez problema, korištenje gita preko ssh će od sad na dalje ići bez upita za lozinku za svaki \emph{push}, \emph{fetch} i \emph{pull}.

\section*{Git pluginovi}
\addcontentsline{toc}{section}{Git pluginovi}

Ukoliko vam se učini da je skup naredbi koje možemo dobiti s \verb+git <naredba>+ limitiran -- lako je dodati nove.
Recimo da trebamo naredbu \verb+git gladan-sam+\footnote{Irelevantno što bi ta naredba radila, ali daje naslutiti moje mentalno stanje u trenutku dok sam ovo pisao :)}.
Sve što treba je snimiti negdje izvršivu datoteku \verb+git-gladan-sam+ i potruditi se da je dostupna u komandnoj liniji.

Na unixoidnim računalima, to bi izgledalo ovako nekako:

\gitoutputcommand{mkdir moj-git-plugin\\
cd moj-git-plugin\\
touch git-gladan-sam\\
\# Tu bi sad trebalo editirati skriptu git-gladan-sam...\\
chmod +x git-gladan-sam\\
export PATH=\$PATH:\textasciitilde{}/moj-git-plugin}

Ovu zadnju liniju ćete, vjerojatno, dodati u neku inicijalizacijsku skriptu (\verb+.bashrc+, isl.) tako da bude dostupna i nakon restarta računala.

%\section*{Git workflow}
%\addcontentsline{toc}{section}{Git workflow}
%
%\TODO

\section*{Git i Mercurial}
\addcontentsline{toc}{section}{Git i Mercurial}

Mercurial je distribuirani sustav za verzioniranje sličan gitu.
S obzirom da su nastali u isto vrijeme i bili pod utjecajem jedan drugog, imaju slične funkcionalnosti.
Postoji i \emph{plugin} koji omogućuje da naredbe iz jednog koristite u radu s drugim\footnote{http://hg-git.github.com}.

Mercurial ima malo konzistentnije imenovane naredbe, ali isto tako i manji broj korisnika.
Međutim, ukoliko vam je git neintuitivan, mercurial bi trebao biti prirodna alternativa (naravno, ukoliko uopće želite distribuirani sustav).

Ovdje ćemo proći samo nekoliko osnovnih naredbi u mercurialu, tek toliko da steknete osjećaj o tome kako je s njime raditi:

Inicijalizacija repozitorija:

\gitoutputcommand{hg init}

Dodavanje datoteke \verb+README.txt+ u prostor predviđen za sljedeći \emph{commit} (ono što je u gitu \emph{index}):

\gitoutputcommand{hg add README.txt}

Micanje datoteke iz istog:

\gitoutputcommand{hg forget README.txt}

\emph{Commit}:

\gitoutputcommand{hg commit}

Trenutni status repozitorija:

\gitoutputcommand{hg status}

Izmjene u odnosu na repozitorij:

\gitoutputcommand{hg diff}

Kopiranje i izmjenu datoteka je poželjno raditi direktno iz mercuriala:

\gitoutputcommand{hg mv datoteka1 datoteka2\\hg cp datoteka3 datoteka 4}

Povijest repozitorija:

\gitoutputcommand{hg log}

"Vraćanje" na neku reviziju (\emph{commit}) u povijesti (za reviziju "1"):

\gitoutputcommand{hg update 1}

Vraćanje na zadnju reviziju:

\gitoutputcommand{hg update tip}

Pregled svih trenutnih grana:

\gitoutputcommand{hg branches}

Kreiranje nove grane:

\gitoutputcommand{hg branch nova\_grana}

Grana će biti stvarno i stvorena tek nakon prvog \emph{commit}a.

Jedna razlika između grana u mercurialu i gitu je što su u prvome grane permanentne.
Grane mogu biti aktivne i neaktivne, ali u principu one ostaju u repozitoriju.

Glavna grana (ono što je u gitu \verb+master+) je ovdje \verb+default+.

Prebacivanje s grane na granu:

\gitoutputcommand{hg checkout naziv\_grane}

\emph{Merge}anje grana:

\gitoutputcommand{hg merge naziv\_grane}

Pomoć:

\gitoutputcommand{hg help}

Za objašnjenje mercurialove terminologije:

\gitoutputcommand{hg help glossary}

%\section*{}
%\addcontentsline{toc}{section}{}
