\chapter*{Dodaci}
\addcontentsline{toc}{chapter}{Dodaci}

\section*{Git hosting}
\addcontentsline{toc}{section}{Git hosting}

Projekt na kojem radi samo jedna osoba je jednostavno organizirati. 
Ne trebaju udaljeni repozitoriji.
Dovoljno je jedno računalo i eventualno neki mehanizam snimanja sigurnosnih kopija \verb+.git+ direktorija.
Radimo li s drugim programerima ili možda imamo ambiciju kod našeg projekta pokazati svijetu -- tada nam treba repozitorij na nekom vidljivijem mjestu.

Prvo što se moramo odlučiti je -- hoće li taj repozitorij biti na našem serveru ili ćemo ga \emph{host}ati na nekom od postojećih javnih servisa.
Ukoliko je u pitanju ovo drugo, to je jednostavno.
Većina ozbiljnih servisa za verzioniranje koda podržava git.
Ovdje ću samo nabrojati neke od najpopularnijih:

\begin{itemize}
    \item GitHub (http://github.com) -- besplatan za projekte otvorenog koda, košta za privatne projekte (cijena ovisna o broju repozitorija i programera). Najpopularniji, brz i pregledan.
    \item BitBucket (http://bitbucket.org/) -- besplatan čak i za privatne repozitorije, malo manje popularan. U početku je bio zamišljen samo za projekte na mercurialu, ali sad nudi mercurial i git.
    \item Google Code (http://code.google.com) -- također ima mogućnost hostanja na gitu. Samo za projekte otvorenog koda.
    \item Sourceforge (http://sourceforge.net) -- jedan od najstarijih takvih servisa. Isključivo za projekte otvorenog koda.
    \item codeplex (http://www.codeplex.com) -- Microsoftova platforma za projekte otvorenog koda. Iako oni "guraju" TFS -- vjerojatno im je postalo očito da je git danas \emph{de facto} standard za tvoreni kod.
\end{itemize}

Za privatne repozitorije s više članova, moja preporuka je da platite tih par dolara Githubu ili BitBucketu. 
Osim što dobijete vrhunsku uslugu -- tim novcem implicitno subvencionirate hosting svim ostalim projektima otvorenog koda koji su hostani tamo.

\section*{Rad s certifikatima}
\addcontentsline{toc}{section}{Rad s certifikatima}

\TODO

\section*{Vlastiti server}
\addcontentsline{toc}{section}{Vlastiti server}

\TODO

\subsection*{Git shell}
\addcontentsline{toc}{subsection}{Git shell}

\gitoutputcommand{cp -R /usr/share/doc/git/contrib/git-shell-commands /home/git/\\
chmod +x /home/git/git-shell-commands/help\\
chmod +x /home/git/git-shell-commands/list}

\section*{Git pluginovi}
\addcontentsline{toc}{section}{Git pluginovi}

\TODO

\section*{Git workflow}
\addcontentsline{toc}{section}{Git workflow}

\TODO

\section*{Git i Mercurial}
\addcontentsline{toc}{section}{Git i Mercurial}

Mercurial je distribuirani sustav za verzioniranje sličan gitu.
S obzirom da su nastali u isto vrijeme i bili pod utjecajem jedan drugog, imaju slične funkcionalnosti.
Postoji i \emph{plugin} koji omogućuje da naredbe iz jednog koristite u radu s drugim\footnote{http://hg-git.github.com}.

Mercurial ima malo konzistentnije imenovane naredbe, ali isto tako i manji broj korisnika.
Međutim, ukoliko vam je git neintuitivan, mercurial bi trebao biti prirodna alternativa (naravno, ukoliko uopće želite distribuirani sustav).

Ovdje ćemo proći samo nekoliko osnovnih naredbi u mercurialu, tek toliko da steknete osjećaj o tome kako je s njime raditi:

Inicijalizacija repozitorija:

\gitoutputcommand{hg init}

Dodavanje datoteke \verb+README.txt+ u prostor predviđen za sljedeći \emph{commit} (ono što je u gitu \emph{index}):

\gitoutputcommand{hg add README.txt}

Micanje datoteke iz istog:

\gitoutputcommand{hg forget README.txt}

\emph{Commit}:

\gitoutputcommand{hg commit}

Trenutni status repozitorija:

\gitoutputcommand{hg status}

Izmjene u odnosu na repozitorij:

\gitoutputcommand{hg diff}

Kopiranje i izmjenu datoteka je poželjno raditi direktno iz mercuriala:

\gitoutputcommand{hg mv datoteka1 datoteka2\\hg cp datoteka3 datoteka 4}

Povijest repozitorija:

\gitoutputcommand{hg log}

"Vraćanje" na neku reviziju (\emph{commit}) u povijesti (za reviziju "1"):

\gitoutputcommand{hg update 1}

Vraćanje na zadnju reviziju:

\gitoutputcommand{hg update tip}

\TODO Mranchanje i mergeanje

%\section*{}
%\addcontentsline{toc}{section}{}
