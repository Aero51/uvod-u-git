\chapter*{Dodaci}
\addcontentsline{toc}{chapter}{Dodaci}

\TODO

\section*{Git hosting}
\addcontentsline{toc}{section}{Git hosting}

\section*{Rad s certifikatima}
\addcontentsline{toc}{section}{Rad s certifikatima}

\section*{Privatni server}
\addcontentsline{toc}{section}{Privatni server}

\subsection*{Git shell}
\addcontentsline{toc}{subsection}{Git shell}

\gitoutputcommand{cp -R /usr/share/doc/git/contrib/git-shell-commands /home/git/\\
chmod +x /home/git/git-shell-commands/help\\
chmod +x /home/git/git-shell-commands/list}

\section*{Git pluginovi}
\addcontentsline{toc}{section}{Git pluginovi}

\TODO

\section*{Git workflow}
\addcontentsline{toc}{section}{Git workflow}

\TODO

\section*{Git i Mercurial}
\addcontentsline{toc}{section}{Git i Mercurial}

Mercurial je distribuirani sustav za verzioniranje sličan gitu.
S obzirom da su nastali u isto vrijeme i bili pod utjecajem jedan drugog, imaju slične funkcionalnosti.
Postoji i \emph{plugin} koji omogućuje da naredbe iz jednog koristite u radu s drugim\footnote{http://hg-git.github.com}.

Mercurial ima malo konzistentnije imenovane naredbe, ali isto tako i manji broj korisnika.
Međutim, ukoliko vam je git neintuitivan, mercurial bi trebao biti prirodna alternativa (naravno, ukoliko uopće želite distribuirani sustav).

Ovdje ćemo proći samo nekoliko osnovnih naredbi u mercurialu, tek toliko da steknete osjećaj o tome kako je s njime raditi:

Inicijalizacija repozitorija:

\gitoutputcommand{hg init}

Dodavanje datoteke \verb+README.txt+ u prostor predviđen za sljedeći \emph{commit} (ono što je u gitu \emph{index}):

\gitoutputcommand{hg add README.txt}

Micanje datoteke iz istog:

\gitoutputcommand{hg forget README.txt}

\emph{Commit}:

\gitoutputcommand{hg commit}

Trenutni status repozitorija:

\gitoutputcommand{hg status}

Izmjene u odnosu na repozitorij:

\gitoutputcommand{hg diff}

Kopiranje i izmjenu datoteka je poželjno raditi direktno iz mercuriala:

\gitoutputcommand{hg mv datoteka1 datoteka2\\hg cp datoteka3 datoteka 4}

Povijest repozitorija:

\gitoutputcommand{hg log}

"Vraćanje" na neku reviziju (\emph{commit}) u povijesti (za reviziju "1"):

\gitoutputcommand{hg update 1}

Vraćanje na zadnju reviziju:

\gitoutputcommand{hg update tip}

\TODO Mranchanje i mergeanje

%\section*{}
%\addcontentsline{toc}{section}{}
