\tocChapter{Prikaz grana u git alatima}

Način kako su grafovi repozitorija prikazivani u ovoj knjizi nije isti kako ih prikazuju različiti aliati za rad s gitom.
Ovako mi se činilo za nijansu lakše za praćenje, ali da bi lakše radili s alatima kao što je \verb+gitk+ napraviti ćemo ovdje kratak uvod u to kako alati prikazuju povijest, $commit$ove i grane.

Osnovna razlika je u tome što grafički alati obično prikazuju povijest od dolje (starije) prema gore (novije). 
Druga bitna razlika je što neće \emph{commit}ove iz jedne grane prikazivati u posebnom retku (ili stupcu).

\tocSection{Prikaz lokalnih grana}

Situacija kojoj imamo samo \verb+master+ i onda stvorimo ovu granu u kojoj još nismo ništa $commit$ali:

\input{graphs/git_merge_01}

\dots{}u \verb+gitk+u će biti prikazano ovako:

\gitgraphics{images/gitk-001.png}{8cm}

Drugim riječime, \verb+master+ i \verb+eksperimentalna-grana+ pokazuju na isti $commit$.

Slična situacija:

\input{graphs/git_merge_02}

\dots{}će biti prikazana ovako:

\gitgraphics{images/gitk-002.png}{8cm}

Kao što vidite, u istom stupcu su prikazane obje grane, samo je prikazan zadnji $commit$ za svaku granu.

Ukoliko imate dvije grane u kojima se paralelno razvijao kod:

\input{graphs/git_merge_1}

\dots{}\verb+gitk+ će prikazati ovako:

\gitgraphics{images/gitk-003.png}{8cm}

\dots{}u principu slično, jedino što se alat trudi da pojedine $commit$ove prikazuje svakog u posebnom redu.
Ukoliko se s \verb+master+ prebacite na \verb+eksperimentalna-grana+, graf će biti isti jedino će se prikazati na kojoj ste točno grani:

\gitgraphics{images/gitk-004.png}{8cm}

Nakon ovakvog $merge$a:

\input{graphs/git_merge_2}

\dots{}graf je:

\gitgraphics{images/gitk-005.png}{8cm}

\tocSection{Prikaz grana udaljenog repozitorija}

Ukoliko imamo posla i s udaljenim repozitorijima, onda će njihove grane biti prikazane na istom grafu kao npr. \verb+remotes/origin/master+ ili \verb+origin/grana+.

Recimo da nemamo ništa za $push$ati na na udaljeni repozitorij (nego čak imamo nešto za \textbf{preuzeti} iz njega):

\gitgraphics{images/origin_master_ispred_master.png}{5cm}

Iz ovoga je jasno da za naš \verb+master+ "zaostaje" za četiri $commit$a u odnosu na udaljeni \verb+master+.

Primjer gdje imamo dva $commit$a koje nismo $push$ali, a mogli bismo:

\gitgraphics{images/master_ispred_origin_master.png}{5cm}

Primjer gdje imamo tri $commit$a za $push$anje, ali trebamo prije toga preuzeti četiri $commit$a iz \verb+origin/master+ i $merge$ati ih u našu granu:

\gitgraphics{images/master_i_origin_master_divergirani.png}{5cm}

I, situacija u kojoj je lokalni \verb+master+ potpuno isti kao udaljeni \verb+origin/master+:

\gitgraphics{images/master_i_origin_master_isti.png}{6.5cm}

