\chapter*{Podstabla}
\addcontentsline{toc}{chapter}{Podstabla}

Kad imamo dva projekta, svi mi programeri prirodno težimo k tome da izvučemo zajednički kod u neki treći (\emph{utility}) projekt.
Tako da isti kod koristimo na više mjesta, a ne da svaki put izmišljamo toplu vodu.
Postoje mnogi načini kako se takvi projekti organiziraju.

Na primjer, ako radimo u javi, onda možete takve \emph{utility} projekte upakirati u \verb+.jar+ datoteke i koristiti ih u ostalim projektima.
U \emph{IDE}ovima\footnote{Integrated development environment -- alati kao Eclipse, NetBeans, Idea, Visual studio, \dots} možemo imati projekte koji referenciraju druge projekte, isl.

Git nudi jednu sličnu mogućnost koju možemo koristiti, a to su \emph{submodul}i.

Git \emph{submodul} je \textbf{referenca na točno određeni \emph{commit} nekog drugog projekta na nekom mjestu u direktoriju našeg projekta}.
I onda uz pomoć git-a možemo klonirati cijeli pomoćni projekt i postaviti ga na taj \emph{commit}.

U stvari\dots Zaboravite zadnje dvije rečenice. Idemo ispočetka!

Taman napisah gornje dvije rečenica, kad mi je došla vijest da git u zadnjoj verziji ima jedan noviji i bolji mehanizam koji će (nadam se) u potpunosti zamijeniti \emph{submodul}e.

\TODO Dodavanje

\TODO Kako je konfigurirano

\TODO Brisanje

\TODO Izmjene

\TODO Postupak kad kloniramo projekt koji ima submodule

%\section*{}
%\addcontentsline{toc}{section}{}

